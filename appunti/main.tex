\documentclass[a4paper,oneside]{book}
\usepackage[utf8]{inputenc}
\usepackage[italian]{babel}
\usepackage{amsmath,amssymb,amsfonts,amsthm}
\usepackage{mathtools} % dcases
\usepackage{marginnote}
\usepackage[top=3cm, bottom=3cm, heightrounded, marginparwidth=2.5cm, marginparsep=1.5cm]{geometry}
\usepackage{graphicx}
\graphicspath{ {images/} }
\usepackage[nottoc]{tocbibind}	% to generate Bibliography entry in toc
\usepackage{lipsum}
\usepackage{bm}
\usepackage{fancyhdr}
\usepackage{hyperref}	% enable hyperlinks to referenced elements
\hypersetup{
	colorlinks=true,
	linkcolor=cyan,
	citecolor=cyan,
	linkcolor=cyan
}
\usepackage{enumitem} % more control over lists: stackoverflow.com/a/4974583
\usepackage{indentfirst} % YES indent after section/chapter heading
\usepackage[Algoritmo]{algorithm}
\usepackage[noend]{algpseudocode} % algoritmi decenti, senza end
\usepackage[makeroom]{cancel} % strike math
\usepackage{nicefrac} % nice inline fractions

%%   CUSTOM THEOREM/DEF  %%

\theoremstyle{definition}
\newtheorem{definition}{Definizione}[section]

\theoremstyle{theorem}
\newtheorem{theorem}{Teorema}[section]

%%         END           %%

\title{Appunti di Algoritmi per l'Ingengeria}
\author{lezioni tenute da\\ Geppino Pucci \\ trascritte da \\ Pietro}
% a sto punto se fai un libro crea una pagina di titolo decente
\date{\today}

% magia
% \fancyhead{}
\makeatletter 
\pagestyle{fancy}
% \fancyhead[LE,RO]{\@title}
% \fancyfoot{}
% \fancyfoot[LE,RO]{\thepage}
% \fancyfoot[LO,RE]{Chapter \thechapter}
% \fancyfoot[CO,CE]{\@author}
\lhead{\@title}
\chead{}
\rhead{}
\lfoot{Chapter \thechapter}
\cfoot{\@author}
\rfoot{\thepage}
\makeatother 

%%   COMANDI    %%
\newcommand{\bpi}{\bm{\Pi}}
\newcommand{\bi}{I}
\newcommand{\bs}{S}

\newcommand{\mleq}{\overset{?}{\leq}}
\newcommand{\mgeq}{\overset{?}{\geq}}
\newcommand{\mles}{\overset{?}{<}}
\newcommand{\mges}{\overset{?}{>}}
\newcommand{\meq}{\overset{?}{=}}

\newcommand{\circola}[1]{\mbox{\textcircled{\footnotesize #1}}}

% numerazione sana degli algoritmi
\renewcommand{\thealgorithm}{\arabic{chapter}.\arabic{algorithm}} 

%% FINE COMANDI %%

\begin{document}

\pagestyle{plain}
\pagenumbering{gobble}

% \input{titlepage}
\maketitle
% titolo piu` serio in thesis-example

\cleardoublepage

\frontmatter % non esiste perche` e` un articolo e non un libro

\chapter*{Ringraziamenti}
\addcontentsline{toc}{chapter}{Ringraziamenti}
% Si ringrazia tanta gente

\section*{Riferimenti}

Il libro di riferimento è il CLRS \cite{Cormen:2009:IAT:1614191},
che è proprio un mattonazzo ben fatto.



\tableofcontents

\mainmatter

\pagestyle{fancy}


\chapter{Introduzione agli algoritmi}
\section{Introduzione}

Il corso vuole insegnare a progettare ed analizzare algoritmi efficienti.

\begin{description}
    \item{\textbf{Progetto}} ideare una strategia di risoluzione secondo paradigmi generali:
        \begin{itemize}
            \item divide and conquer
            \item dynamic programming
            \item greedy
        \end{itemize}
    \item{\textbf{Analisi}} si valuta la correttezza e la complessità degli algoritmi con prove matematiche
    \item{\textbf{Algoritmi}} si vedranno applicazioni notevoli, negli ambiti del calcolo matematico e la manipolazione di stringhe
    \item{\textbf{Efficienti}} la complessità degli algoritmi è legata alla teoria riguardante la NP-completezza dei problemi
\end{description}

\section{Problema computazionale}

\begin{definition}[Algoritmo]\label{def:alg}
    Un algoritmo è una procedura computazionale finita (terminante) e deterministica, specificata come una sequenza di passi elementari (istruzioni) estratte da un insieme standard associato a un modello computazionale (astrazione di un computer) che trasforma in maniera univoca un ingresso in un uscita.
\end{definition}


\begin{definition}[Problema computazionale]\label{def:probcomp}
    Un problema computazionale $\bpi{}$ è una relazione tra un insieme di istanze $\bi{}$ e un insieme di soluzioni $\bs{}$: $\bpi{} \subseteq \bi{} \times \bs{}$
    \\
    Un problema computazionale definisce la specifica astratta. % ma de che?
\end{definition}

Note:
% \begin{itemize}[noitemsep,topsep=0pt,parsep=0pt,partopsep=0pt]
\begin{itemize}[noitemsep,parsep=0pt,partopsep=0pt]
    \item[--] le seguenti notazioni sono usate in maniera equivalente: $i_1 \bpi{} s_1 \iff (i_1 , s_1 ) \in \bpi{}$
    \item[--] la relazione non è univoca, ad un'istanza possono essere associate più soluzioni
    \item[--] nella specifica astratta si assume esistano le soluzioni per ogni istanza: $\forall i \in \bi{},\exists s \in \bs{} | i \bpi{} s$
\end{itemize}

Un algoritmo $A_{\bpi{}}$ è un algoritmo per $\bpi{}$, ossia risolve $\bpi{}$ se, quando i suoi ingressi sono elementi di $\bi{}$, le sue uscite sono gli elementi di $\bs{}$ in relazione all'ingresso, formalmente:

$$ A: \bi{} \to \bs{} \quad \textrm{e} \quad A(i)=s \iff i \, \bpi{} \, s $$

L'algoritmo realizza in modo procedurale la relazione specificata in modo astratto dal problema computazionale, ossia risolve istanze di un problema computazionale. Più precisamente, un algoritmo è scritto per un modello di computazione, non lavora con istanze astratte ma con le loro codifiche.

\section{Analisi di correttezza e complessità}

\subsection{Analisi di correttezza}

Occorre provare il teorema di correttezza:
$$ \forall i \in \bi{} \: : \: i \, \bpi{} \, A(i) $$

\subsection{Analisi di complessità}

\subsubsection{Modello di costo}
Ad ogni istruzione elementare viene associato un costo. Scegliendo $c \in \mathbb{R} \cup \{ 0 \} $ l'analisi risulta eccessivamente complessa. I possibili costi vengono quindi limitati a $c \in \{ 0,1 \} $, assegnando $1$ solo alle operazioni che determinano effettivamente la complessità. Chiaramente vanno compiute scelte ragionevoli.

La complessità di eseguire l'algoritmo $A_{\bpi{}}$ su $i \in \bi{}$ si definisce come
$$t_{A_{\bpi{},i}}=\;somma\;dei\;costi\;associati\;alle\;istruzioni\;eseguite\;per\;ottenere\;A_{\bpi{}}(i)\;$$

\subsubsection{Taglia di un'istanza}
Calcolare la complessità per ogni singola istanza risulta nuovamente troppo complesso. L'insieme delle instanze viene quindi partizionato raggruppando tutte le istanze di taglia uguale.

\begin{definition}[Taglia di un'istanza]\label{def:taglia}
    La taglia di un'istanza è una misura intera, non negativa, ragionevole, della lunghezza associata a una qualche codifica ragionevole di una data istanza.\\
    $$|\cdot|:\bi{}\to \mathbb{N} \cup \{0\}$$
\end{definition}

Le istanze di taglia uguale vengono raccolte, partizionando l'insieme $\bi{}$
$$ \bi{}_{n} = \left\{ i \in \bi{} : |i|=n \right\} \quad n \geq 0$$ 

Si definisce una metrica sintetica, la funzione di complessità, che è associata alla taglia delle istanze.
$$ T_{A_{\bpi{}}} : \mathbb{N} \cup \{0\} \to \mathbb{R}^{+} \cup \{0\}$$

\textbf{Nota bene:} l'ingresso della funzione complessità deve essere intero.

\begin{description}
    \item{\textbf{Caso peggiore}} $$ T_{A_{\bpi{}}}^{WORST} (n) = \sup_{i \in \bi{}_{n}} \{t_{A_{\bpi{},i}} \} $$
    \item{\textbf{Caso migliore}} $$ T_{A_{\bpi{}}}^{BEST} (n) = \inf_{i \in \bi{}_{n}} \{t_{A_{\bpi{},i}} \} $$
    \item{\textbf{Caso medio}} $$ T_{A_{\bpi{}}}^{AVE} (n) = \mathbb{E} \{t_{A_{\bpi{},i}} \} $$
\end{description}
\subsubsection{autocomplete}

àg
èg
ùg


\chapter{Paradigma Divide and Conquer}
\section{Paradigma divide and conquer}

Seguendo il paradigma \textit{Divide and Conquer}, si cerca una soluzione ad una data istanza in funzione delle soluzioni a determinate istanze \textbf{più piccole}, dette sottoistanze.

\subsection{Stella di Kleene}
\begin{definition}[Stella di Kleene]\label{def:kleene}
    La stella di Kleene viene definita come l'insieme di tutte le sequenze finite di elementi di $A$.
    $$ A^* = \left\{ < a_1, a_2, \dots, a_k > \: : k \geq 0, \: a_i \in A, \: 1 \leq i \leq k \right\}$$
    La sequenza vuota viene indicata con $ < \; > \: = \epsilon $
\end{definition}

\subsection{Proprietà di sottostruttura}

La proprietà di sottostruttura è una proprietà del problema $\bpi{} \subseteq \bi{} \times \bs{} $ che permette di definire il divide and conquer. Devono esistere due funzioni per avere la proprietà di sottostruttura:

\begin{description}
    \item{Funzione di \textbf{Divisione}} $ D : \bi{} \to \bi{}^*$
    \item{Funzione di \textbf{Ricombinazione}} $ C : \bs{}^* \to \bs{}$
\end{description}

E devono valere le due seguenti proprietà:
\begin{itemize}
    \item $\exists n_0 \in \mathbb{N} \cup \{0\} : \forall i \in \bi{}, |i| > n_0 $
        \begin{itemize}
            \item[--] $ D(i)=<i_1, i_2, \dots, i_k>:|i_j|<|i|, 1 \leq j \leq k$ \\
                (le sottoistanze sono strettamente più piccole dell'istanza originale)
            \item[--] data $ <s_1, s_2, \dots, s_k> \in \bs{}^* $ con $i_j \bpi{} s_j,  1 \leq j \leq k \Rightarrow i \bpi{} C(<s_1, s_2, \dots, s_k>)$ \\
                (la ricombinazione delle soluzioni alle sottoistanze è in relazione con l'istanza originale)
        \end{itemize}
    \item è possibile la risoluzione diretta delle istanze $ i \in \bi{} : |i| \leq n_0 $
\end{itemize}

Note:
% \begin{itemize}
\begin{itemize}[noitemsep,parsep=0pt,partopsep=0pt]
    \item[--] il numero di sottoistanze può variare, in base all'istanza considerata
    \item[--] gli algoritmi che implementano (con sequenze di istruzioni elementari) le funzioni, sono, formalmente, oggetti matematici diversi \\
        $ A_D : i \mapsto <i_1, i_2, \dots, i_k> \sim D(i)$ e $ A_C : <s_1, s_2, \dots, s_k> \mapsto s \sim C(<s_1, s_2, \dots, s_k>) $
\end{itemize}

\subsubsection{Problema del Sorting di interi - definizione}

Introduciamo il problema dell'ordinamento di una sequenza di numeri interi, che verrà usato anche in seguito per esemplificare i concetti teorici esposti.

Come insiemi definiamo $\bi{} = \mathbb{Z}^* $ e $\bs{} = \mathbb{Z}^* $. Non tutti gli elementi di $\bs{}$ saranno associati ad un'istanza.

Quando $ (i,s) \in \bpi{}_{SORT} $ ? A parte il caso degenere $i=s=\epsilon$, deve esistere una corrispondenza biunivoca fra gli indici per cui
\begin{itemize}[noitemsep,parsep=0pt,partopsep=0pt]
    \item[--] $i = <i_1, i_2, \dots, i_k> $ e $ s = <s_1, s_2, \dots, s_k> $ con $k \geq 1$
    \item[--] $ \exists \;\phi:\{1, \dots, n\} \leftrightarrow \{1, \dots, n\} \;|\; s_i = i_{\phi(i)}$
    \item[--] $ s_1 \leq s_2 \leq \dots \leq s_k$ 
\end{itemize}

\subsection{Paradigma Divide and Conquer}

\begin{algorithm}[H]
\caption{Divide and Conquer}\label{alg:dnc}
\begin{algorithmic}[1]
    \Procedure{D\&C}{$i$}
        \If{$|i| \leq n_0$}                             \Comment{BASE}
            \State *risolvo direttamente*
        \EndIf
        \State $<i_1, i_2, \dots, i_k> \gets A_D(i)$    \Comment{DIVIDE}
        \For{$j \gets 1 $ to $ k $ }                    \Comment{RECURSE}
            \State $s_j \gets $ \Call{D\&C}{$i_j$}
        \EndFor
        \State $s \gets A_C(<s_1, s_2, \dots, s_k>)$    \Comment{CONQUER}
        \State return $s$
    \EndProcedure
\end{algorithmic}
\end{algorithm}

\subsubsection{Albero delle chiamate}
Nel corso del D\&C, si alternano chiamate all'algoritmo di Divide, e quindi una fase di espansione, generazione \textit{top-down} delle sottoistanze, e chiamate all'algoritmo di Conquer, associato a fasi di contrazione, risoluzione \textit{bottom-up}. L'albero non viene generato tutto contemporaneamente, ma viene creato (e distrutto) nel corso dell'algoritmo, seguendo il cammino di una visita anticipata (\textit{depth-first search}).

Il fatto che nella fase di Conquer venga eliminata la porzione di albero generato per risolvere una sottoistanza, è uno dei difetti principali di questo paradigma. Il non ricordare le soluzioni parziali trovate lo rende un processo computazionale \textit{memoryless}.

TODO pagina 4, albero delle chiamate, disegnini e grafichetti

\subsubsection{Esempio algoritmo D\&C, ricerca del massimo}

Come esempio di algoritmo D\&C, viene presentata una procedura che trova il massimo intero in una sequenza di interi $A[1 \dots n] \in \mathbb{Z}^*$

\begin{algorithm}[H]
\caption{Massimo}\label{alg:max}
\begin{algorithmic}[1]
    \Procedure{MAX}{$A,i,j$}                            
        \If{$i=j$}                             \Comment{BASE}
            \State return $A[i]$
        \EndIf
        \State $k \gets \left\lfloor \frac{i+j}{2} \right\rfloor$    \Comment{DIVIDE - punto di mezzo discreto}
        \State $m_1 \gets $ \Call{MAX}{$A,i,k$} \Comment{RECURSE}
        \State $m_1 \gets $ \Call{MAX}{$A,k+1,j$}
        \If{$m_1 \geq m_2$}              \Comment{CONQUER}
            \State return $m_1$
        \Else
            \State return $m_2$
        \EndIf
    \EndProcedure
\end{algorithmic}
\end{algorithm}

Nota: Nella firma della procedura, sono presenti tutti i parametri necessari a identificare la \textit{generica} sottostruttura

\subsection{Correttezza del DnC}

Induzione e magia

Relazione di ricorrenza

\subsection{Analisi della complessità}

\subsubsection{autocomplete}

Il \textit{merge}, cuore dell'algoritmo \ref{alg:mergesort}, avviene alla riga \ref{alg:mergeline}, quindi guarda a \algref{alg:mergesort}{alg:mergeline} per saperne di più.

\begin{algorithm}[h]
\caption{MergeSort}\label{alg:mergesort}
\begin{algorithmic}[1]
    \Procedure{MergeSort}{$A,p,r$}
    \If{$p<r$}
        \State $ q \gets \left\lfloor \frac{p+r}{2} \right\rfloor $ 
        \State \Call{MergeSort}{$A,p,q$} \Comment{Ordina prima}
        \State \Call{MergeSort}{$A,q+1,r$} \Comment{Ordina dopo}
        \State
            \Call{Merge}{$A,p,q,r$}
            \label{alg:mergeline}
            \Comment{Merge in $\Theta(n)$}
    \EndIf
    \EndProcedure
\end{algorithmic}
\end{algorithm}

àg
èg
òg
ùg


\chapter{Master Theorem}
\section{Contrazioni}
% definizione contrazioni, pag 13.5
\begin{definition}[Contrazione]\label{def:contraz}
    Una contrazione è una funzione $f: \mathbb{N} \rightarrow \mathbb{N}$, per cui vale
    \[ f(n) < n \quad \forall n>n_0 \]
\end{definition}

\subsection{Iterate}
% definizione iterata, pag 13.6
\begin{definition}[Iterata]\label{def:iterata}
    Ad una contrazione sono associate le iterate di $f(n)$
    \[
    \begin{cases} 
        f^{(0)} (n) = n      &  i = 0 \\
        f^{(i)} (n) = f \left( f^{(i-i)} (n) \right)   &  i > 0 \\
        
    \end{cases}
    \]
\end{definition}

Nota: le iterate formano una successione decrescente $ f^{(0)} (n) > f^{(1)} (n) > \cdots > f^{(i)} (n) $ \\
Nella prossima sezione, l'iterata $0$ sarà associata alla radice dell'abero e all'istanza generale, l'iterata $i-esima$ rappresenterà la taglia al livello $i$ dell'albero.

\subsection{Ausiliaria}
% Def ausiliaria, pag 14.6
\begin{definition}[Ausiliaria]\label{def:ausiliaria}
    Alle iterate di una funzione si associa anche una funzione ausiliaria, che indica il maggior indice di iterata per cui il valore è ancora maggiore del valore di base. Nell'albero delle ricorrenze, questo indicherà l'ultimo livello.
    \[ f^*(n, n_0) = \max \{ i>0 : f^{(i)}(n) > n_0 \} \]
    La funzione è definita solo per $n>n_0$, per convenzione assume il valore $f^*(n,n_0)=-1$ se $ n \leq n_0$
\end{definition}

\subsection{Esempi}
\subsubsection{$\bm{f(n) = n/2}$}
% pag 13.9, 14.8
Calcoliamo la forma esplicita dell'iterata e la funzione ausiliaria per
\[ f(n) = \frac{n}{2} \quad \text{con} \quad n=2^k \]
Forma esplicita iterata:
\begin{align*}
    & f^{(0)}(n) = n \\
    & f^{(1)}(n) = f(n) = \frac{n}{2} \\
    & f^{(2)}(n) = f \left( f^{(1)}(n) \right) = f \left( \frac{n}{2} \right) = \frac{n}{4} = \frac{n}{2^2} \\
    & f^{(i)}(n) = \frac{n}{2^i} 
\end{align*}
Dove la generalizzazione  è valida perché gli argomenti restano potenze di due: $n/2^i = 2^{k-i}$ \\
Calcolo funzione ausiliaria:
\begin{align*}
    & f^{(i)}(n) \mgeq n_0 
        & \text{per quali $i\:$?}\\
    & \frac{n}{2^i} \mges n_0 \\
    & 2^i \mles \frac{n}{n_0} \\
    & i < \log_2 \left( \frac{n}{n_0} \right)
        & \text{il vincolo è risolto} \\
    \rightarrow \quad & f^*(n, n_0) = \log_2 \left( \frac{n}{n_0} \right) - 1
        & \text{ne prendo il massimo} \\
    n_0 = 1 \rightarrow \quad & f^*(n, 1) = \log_2 \left( n \right) - 1
        & \text{se $n_0=1$}
\end{align*}

\subsubsection{$\bm{f(n) = n-1}$}
% pag 14, 15
Calcoliamo la forma esplicita dell'iterata e la funzione ausiliaria per
\[ f(n) = n-1 \]
Forma esplicita iterata:
\begin{align*}
    & f^{(0)}(n) = n \\
    & f^{(1)}(n) = f(n) = n-1 \\
    & f^{(2)}(n) = f \left( f^{(1)}(n) \right) = f \left( n-1 \right) = n-1-1 = n-2 \\
    & f^{(i)}(n) = n-i
\end{align*}
Calcolo funzione ausiliaria:
\begin{align*}
    & f^{(i)}(n) \mgeq n_0 
        & \text{per quali $i\:$?}\\
    & n-i \mges n_0 \\
    & i < n - n_0
        & \text{il vincolo è risolto} \\
    \rightarrow \quad & f^*(n, n_0) = n - n_0 - 1
        & \text{ne prendo il massimo} \\
    n_0 = 1 \rightarrow \quad & f^*(n, 1) = n - 2
        & \text{se $n_0=1$}
\end{align*}

\subsubsection{$\bm{f(n) = \sqrt{n}}$}
% pag 14.2, 15.2
Calcoliamo la forma esplicita dell'iterata e la funzione ausiliaria per
\[ f(n) = \sqrt{n} \quad \text{con} \quad n=2^{2^k} \]
Forma esplicita iterata:
\begin{align*}
    & f^{(0)}(n) = n \\
    & f^{(1)}(n) = f(n) = \sqrt{n} = n^{\nicefrac{1}{2}} \\
    & f^{(2)}(n) = f \left( f^{(1)}(n) \right) = f \left( n^{\nicefrac{1}{2}} \right) = n^{\nicefrac{1}{2^2}} \\
    & f^{(i)}(n) = n^{\nicefrac{1}{2^i}} \\
\end{align*}
Calcolo funzione ausiliaria:
\begin{align*}
    & f^{(i)}(n) \mgeq n_0 
        & \text{per quali $i\:$?}\\
    & n^{\nicefrac{1}{2^i}} \mges n_0 \\
    & \log_2 n^{\nicefrac{1}{2^i}} \mges \log_2 n_0 \\
    & \frac{1}{2^i} \log_2 n \mges \log_2 n_0 \\
    & 2^i \mles \frac{\log_2 n}{\log_2 n_0} \\
    & i < \log_2 \frac{\log_2 n}{\log_2 n_0} 
        & \text{il vincolo è risolto} \\
    & i < \log_2 \log_2 n + \log_2 \log_2 n_0 \\
    \rightarrow \quad & f^*(n, n_0) = \log_2 \log_2 n + \log_2 \log_2 n_0 - 1
        & \text{ne prendo il massimo} \\
    n_0 = 2 \rightarrow \quad & f^*(n, 2) = \log_2 \log_2 n - 1
        & \text{se $n_0=2$}
\end{align*}
Il vincolo su $n$ può essere generalizzato a $n=a^{2^k}$, in questo caso $n_0 = a$

\subsubsection{$\bm{f(n) = \left\lfloor n/2 \right\rfloor}$}
Nel caso $f(n) = \left\lfloor n/2 \right\rfloor$, si può dimostrare che
$ f^{(2)}(n) = \left\lfloor \left\lfloor n/2 \right\rfloor / 2 \right\rfloor = \left\lfloor n / 2^2 \right\rfloor $
ma non è per niente banale.

\section{Modifica al \textit{Divide and Conquer}}
% titolo meno informativo della storia
\subsection{Metaalgoritmo}
Metaalgoritmo DnC modificato, pag 12.5

Parametri noti, pag 13

Nuova eq di ricorrenza, pag 13.3

\subsubsection{esempio parametri MAX}
parametri max, pag 13.5

\subsection{Equazione delle ricorrenze generica}
Alberone, pag 15.8

Formulone, pag 16

Da qualche parte le convenzioni sugli operatori, pag 15.5; insieme alle cose a fine capitolo DnC -> Appendice

\subsubsection{Esempio formula}
Esempio, pag 16.5

\subsubsection{Esempio formula}
Esempio, pag 17.5

\section{\textit{Master theorem}}
Ipotesi, pag 18

Tesi, pag 18.5

Dimostrazione, pag 20

Considerazioni sull'asintotico, pag 18.9, 19

\subsection{autocomplete}
Una bella scatola:
\begin{equation}
    \boxed{x^2+y^2 = z^2}
\end{equation}

àg
èg
ìg
òg
ùg
perché


\fancyfoot[LO,RE]{Bibliography}
\bibliographystyle{unsrt}
\bibliography{biblio}
\cleardoublepage

\appendix
\fancyfoot[LO,RE]{Appendix \thechapter}

\chapter{Appendice}
Appendice con info utili

Dovrebbe chiamarsi Appendice F


\end{document}


%% ci sono cose interessanti su fancyfoot in thesis-example
