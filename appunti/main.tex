\documentclass[a4paper,oneside]{book}
\usepackage[utf8]{inputenc}
\usepackage[italian]{babel}
\usepackage{amsmath,amssymb,amsfonts,amsthm}
\usepackage{mathtools} % dcases
\usepackage{marginnote}
\usepackage[top=3cm, bottom=3cm, heightrounded, marginparwidth=2.5cm, marginparsep=1.5cm]{geometry}
\usepackage{graphicx}
\graphicspath{ {images/} }
\usepackage[nottoc]{tocbibind}	% to generate Bibliography entry in toc
\usepackage{lipsum}
\usepackage{bm}
\usepackage{fancyhdr}
\usepackage{hyperref}	% enable hyperlinks to referenced elements
\hypersetup{
	colorlinks=true,
	linkcolor=cyan,
	citecolor=cyan,
	linkcolor=cyan
}
\usepackage{enumitem} % more control over lists: stackoverflow.com/a/4974583
\usepackage{indentfirst} % YES indent after section/chapter heading
\usepackage[Algoritmo]{algorithm}
\usepackage[noend]{algpseudocode} % algoritmi decenti, senza end
\usepackage[makeroom]{cancel} % strike math
\usepackage{nicefrac} % nice inline fractions
\allowdisplaybreaks % allow breaking equations over pages tex.stackexchange.com/a/431236
\usepackage{cases} % numbering eq inside case env
\usepackage{upquote} % display '` in verbatim

\usepackage{tikz} % for tikzpicture
\usetikzlibrary{shapes} % node shapes


%%   CUSTOM THEOREM/DEF  %%

\theoremstyle{definition}
\newtheorem{definition}{Definizione}[section]

\theoremstyle{theorem}
\newtheorem{theorem}{Teorema}[section]

\theoremstyle{theorem}
\newtheorem{lemma}[theorem]{Lemma}
% \newtheorem{lemma}{Lemma}[section]

%%         END           %%

\title{Appunti di Algoritmi per l'Ingengeria}
\author{lezioni tenute da\\ Geppino Pucci \\ trascritte da \\ Pietro}
% a sto punto se fai un libro crea una pagina di titolo decente
\date{\today}

% magia
% \fancyhead{}
\makeatletter 
\pagestyle{fancy}
% \fancyhead[LE,RO]{\@title}
% \fancyfoot{}
% \fancyfoot[LE,RO]{\thepage}
% \fancyfoot[LO,RE]{Chapter \thechapter}
% \fancyfoot[CO,CE]{\@author}
\lhead{\@title}
\chead{}
\rhead{}
\lfoot{Capitolo \thechapter}
\cfoot{\@author}
\rfoot{\thepage}
\makeatother 

%%   COMANDI    %%
\newcommand{\bpi}{\bm{\Pi}}
% \newcommand{\bi}{I}
\newcommand{\bi}{\mathcal{I}}
\newcommand{\bs}{\mathcal{S}}
\newcommand{\ba}{S}
\newcommand{\bb}{S}
\newcommand{\bc}{S}

\newcommand{\mleq}{\overset{?}{\leq}}
\newcommand{\mgeq}{\overset{?}{\geq}}
\newcommand{\mles}{\overset{?}{<}}
\newcommand{\mges}{\overset{?}{>}}
\newcommand{\meq}{\overset{?}{=}}

\newcommand{\circola}[1]{\mbox{\textcircled{\footnotesize #1}}}
\newcommand{\labeq}[1]{\overset{\circola{#1}}{=}}

\newcommand{\rfig}[1]{Figura~\ref{#1}}
%% MATH OPERATOR %%

\DeclareMathOperator*{\argmin}{arg\,min}

% numerazione sana degli algoritmi
\renewcommand{\thealgorithm}{\arabic{chapter}.\arabic{algorithm}} 

%% FINE COMANDI %%

\begin{document}

\pagestyle{plain}
\pagenumbering{gobble}

% \input{titlepage}
\maketitle
% titolo più serio in thesis-example

\cleardoublepage

\frontmatter % non esiste perche` e` un articolo e non un libro

\chapter*{Ringraziamenti}
\addcontentsline{toc}{chapter}{Ringraziamenti}
% Si ringrazia tanta gente

\section*{Riferimenti}

Il libro di riferimento è il CLRS \cite{Cormen:2009:IAT:1614191},
che è proprio un mattonazzo ben fatto.


\tableofcontents

\mainmatter

\pagestyle{fancy}

\chapter{Introduzione agli algoritmi}
\section{Introduzione}

Il corso vuole insegnare a progettare ed analizzare algoritmi efficienti.

\begin{description}
    \item{\textbf{Progetto}} ideare una strategia di risoluzione secondo paradigmi generali:
        \begin{itemize}
            \item divide and conquer
            \item dynamic programming
            \item greedy
        \end{itemize}
    \item{\textbf{Analisi}} si valuta la correttezza e la complessità degli algoritmi con prove matematiche
    \item{\textbf{Algoritmi}} si vedranno applicazioni notevoli, negli ambiti del calcolo matematico e la manipolazione di stringhe
    \item{\textbf{Efficienti}} la complessità degli algoritmi è legata alla teoria riguardante la NP-completezza dei problemi
\end{description}

\section{Problema computazionale}

\begin{definition}[Algoritmo]\label{def:alg}
    Un algoritmo è una procedura computazionale finita (terminante) e deterministica, specificata come una sequenza di passi elementari (istruzioni) estratte da un insieme standard associato a un modello computazionale (astrazione di un computer) che trasforma in maniera univoca un ingresso in un uscita.
\end{definition}


\begin{definition}[Problema computazionale]\label{def:probcomp}
    Un problema computazionale $\bpi{}$ è una relazione tra un insieme di istanze $\bi{}$ e un insieme di soluzioni $\bs{}$: $\bpi{} \subseteq \bi{} \times \bs{}$
    \\
    Un problema computazionale definisce la specifica astratta. % ma de che?
\end{definition}

Note:
% \begin{itemize}[noitemsep,topsep=0pt,parsep=0pt,partopsep=0pt]
\begin{itemize}[noitemsep,parsep=0pt,partopsep=0pt]
    \item[--] le seguenti notazioni sono usate in maniera equivalente: $i_1 \bpi{} s_1 \iff (i_1 , s_1 ) \in \bpi{}$
    \item[--] la relazione non è univoca, ad un'istanza possono essere associate più soluzioni
    \item[--] nella specifica astratta si assume esistano le soluzioni per ogni istanza: $\forall i \in \bi{},\exists s \in \bs{} | i \bpi{} s$
\end{itemize}

Un algoritmo $A_{\bpi{}}$ è un algoritmo per $\bpi{}$, ossia risolve $\bpi{}$ se, quando i suoi ingressi sono elementi di $\bi{}$, le sue uscite sono gli elementi di $\bs{}$ in relazione all'ingresso, formalmente:

$$ A: \bi{} \to \bs{} \quad \textrm{e} \quad A(i)=s \iff i \, \bpi{} \, s $$

L'algoritmo realizza in modo procedurale la relazione specificata in modo astratto dal problema computazionale, ossia risolve istanze di un problema computazionale. Più precisamente, un algoritmo è scritto per un modello di computazione, non lavora con istanze astratte ma con le loro codifiche.

\section{Analisi di correttezza e complessità}

\subsection{Analisi di correttezza}

Occorre provare il teorema di correttezza:
$$ \forall i \in \bi{} \: : \: i \, \bpi{} \, A(i) $$

\subsection{Analisi di complessità}

\subsubsection{Modello di costo}
Ad ogni istruzione elementare viene associato un costo. Scegliendo $c \in \mathbb{R} \cup \{ 0 \} $ l'analisi risulta eccessivamente complessa. I possibili costi vengono quindi limitati a $c \in \{ 0,1 \} $, assegnando $1$ solo alle operazioni che determinano effettivamente la complessità. Chiaramente vanno compiute scelte ragionevoli.

La complessità di eseguire l'algoritmo $A_{\bpi{}}$ su $i \in \bi{}$ si definisce come
$$t_{A_{\bpi{},i}}=\;somma\;dei\;costi\;associati\;alle\;istruzioni\;eseguite\;per\;ottenere\;A_{\bpi{}}(i)\;$$

\subsubsection{Taglia di un'istanza}
Calcolare la complessità per ogni singola istanza risulta nuovamente troppo complesso. L'insieme delle instanze viene quindi partizionato raggruppando tutte le istanze di taglia uguale.

\begin{definition}[Taglia di un'istanza]\label{def:taglia}
    La taglia di un'istanza è una misura intera, non negativa, ragionevole, della lunghezza associata a una qualche codifica ragionevole di una data istanza.\\
    $$|\cdot|:\bi{}\to \mathbb{N} \cup \{0\}$$
\end{definition}

Le istanze di taglia uguale vengono raccolte, partizionando l'insieme $\bi{}$
$$ \bi{}_{n} = \left\{ i \in \bi{} : |i|=n \right\} \quad n \geq 0$$ 

Si definisce una metrica sintetica, la funzione di complessità, che è associata alla taglia delle istanze.
$$ T_{A_{\bpi{}}} : \mathbb{N} \cup \{0\} \to \mathbb{R}^{+} \cup \{0\}$$

\textbf{Nota bene:} l'ingresso della funzione complessità deve essere intero.

\begin{description}
    \item{\textbf{Caso peggiore}} $$ T_{A_{\bpi{}}}^{WORST} (n) = \sup_{i \in \bi{}_{n}} \{t_{A_{\bpi{},i}} \} $$
    \item{\textbf{Caso migliore}} $$ T_{A_{\bpi{}}}^{BEST} (n) = \inf_{i \in \bi{}_{n}} \{t_{A_{\bpi{},i}} \} $$
    \item{\textbf{Caso medio}} $$ T_{A_{\bpi{}}}^{AVE} (n) = \mathbb{E} \{t_{A_{\bpi{},i}} \} $$
\end{description}
\subsubsection{autocomplete}

àg
èg
ùg


\chapter{Paradigma Divide and Conquer}
\section{Paradigma divide and conquer}

Seguendo il paradigma \textit{Divide and Conquer}, si cerca una soluzione ad una data istanza in funzione delle soluzioni a determinate istanze \textbf{più piccole}, dette sottoistanze.

\subsection{Stella di Kleene}
\begin{definition}[Stella di Kleene]\label{def:kleene}
    La stella di Kleene viene definita come l'insieme di tutte le sequenze finite di elementi di $A$.
    $$ A^* = \left\{ < a_1, a_2, \dots, a_k > \: : k \geq 0, \: a_i \in A, \: 1 \leq i \leq k \right\}$$
    La sequenza vuota viene indicata con $ < \; > \: = \epsilon $
\end{definition}

\subsection{Proprietà di sottostruttura}

La proprietà di sottostruttura è una proprietà del problema $\bpi{} \subseteq \bi{} \times \bs{} $ che permette di definire il divide and conquer. Devono esistere due funzioni per avere la proprietà di sottostruttura:

\begin{description}
    \item{Funzione di \textbf{Divisione}} $ D : \bi{} \to \bi{}^*$
    \item{Funzione di \textbf{Ricombinazione}} $ C : \bs{}^* \to \bs{}$
\end{description}

E devono valere le due seguenti proprietà:
\begin{itemize}
    \item $\exists n_0 \in \mathbb{N} \cup \{0\} : \forall i \in \bi{}, |i| > n_0 $
        \begin{itemize}
            \item[--] $ D(i)=<i_1, i_2, \dots, i_k>:|i_j|<|i|, 1 \leq j \leq k$ \\
                (le sottoistanze sono strettamente più piccole dell'istanza originale)
            \item[--] data $ <s_1, s_2, \dots, s_k> \in \bs{}^* $ con $i_j \bpi{} s_j,  1 \leq j \leq k \Rightarrow i \bpi{} C(<s_1, s_2, \dots, s_k>)$ \\
                (la ricombinazione delle soluzioni alle sottoistanze è in relazione con l'istanza originale)
        \end{itemize}
    \item è possibile la risoluzione diretta delle istanze $ i \in \bi{} : |i| \leq n_0 $
\end{itemize}

Note:
% \begin{itemize}
\begin{itemize}[noitemsep,parsep=0pt,partopsep=0pt]
    \item[--] il numero di sottoistanze può variare, in base all'istanza considerata
    \item[--] gli algoritmi che implementano (con sequenze di istruzioni elementari) le funzioni, sono, formalmente, oggetti matematici diversi \\
        $ A_D : i \mapsto <i_1, i_2, \dots, i_k> \sim D(i)$ e $ A_C : <s_1, s_2, \dots, s_k> \mapsto s \sim C(<s_1, s_2, \dots, s_k>) $
\end{itemize}

\subsubsection{Problema del Sorting di interi - definizione}

Introduciamo il problema dell'ordinamento di una sequenza di numeri interi, che verrà usato anche in seguito per esemplificare i concetti teorici esposti.

Come insiemi definiamo $\bi{} = \mathbb{Z}^* $ e $\bs{} = \mathbb{Z}^* $. Non tutti gli elementi di $\bs{}$ saranno associati ad un'istanza.

Quando $ (i,s) \in \bpi{}_{SORT} $ ? A parte il caso degenere $i=s=\epsilon$, deve esistere una corrispondenza biunivoca fra gli indici per cui
\begin{itemize}[noitemsep,parsep=0pt,partopsep=0pt]
    \item[--] $i = <i_1, i_2, \dots, i_k> $ e $ s = <s_1, s_2, \dots, s_k> $ con $k \geq 1$
    \item[--] $ \exists \;\phi:\{1, \dots, n\} \leftrightarrow \{1, \dots, n\} \;|\; s_i = i_{\phi(i)}$
    \item[--] $ s_1 \leq s_2 \leq \dots \leq s_k$ 
\end{itemize}

\subsection{Paradigma Divide and Conquer}

\begin{algorithm}[H]
\caption{Divide and Conquer}\label{alg:dnc}
\begin{algorithmic}[1]
    \Procedure{D\&C}{$i$}
        \If{$|i| \leq n_0$}                             \Comment{BASE}
            \State *risolvo direttamente*
        \EndIf
        \State $<i_1, i_2, \dots, i_k> \gets A_D(i)$    \Comment{DIVIDE}
        \For{$j \gets 1 $ to $ k $ }                    \Comment{RECURSE}
            \State $s_j \gets $ \Call{D\&C}{$i_j$}
        \EndFor
        \State $s \gets A_C(<s_1, s_2, \dots, s_k>)$    \Comment{CONQUER}
        \State return $s$
    \EndProcedure
\end{algorithmic}
\end{algorithm}

\subsubsection{Albero delle chiamate}
Nel corso del D\&C, si alternano chiamate all'algoritmo di Divide, e quindi una fase di espansione, generazione \textit{top-down} delle sottoistanze, e chiamate all'algoritmo di Conquer, associato a fasi di contrazione, risoluzione \textit{bottom-up}. L'albero non viene generato tutto contemporaneamente, ma viene creato (e distrutto) nel corso dell'algoritmo, seguendo il cammino di una visita anticipata (\textit{depth-first search}).

Il fatto che nella fase di Conquer venga eliminata la porzione di albero generato per risolvere una sottoistanza, è uno dei difetti principali di questo paradigma. Il non ricordare le soluzioni parziali trovate lo rende un processo computazionale \textit{memoryless}.

TODO pagina 4, albero delle chiamate, disegnini e grafichetti

\subsubsection{Esempio algoritmo D\&C, ricerca del massimo}

Come esempio di algoritmo D\&C, viene presentata una procedura che trova il massimo intero in una sequenza di interi $A[1 \dots n] \in \mathbb{Z}^*$

\begin{algorithm}[H]
\caption{Massimo}\label{alg:max}
\begin{algorithmic}[1]
    \Procedure{MAX}{$A,i,j$}                            
        \If{$i=j$}                             \Comment{BASE}
            \State return $A[i]$
        \EndIf
        \State $k \gets \left\lfloor \frac{i+j}{2} \right\rfloor$    \Comment{DIVIDE - punto di mezzo discreto}
        \State $m_1 \gets $ \Call{MAX}{$A,i,k$} \Comment{RECURSE}
        \State $m_1 \gets $ \Call{MAX}{$A,k+1,j$}
        \If{$m_1 \geq m_2$}              \Comment{CONQUER}
            \State return $m_1$
        \Else
            \State return $m_2$
        \EndIf
    \EndProcedure
\end{algorithmic}
\end{algorithm}

Nota: Nella firma della procedura, sono presenti tutti i parametri necessari a identificare la \textit{generica} sottostruttura

\subsection{Correttezza del DnC}

Induzione e magia

Relazione di ricorrenza

\subsection{Analisi della complessità}

\subsubsection{autocomplete}

Il \textit{merge}, cuore dell'algoritmo \ref{alg:mergesort}, avviene alla riga \ref{alg:mergeline}, quindi guarda a \algref{alg:mergesort}{alg:mergeline} per saperne di più.

\begin{algorithm}[h]
\caption{MergeSort}\label{alg:mergesort}
\begin{algorithmic}[1]
    \Procedure{MergeSort}{$A,p,r$}
    \If{$p<r$}
        \State $ q \gets \left\lfloor \frac{p+r}{2} \right\rfloor $ 
        \State \Call{MergeSort}{$A,p,q$} \Comment{Ordina prima}
        \State \Call{MergeSort}{$A,q+1,r$} \Comment{Ordina dopo}
        \State
            \Call{Merge}{$A,p,q,r$}
            \label{alg:mergeline}
            \Comment{Merge in $\Theta(n)$}
    \EndIf
    \EndProcedure
\end{algorithmic}
\end{algorithm}

àg
èg
òg
ùg


\chapter{Master Theorem}
\section{Contrazioni}
% definizione contrazioni, pag 13.5
\begin{definition}[Contrazione]\label{def:contraz}
    Una contrazione è una funzione $f: \mathbb{N} \rightarrow \mathbb{N}$, per cui vale
    \[ f(n) < n \quad \forall n>n_0 \]
\end{definition}

\subsection{Iterate}
% definizione iterata, pag 13.6
\begin{definition}[Iterata]\label{def:iterata}
    Ad una contrazione sono associate le iterate di $f(n)$
    \[
    \begin{cases} 
        f^{(0)} (n) = n      &  i = 0 \\
        f^{(i)} (n) = f \left( f^{(i-i)} (n) \right)   &  i > 0 \\
        
    \end{cases}
    \]
\end{definition}

Nota: le iterate formano una successione decrescente $ f^{(0)} (n) > f^{(1)} (n) > \cdots > f^{(i)} (n) $ \\
Nella prossima sezione, l'iterata $0$ sarà associata alla radice dell'abero e all'istanza generale, l'iterata $i-esima$ rappresenterà la taglia al livello $i$ dell'albero.

\subsection{Ausiliaria}
% Def ausiliaria, pag 14.6
\begin{definition}[Ausiliaria]\label{def:ausiliaria}
    Alle iterate di una funzione si associa anche una funzione ausiliaria, che indica il maggior indice di iterata per cui il valore è ancora maggiore del valore di base. Nell'albero delle ricorrenze, questo indicherà l'ultimo livello.
    \[ f^*(n, n_0) = \max \{ i>0 : f^{(i)}(n) > n_0 \} \]
    La funzione è definita solo per $n>n_0$, per convenzione assume il valore $f^*(n,n_0)=-1$ se $ n \leq n_0$
\end{definition}

\subsection{Esempi}
\subsubsection{$\bm{f(n) = n/2}$}
% pag 13.9, 14.8
Calcoliamo la forma esplicita dell'iterata e la funzione ausiliaria per
\[ f(n) = \frac{n}{2} \quad \text{con} \quad n=2^k \]
Forma esplicita iterata:
\begin{align*}
    & f^{(0)}(n) = n \\
    & f^{(1)}(n) = f(n) = \frac{n}{2} \\
    & f^{(2)}(n) = f \left( f^{(1)}(n) \right) = f \left( \frac{n}{2} \right) = \frac{n}{4} = \frac{n}{2^2} \\
    & f^{(i)}(n) = \frac{n}{2^i} 
\end{align*}
Dove la generalizzazione  è valida perché gli argomenti restano potenze di due: $n/2^i = 2^{k-i}$ \\
Calcolo funzione ausiliaria:
\begin{align*}
    & f^{(i)}(n) \mgeq n_0 
        & \text{per quali $i\:$?}\\
    & \frac{n}{2^i} \mges n_0 \\
    & 2^i \mles \frac{n}{n_0} \\
    & i < \log_2 \left( \frac{n}{n_0} \right)
        & \text{il vincolo è risolto} \\
    \rightarrow \quad & f^*(n, n_0) = \log_2 \left( \frac{n}{n_0} \right) - 1
        & \text{ne prendo il massimo} \\
    n_0 = 1 \rightarrow \quad & f^*(n, 1) = \log_2 \left( n \right) - 1
        & \text{se $n_0=1$}
\end{align*}

\subsubsection{$\bm{f(n) = n-1}$}
% pag 14, 15
Calcoliamo la forma esplicita dell'iterata e la funzione ausiliaria per
\[ f(n) = n-1 \]
Forma esplicita iterata:
\begin{align*}
    & f^{(0)}(n) = n \\
    & f^{(1)}(n) = f(n) = n-1 \\
    & f^{(2)}(n) = f \left( f^{(1)}(n) \right) = f \left( n-1 \right) = n-1-1 = n-2 \\
    & f^{(i)}(n) = n-i
\end{align*}
Calcolo funzione ausiliaria:
\begin{align*}
    & f^{(i)}(n) \mgeq n_0 
        & \text{per quali $i\:$?}\\
    & n-i \mges n_0 \\
    & i < n - n_0
        & \text{il vincolo è risolto} \\
    \rightarrow \quad & f^*(n, n_0) = n - n_0 - 1
        & \text{ne prendo il massimo} \\
    n_0 = 1 \rightarrow \quad & f^*(n, 1) = n - 2
        & \text{se $n_0=1$}
\end{align*}

\subsubsection{$\bm{f(n) = \sqrt{n}}$}
% pag 14.2, 15.2
Calcoliamo la forma esplicita dell'iterata e la funzione ausiliaria per
\[ f(n) = \sqrt{n} \quad \text{con} \quad n=2^{2^k} \]
Forma esplicita iterata:
\begin{align*}
    & f^{(0)}(n) = n \\
    & f^{(1)}(n) = f(n) = \sqrt{n} = n^{\nicefrac{1}{2}} \\
    & f^{(2)}(n) = f \left( f^{(1)}(n) \right) = f \left( n^{\nicefrac{1}{2}} \right) = n^{\nicefrac{1}{2^2}} \\
    & f^{(i)}(n) = n^{\nicefrac{1}{2^i}} \\
\end{align*}
Calcolo funzione ausiliaria:
\begin{align*}
    & f^{(i)}(n) \mgeq n_0 
        & \text{per quali $i\:$?}\\
    & n^{\nicefrac{1}{2^i}} \mges n_0 \\
    & \log_2 n^{\nicefrac{1}{2^i}} \mges \log_2 n_0 \\
    & \frac{1}{2^i} \log_2 n \mges \log_2 n_0 \\
    & 2^i \mles \frac{\log_2 n}{\log_2 n_0} \\
    & i < \log_2 \frac{\log_2 n}{\log_2 n_0} 
        & \text{il vincolo è risolto} \\
    & i < \log_2 \log_2 n + \log_2 \log_2 n_0 \\
    \rightarrow \quad & f^*(n, n_0) = \log_2 \log_2 n + \log_2 \log_2 n_0 - 1
        & \text{ne prendo il massimo} \\
    n_0 = 2 \rightarrow \quad & f^*(n, 2) = \log_2 \log_2 n - 1
        & \text{se $n_0=2$}
\end{align*}
Il vincolo su $n$ può essere generalizzato a $n=a^{2^k}$, in questo caso $n_0 = a$

\subsubsection{$\bm{f(n) = \left\lfloor n/2 \right\rfloor}$}
Nel caso $f(n) = \left\lfloor n/2 \right\rfloor$, si può dimostrare che
$ f^{(2)}(n) = \left\lfloor \left\lfloor n/2 \right\rfloor / 2 \right\rfloor = \left\lfloor n / 2^2 \right\rfloor $
ma non è per niente banale.

\section{Modifica al \textit{Divide and Conquer}}
% titolo meno informativo della storia
\subsection{Metaalgoritmo}
Metaalgoritmo DnC modificato, pag 12.5

Parametri noti, pag 13

Nuova eq di ricorrenza, pag 13.3

\subsubsection{esempio parametri MAX}
parametri max, pag 13.5

\subsection{Equazione delle ricorrenze generica}
Alberone, pag 15.8

Formulone, pag 16

Da qualche parte le convenzioni sugli operatori, pag 15.5; insieme alle cose a fine capitolo DnC -> Appendice

\subsubsection{Esempio formula}
Esempio, pag 16.5

\subsubsection{Esempio formula}
Esempio, pag 17.5

\section{\textit{Master theorem}}
Ipotesi, pag 18

Tesi, pag 18.5

Dimostrazione, pag 20

Considerazioni sull'asintotico, pag 18.9, 19

\subsection{autocomplete}
Una bella scatola:
\begin{equation}
    \boxed{x^2+y^2 = z^2}
\end{equation}

àg
èg
ìg
òg
ùg
perché


\chapter{Operazioni fra matrici}
\section{Introduzione - Notazione}
% notazione, pag 25.5
Saranno principalmente considerate matrici quadrate $\mathbb{R}^{n \times n}$, con taglia associata $n$, il numero di righe, o $N=n^2$, il numero di elementi.\\
Verranno considerate operazioni costose solo somme, sottrazioni, prodotti e divisioni tra scalari reali. A volte verrà posta particolare attenzione a prodotti e divisioni, che sono più computazionalmente intensi.\\
Una matrice $A$ è composta da elementi $a_{i,j}$ e si indica
\[ A = \left( a_{i,j} \right)_{1 \leq i,j \leq n} \]

\section{Somma e sottrazione}
% sum sub, pag 25.8, 26
La somma (e, in tutta la sezione, in modo analogo la sottrazione) è una delle operazioni definita fra le matrici.
\begin{definition}
    La somma fra matrici è definita come
    \[
    C = A + B \rightarrow
    c_{i,j} = a_{i,j} + b_{i,j}
    \]
    \label{def:somma}
\end{definition}
\begin{algorithm}[H]
\caption{Somma tra matrici}\label{alg:sum}
\begin{algorithmic}[1]
\Procedure{SUM}{$A,B$}
    \State $n \gets A.rows$
        \For{$i \gets 1 $ to $ n $ }
            \For{$j \gets 1 $ to $ n $ }
                \State $c_{i,j} \gets a_{i,j} + b_{i,j} $
            \EndFor
        \EndFor
    \State return $C$
\EndProcedure
\end{algorithmic}
\end{algorithm}
La correttezza dell'algoritmo discende in maniera diretta dalla definizione.

La complessità dell'algoritmo è $T(n) = n^2$

\section{Moltiplicazione di matrici}
\subsection{Definizione}
% definizione, pag 26.4
La moltiplicazione tra matrici è definita secondo il prodotto righe per colonne.
\begin{definition}
    Il prodotto righe per colonne è definito come
    \[
    C = A \times B \rightarrow
    c_{i,j} = \sum_{k=1}^{n}a_{i,k} b_{k,j}
    \]
    \label{def:prodrc}
\end{definition}
\begin{algorithm}[H]
\caption{Prodotto righe per colonne secondo la definizione}\label{alg:mulrcdef}
\begin{algorithmic}[1]
\Procedure{MUL}{$A,B$}
    \State $n \gets A.rows$
        \For{$i \gets 1 $ to $ n $ }
            \For{$j \gets 1 $ to $ n $ }
                \State $c_{i,j} \gets a_{i,j} \cdot b_{i,j} $
                \For{$k \gets 2 $ to $ n $ }
                \State $c_{i,j} \gets c_{i,j} + a_{i,j} \cdot b_{i,j} $
                \EndFor
            \EndFor
        \EndFor
    \State return $C$
\EndProcedure
\end{algorithmic}
\end{algorithm}
La correttezza dell'algoritmo discende in maniera diretta dalla definizione.

La complessità dell'algoritmo è $T(n) = \Theta (n^3)$, infatti trasformando i cicli $for$ in sommatorie, si ottiene la formula
\[
    T(n) = \sum_{i=1}^{n} \sum_{j=1}^{n} \left( 1 + \sum_{k=2}^{n} 2 \right) = 
    = \sum_{i=1}^{n} \sum_{j=1}^{n} \left( 2n-2+1 \right) 
    = \left( 2n-1 \right) \sum_{i=1}^{n} \sum_{j=1}^{n} 1 
    = (2n-1)n^2 = 2n^3 -n^2
\]

\subsection{Implementazione ricorsiva}
% moltiplicazione ricorsiva, pag 27.3, 28
Nel corso del capitolo ci limiteremo a studiare matrici di taglia $n=2^k$, semplificando notevolmente l'analisi asintotica. Questo non comporta una perdita di generalità, infatti $\forall m \neq 2^k$ è sufficiente scegliere $n=2^{\left\lceil \log_2 m \right\rceil} < 2m$ e si può procedere con gli algoritmi che saranno presentati, avendo al più quadruplicato il numero di elementi.

TODO disegni delle matrici pag 27.5

Il prodotto righe per colonne ha la proprietà frattale, ovvero lo si può risolvere dal prodotto righe per colonne dei suoi blocchi.
% TODO l'italiano quà è brutto forte
Infatti dividendo le matrici in blocchi di dimensione $\frac{n}{2} \times \frac{n}{2}$ vale
\[
    1 \leq r \leq 2,\, 1 \leq s \leq 2 \rightarrow C_{r,s} = A_{r,1} B_{1,s} + A_{r,2} B_{2,s}
\]

TODO divisione in blocchi di C, pag 27.6

TODO altri disegni pag 27.9

La proprietà frattale descritta, che permette di ottenere il prodotto righe per colonne di due matrici $n \times n$ tramite prodotti righe per colonne di matrici $\frac{n}{2} \times \frac{n}{2}$ che poi vengono combinati è una proprietà di sottostruttura. Il caso di base è semplicemente il prodotto di due scalari. Quando $n>1$ si calcolano ricorsivamente tutti i prodotti delle sottomatrici.

\begin{algorithm}[H]
\caption{Prodotto righe per colonne ricorsivo}\label{alg:mulrcric}
\begin{algorithmic}[1]
\Procedure{RMUL}{$A, B$}
    \State $n \gets A.rows$
    \If{$n = 1$}                             \Comment{BASE}
        \State return $\left( a_{11} \cdot b_{11} \right)$
    \EndIf
    \State *divido $A, B$ in blocchi $A_{11}, \cdots, B_{22}$*    \Comment{DIVIDE}
    \For{$r \gets 1 $ to $ 2 $ }                                    \Comment{RECURSE e CONQUER sono simultanei}
        \For{$s \gets 1 $ to $ 2 $ }
        \State $C_{r,s} \gets 
        \Call{SUM}{}
        % \Call{SUM}{ % boh si offende un sacco
        \left(
            \Call{RMUL}{A_{r,1}, B_{1,s}}, 
            \Call{RMUL}{A_{r,2}, B_{2,s}} 
        \right)
        $
        % } $
        \EndFor
    \EndFor
    \State return $C$
    \EndProcedure
\end{algorithmic}
\end{algorithm}
Osservando che il \textit{Divide} non comporta alcun costo, il \textit{Recurse} comporta $2$ chiamate ricorsive per ciascuna delle quattro iterazioni dei cicli $for$ e il \textit{Conquer} necessita di $4$ somme su matrici di dimensione $\frac{n}{2} \times \frac{n}{2}$, si può scrivere l'equazione di ricorrenza
\begin{equation}
    T(n) = 
    \begin{cases}
        1 & n=1 \\
        8 T \left( \frac{n}{2} \right) + 4 \left( \frac{n}{2} \right)^2 & n>1
    \end{cases}
    \label{eq:rcric}
\end{equation}
con parametri
\begin{align*}
    a&=8 & b&=2 & w(n)&=n^2 & n^{log_2 8}&=n^3
\end{align*}
per cui siamo nel caso uno del \textit{Master Theorem}, e risulta
\begin{equation*}
    T(n) = \Theta\left( n^{log_2 8} \right) = \Theta\left( n^{3} \right)  
\end{equation*}
TODO applica per ESERCIZIO la formula generale per ricavare le costanti $2n^3-n^2$

\subsection{Algoritmo di \textit{Strassen}}
% moltiplicazione Strassen, pag 29
Il \textit{Master Theorem}, oltre ad essere strumento di analisi, può essere usato per la progettazione di algoritmi. Si può notare dall'equazione \ref{eq:rcric} che il costo delle chiamate ricorsive è molto maggiore rispetto al costo di ricombinazione. Cercando un algoritmo che riduca il primo, eventualmente a scapito del secondo, si può provare ad abbassare la complessità dell'algoritmo.

L'algoritmo di \textit{Strassen} segue proprio questa strada, mostrando una maniera per ottenere il prodotto di due matrici $n \times n$ usando solo $7$ prodotti di matrici $\frac{n}{2} \times \frac{n}{2}$ e un numero costante($18$) di somme e sottrazioni tra matrici $\frac{n}{2} \times \frac{n}{2}$.

\begin{algorithm}[H]
    \caption{Algoritmo di \textit{Strassen}}\label{alg:strassen}
\begin{algorithmic}[1]
% \Procedure{$D_S$}{$A, B$}
\Procedure{D\textsubscript{S}}{$A, B$}
    \State $A_1 \gets A_{11}$; $B_1 \gets \Call{SUB}{B_{12}, B_{22} }$
    \State $A_2 \gets \Call{SUM}{A_{11}, A_{12}}$; $B_2 \gets B_{22}$
    \State $A_3 \gets \Call{SUM}{A_{21}, A_{22}}$; $B_3 \gets B_{11}$
    \State $A_4 \gets A_{22}$; $B_4 \gets \Call{SUB}{B_{21}, B_{11} }$
    \State $A_5 \gets \Call{SUM}{A_{11}, A_{22}}$; $B_5 \gets \Call{SUM}{B_{11}, B_{22} }$
    \State $A_6 \gets \Call{SUM}{A_{12}, A_{22}}$; $B_6 \gets \Call{SUM}{B_{21}, B_{22} }$
    \State $A_7 \gets \Call{SUB}{A_{11}, A_{21}}$; $B_7 \gets \Call{SUB}{B_{11}, B_{12} }$
    \State return $A_1, \cdots, A_7, B_1, \cdots, B_7$
\EndProcedure
\Procedure{R\textsubscript{S}}{$A_1, \cdots, A_7, B_1, \cdots, B_7$}
    \For{$i \gets 1 $ to $ 7 $ }
    \State $P_i \gets \Call{SMUL}{A_{i}, B_{i} }$
    \EndFor
    \State return $P_1, \cdots, P_7$
\EndProcedure
\Procedure{C\textsubscript{S}}{$P_1, \cdots, P_7$}
    \State $C_{11} \gets \Call{SUB}{} \left( \Call{SUM}{P_4, P_5}, \Call{SUB}{P_2, P_6} \right) $
    \State $C_{12} \gets \Call{SUM}{P_1, P_2}$
    \State $C_{21} \gets \Call{SUM}{P_3, P_4}$
    \State $C_{22} \gets \Call{SUB}{} \left( \Call{SUM}{P_1, P_5}, \Call{SUB}{P_3, P_7} \right) $
    \State return $C_1, \cdots, C_4$
\EndProcedure
\Procedure{SMUL}{$A, B$}
    \State $n \gets A.rows$
    \If{$n = 1$}                             \Comment{BASE}
        \State return $\left( a_{11} \cdot b_{11} \right)$
    \EndIf
    \State $\left( A_1, \cdots, A_7, B_1, \cdots, B_7 \right) \gets \Call{D\textsubscript{S}}{A, B}$
    \Comment{DIVIDE}
    \State $\left( P_1, \cdots, P_7 \right) \gets \Call{R\textsubscript{S}}{A_1, \cdots, A_7, B_1, \cdots, B_7}$
    \Comment{RECURSE}
    \State $C  \gets \Call{C\textsubscript{S}}{P_1, \cdots, P_7}$
    \Comment{CONQUER}
    \State return $C$
    \EndProcedure
\end{algorithmic}
\end{algorithm}

La correttezza dell'algoritmo si prova mostrando la correttezza dei singoli blocchi di $C$, per esempio per $C_{12}$
\begin{equation*}
    C_{12} = P_1+P_2 = A_{11} \left( B_{12}-B_{22} \right) + \left( A_{11}+A_{12} \right) B_{22} =
    A_{11}B_{12} - \cancel{A_{11}B_{22}} + \cancel{A_{11}B_{22}} + A_{12}B_{22}
\end{equation*}

L'equazione di ricorrenza associata all'algoritmo è
\begin{equation}
    T(n) = 
    \begin{cases}
        1 & n=1 \\
        7 T \left( \frac{n}{2} \right) + 18 \left( \frac{n}{2} \right)^2 & n>1
    \end{cases}
    \label{eq:ricstrassen}
\end{equation}
per cui la funzione di soglia è diventata $n^{log_2 7} \approx n^{2.80}$, si è ancora nel caso uno e vale
\begin{equation*}
    T_S(n) = \Theta \left( n^{log_2 7} \right)
\end{equation*}
TODO verifica per ESERCIZIO il valore alle costanti, deve venire $T_S(n)=7n^{log_2 7}-6n^2$

Per quanto l'esponente sia diminuito rispetto all'implementazione ricorsiva base, è aumentato il coefficiente da $2$ a $7$. Per valori piccoli di $n$ quindi, la complessità dell'algoritmo di \textit{Strassen} è maggiore, e solo per $n>1024$ diventa conveniente.

\subsection{Algoritmo ibrido}
Si può notare come l'algoritmo di Strassen sia asintoticamente veloce, ma con costanti elevate, mentre l'algoritmo diretto sia asintoticamente più lento, ma abbia costanti minori. Utilizzando la tecnica dell'ibridizzazione, si possono combinare i due algoritmi, mantenendo l'andamento asintotico del primo e diminuendo le costanti.

Il primo approccio, immediato, si limita a selezionare uno dei due algoritmi in base alla taglia della matrice


\begin{algorithm}[H]
    \caption{Algoritmo ibrido naive}\label{alg:mulrcnaive}
\begin{algorithmic}[1]
\Procedure{Naive\_Hybrid\_MUL}{$A, B$}
    \State $n \gets A.rows$
    \If{$n < 1024 $}
        \State return $\Call{MUL}{A,B}$
        \Else
        \State return $\Call{SMUL}{A,B}$
    \EndIf
    \EndProcedure
\end{algorithmic}
\end{algorithm}

L'equazione di ricorrenza associata all'algoritmo è
\begin{equation*}
    T_{NH}(n) = 
    \begin{cases}
        2n^3 - n^2 & n<1024 \\
        7 n^{log_2 7} - 6n^2& n \geq 1024
    \end{cases}
    \label{eq:ricnaivehybrid}
\end{equation*}
e si può notare non è che sia cambiata molto, in particolare per $n \geq 1024 $ è rimasta identica.

Innestiamo quindi in modo più efficace l'algoritmo veloce, modificando il caso base di \textit{Strassen}, cambiando $n_0$ e bloccando prima la ricorsione.

TODO albero pag 31.7

\begin{algorithm}[H]
    \caption{Algoritmo ibrido}\label{alg:mulrcibrido}
\begin{algorithmic}[1]
\Procedure{Hybrid\_SMUL}{$A, B$}
    \State $n \gets A.rows$
    \If{$n \leq n_0 $}
        \State return $\Call{MUL}{A,B}$
        \Comment{BASE}
    \EndIf
    \State $\left( A_1, \cdots, A_7, B_1, \cdots, B_7 \right) \gets \Call{D\textsubscript{S}}{A, B}$
    \Comment{DIVIDE}
    \State $\left( P_1, \cdots, P_7 \right) \gets \Call{R\textsubscript{HS}}{A_1, \cdots, A_7, B_1, \cdots, B_7}$
    \Comment{RECURSE}
    \State $C  \gets \Call{C\textsubscript{S}}{P_1, \cdots, P_7}$
    \Comment{CONQUER}
    \State return $C$
    \EndProcedure
\end{algorithmic}
\end{algorithm}

Nota:
\begin{itemize}[noitemsep,topsep=0pt,parsep=0pt,partopsep=0pt]
    \item[--] nell'algoritmo \ref{alg:mulrcibrido} la chiamata ricorsiva è stata modificata, in modo da chiamare l'algoritmo ibrido
    \item[--] l'analisi di correttezza va modificata, la base è valida per la correttezza di MUL, il resto dell'induzione è analogo
\end{itemize}

L'equazione di ricorrenza associata all'algoritmo è
\begin{equation}
    T(n, n_0) = 
    \begin{cases}
        2n^3 - n^2 & n<n_0 \\
        7 T \left( \frac{n}{2}, n_0 \right) + \frac{9}{2}n^2 & n \geq n_0
    \end{cases}
    \label{eq:richybrid}
\end{equation}

Risolvendo l'equazione ricorsiva, si ottiene una complessità in funzione di $n_0$ e si può minimizzare la funzione

Utilizzando la formula generale \ref{eq:formachiusaric}, con $n=2^k$ e scegliendo $n_0 = 2^k$, si ricava per il caso base
\begin{align*}
    T_0 &= 2 n_0^3 - n_0^2
    \intertext{e per l'ultimo livello prima dei nodi foglia $f^*\left( n, n_0 \right)$}
    l \text{ generale } &\rightarrow \text{ taglia } \frac{n}{2^l} \text{ che pongo uguale a } n_0 \\
    \frac{n}{2^l} &= n_0 \rightarrow 2^l = \frac{n}{n_0} \rightarrow l = log_2 \left( \frac{n}{n_0} \right)
    \intertext{come livello delle foglie, per cui}
    f^*\left( n, n_0 \right) &= log_2 \left( \frac{n}{n_0} \right) -1
    \intertext{che si comporta come previsto, al crescere di $n_0$ il numero di livelli diminuisce; ricordando inoltre}
    S(n) = 7 \quad f(n) &= \frac{n}{2} \quad w(n) = \frac{9}{2} n^2
\end{align*}
risulta
\begin{align*}
    T(n, n_0) &=
    \sum_{l=0}^{ log_2 \left( \frac{n}{n_0} \right) -1} 7^l \frac{9}{2} \left( \frac{n}{2^l} \right)^2
    + 7^{log_2 \frac{n}{n_0}} \left( s n_0^3 - n_0^2 \right) \\
    &= \ldots \text{TODO ESERCIZIO} \\
    &= \left( \frac{2n_0+5}{n_0^{log_2 7}-2} \right)n^{log_2 7} - 6n^2
\end{align*}
Per cui la relazione con $n_0$ risulta solo nella costante moltiplicativa, e possiamo minimizzare $g(n_0)$, vincolato a $n_0 = 2^k$, in generale la derivata che si deve studiare è
\begin{equation}
    \frac{\partial}{\partial n_0} T(n, n_0)
\end{equation}

% \begin{align*}
\begin{gather*}
    \left( \frac{d}{d n_0} g(n_0) \right) n^{log_2 7} \quad \text{di cui studio il segno} \\
    \text{sign} \left( \frac{d}{d n_0} g(n_0) \right) 
    = \text{sign} \left( 2 n_0 - \left( 2n_0+5 \right)\left( log_2 7-2 \right) \right) \\
    % TODO parentesi più grandi
    \frac{d}{d n_0} g(n_0) \geq 0  \rightarrow n_0 \geq \frac{5\left( log_2 7 - 2 \right)}{2\left( 3-log_2 7 \right)} \approx 10,48
    \intertext{per cui il massimo vincolato a $n_0=2^k$ vale}
    n_0' = 8 \rightarrow g(8) \approx 3,92 \\
    n_0'' = 16 \rightarrow g(16) \approx 3,94 \\
% \end{align*}
\end{gather*}

L'equazione di ricorrenza associata all'algoritmo è
\begin{equation}
    T(n, 8) 
    \begin{cases}
        = 2n^3 - n^2 & n \leq 8 \\
        \approx 3,92 n^{log_2 7} + 6n^2 & n > 8
    \end{cases}
    \label{eq:richybrid8}
\end{equation}
Che risulta in quasi il $50\%$ di miglioramento. Notare come $8$ non abbia niente a che fare con il $1024$ che si era ottenuto in precedenza. Il metodo con cui è stato ricavato il valore ci assicura però che questo metodo sia sempre migliore di entrambi.

Nota: il modello di costo considerato non è sufficientemente descrittivo, e trascura tutti i costi associati alla ricorsione (chiamate ricorsive, allocazioni di memoria, salvataggio dello stato del programma\ldots) quindi nella realtà si consiglia di fare test empirici modificando il valore di $n_0$ per individuare il valore effettivamente migliore.

\subsection{autocomplete}
Una bella scatola:
\begin{equation}
    \boxed{x^2+y^2 = z^2}
\end{equation}

Numeri nei casi
\begin{numcases}{T(n)=}
    2^3 \label{escaso1} \\
    2^4 \label{escaso2} 
\end{numcases}

Sotto numeri
\begin{subnumcases}{T(n)=}
    2^3 \label{escaso3} \\
    2^4 
\end{subnumcases}

\begin{itemize}[noitemsep,topsep=0pt,parsep=0pt,partopsep=0pt]
    \item qualcosa
    \item[+] qualcosa
    \item[*] qualcosa
    \item[--] qualcosa
\end{itemize}
àg
èg
ìg
òg
ùg
perché



\chapter{Trasformata veloce di \textit{Fourier}}
\section{Rappresentazione di polinomi}
% polinomi definizione 33.7
In generale, avere diverse rappresentazioni di un oggetto permette di eseguire operazioni diverse su rappresentazioni diverse, dove risulta più comodo. Chiaramente la rappresentazione è legata a come si possono fare le operazioni, e sono inoltre necessarie operazioni efficienti di conversione.

Introduciamo quindi un nuovo dominio applicativo, i polinomi, su cui verrà costruito l'algoritmo per la trasformata veloce di \textit{Fourier}.

\begin{definition}[Polinomio]
    Un polinomio è una funzione $p: \mathbb{C} \rightarrow \mathbb{C}$ definita su un'indeterminata e un insieme di coefficienti, come somma di monomi.
    \begin{equation*}
        p(x) = a_0 + a_1 x + a_2 x^2 + \dots + a_{n-1} x^{n-1} = \sum_{j=0}^{n-1}a_j x^j
    \end{equation*}
    \label{def:polinomio}
\end{definition}

\begin{definition}[Grado di un polinomio]
    Il grado di un polinomio è definito come l'indice massimo del coefficiente non nullo
    \begin{equation*}
        deg(p(x)) = \max \left\{ i: a_i \neq 0 \right\}
    \end{equation*}
    \label{def:poligrado}
\end{definition}

% grado limitato da n, pag 33.9
\begin{definition}[Polinomio di grado limitato da $n$]
    Un polinomio si dice di grado limitato da $n$ se il suo massimo grado può essere $n-1$
    \label{def:polilimitato}
\end{definition}

\subsection{Rappresentazione per coefficienti}
% rapp coeff 33.8
Un polinomio può essere rappresentato a $n$ coefficienti
\begin{equation*}
    p(x) \equiv \vec{a} \in \mathbb{C}^n
\end{equation*}
La rappresentazione può essere facilmente estesa con un'operazione di padding
\begin{equation*}
    p(x) \equiv \left( \vec{a}, 0_m \right) \in \mathbb{C}^{n+m}
\end{equation*}

\subsection{Rappresentazione per punti}
% forse non serve una subsubsection
\begin{theorem}[Teorema di interpolazione]
    Date $n$ coppie di punti $\left( x_i, y_i \right) \in \mathbb{C}^2 \text{ con }
    % x_i~\neq~x_j
    x_i \neq x_j
    % \forall~i~\neq~j
    \; \forall i \neq j
    \; \exists ! p(x)
    $
    di grado limitato da $n$ per cui
    $p(x_i) = y_i$
    detto polinomio interpolante.
    \label{teo:interpolazione}
\end{theorem}
C'è quindi una corrispondenza tra $n-uple$ di punti e un \emph{singolo} polinomio, quindi una $n-upla$ di punti è una rappresentazione di un polinomio di grado limitato da $n$
\begin{equation*}
    p(x) \equiv \left( \vec{x}, \vec{y}\right) \quad
    \vec{x}, \vec{y} \in \mathbb{C}^{n}, x_i \neq x_j \; \forall i \neq j
\end{equation*}
dove $\vec{x}$ si dice base della rappresentazione

Anche la rappresentazione per punti si può estendere
\begin{equation*}
    \left( \vec{x}, \vec{y}\right) \rightarrow
    \left( \vec{x}^E, \vec{y}^E\right) \quad
    \vec{x}^E, \vec{y}^E \in \mathbb{C}^{m}, x_i \neq x_j \; \forall i \neq j, \text{ con } m>n
\end{equation*}
Per estendere la rappresentazione occorre valutare il polinomio in $m-n$ punti aggiuntivi

\subsection{Conversione tra rappresentazioni}
Se si dispone della rappresentazione per coefficienti, è sufficiente valutare il polinomio in un certo numero di punti per ricavare la rappresentazione per punti, mentre se si dispone di una tabulazione occorre interpolare il polinomio. Eseguire questa conversione in maniera efficiente sarà argomento della sezione \ref{sez:conversione}

\section{Operazioni tra polinomi}

\subsection{Operazioni utilizzando la rappresentazione per coefficienti}
% somma sottrazione coefficienti 34.8
\subsubsection{Somma e sottrazione}
Siano $A(x) \equiv \vec{a}$ e $B(x) \equiv \vec{b}$, con $\vec{a}, \vec{b} \in \mathbb{C}^n$ e $C(x) \equiv \vec{c}$ omogeneo, con $\vec{c} \in \mathbb{C}^n$
\begin{align*}
    C(x) &= A(x) + B(x)
    = \sum_{j=0}^{n-1} a_j x^j + \sum_{j=0}^{n-1} b_j x^j 
    = \sum_{j=0}^{n-1} \left( a_j + b_j \right) x^j 
    = \sum_{j=0}^{n-1} c_j x^j 
\end{align*}
Per cui $C(x)$ è rappresentato dalla somma vettoriale delle rappresentazioni: $\vec{c} = \vec{a} + \vec{b}$

% prodotto coefficienti 35
\subsubsection{Prodotto}
Siano $A(x) \equiv \vec{a}$ e $B(x) \equiv \vec{b}$, con $\vec{a}, \vec{b} \in \mathbb{C}^n$, in questo caso $C(x)$ è un polinomio di grado limitato da $2n-1$, infatti $(n-1)+(n-1)+1$ (il limite di grado è di $1$ superiore al grado massimo) quindi $C(x) \equiv \vec{c}$, con $\vec{c} \in \mathbb{C}^{2n-1}$
\begin{align*}
    C(x) &= A(x) \cdot B(x)
    = \left( \sum_{j=0}^{n-1} a_j x^j \right) \cdot \left( \sum_{j=0}^{n-1} b_j x^j \right) 
\end{align*}
Da cui, per $ 0 \leq j \leq 2n-2$
\begin{align*}
    c_j &= \sum_{ \substack{k,h \\ k+h=j \\ 0 \leq k,h \leq n-1} } a_k b_h
    \intertext{che semplifichiamo notando che $h=j-k$}
    c_j &= \sum_{k=0}^{n-1} a_k b_{j-k}
    \intertext{ma questa sommatoria è valida solo sfruttando una convenzione per cui se $j-k$ è $<0$ o $\geq n$, il coefficiente $b_{j-k}$ assume valore $0$}
\end{align*}
Si possono cercare vincoli più stretti su $k$
\begin{equation*}
    \left\{ 
        \begin{array}[h]{l}
            0 \leq k \leq n-1 \\
            0 \leq j-k \leq n-1
        \end{array}
    \right.
    \rightarrow
    \left\{ 
        \begin{array}[h]{l}
            0 \leq k \leq n-1 \\
            j-n+1 \leq k \leq j
        \end{array}
    \right.
\end{equation*}
ossia
\begin{equation*}
    \max\left\{ 0, j-n+1 \right\} \leq k \leq \min \left\{ n-1, j \right\}
\end{equation*}
\begin{definition}[Convoluzione lineare]
    La convoluzione lineare è un operatore vettoriale definito come
    \begin{equation*}
        \vec{c}=\vec{a} * \vec{b} \rightarrow 0\leq j \leq 2n : c_j = \sum_{\max\left\{ 0, j-n+1 \right\}}^{\min \left\{ n-1, j \right\}} a_k b_{j-k}
        % ho dubbi su quel \leq 2n, cioè per j=2n k va da n+1 a n-1 quindi la somma è vuota, ed è corretto, ma comunque si è allungato senza dire niente a nessuno e mi da fastidio
    \end{equation*}
    \label{def:convlin}
\end{definition}
TODO ESERCIZIO: supponi $\vec{b} \in \mathbb{C}^m$, ricava i vincoli, dovrebbe venirti $\max \left\{ 0, j-m+1 \right\}$

% magia delle più nere questo calcolo di complessità
La complessità per calcolare direttamente la convoluzione risulta, contando il numero di operazioni: $n^2$ prodotti $+n^2$ somme $-(2n-1)$ somme che risparmi, per cui $T(n)=2n^2-2n+1$ o alternativamente $T(n)=n^2+(n-1)^2$, in cui si mette in evidenza il numero di prodotti $\left[ n^2 \right]$ e di somme $\left[ (n-1)^2 \right]$.

% algoritmo convoluzione lineare 36
L'algoritmo naive per implementare la convoluzione lineare risulta
\begin{algorithm}[H]
\caption{Convoluzione lineare}\label{alg:convlinnaive}
\begin{algorithmic}[1]
    \Procedure{DEF\_LIN\_CONV}{$\vec{a}, \vec{b}$}
        \State $n \gets \vec{a}.length$
        \For{$j \gets 0 $ to $ 2n-2 $ }
            \State $c_j \gets 0$
            \For{$k \gets \max\left\{ 0, j-n+1 \right\} $ to $ \min \left\{ n-1, j \right\} $ }
                \State $c_j \gets c_j + a_k b_{j-k}$
            \EndFor
        \EndFor
        \State return $ \vec{c}$
    \EndProcedure
\end{algorithmic}
\end{algorithm}
In quest'implementazione una somma è stata sprecata inizializzando $c_j=0$, si potrebbe inizializzare a $c_j=a_{\max\left\{ 0, j-n+1 \right\}}b_{j-\max\left\{ 0, j-n+1 \right\}}$

La complessità dell'algoritmo risulta $T_{DLC} = \Theta \left( n^2 \right)$
\subsection{Operazioni utilizzando la rappresentazione per punti}

% somma per punti 36.3
\subsubsection{Somma e sottrazione}
Siano $A(x) \equiv \left( \vec{x}, \vec{y}_A \right)$ e $B(x) \equiv \left( \vec{x}, \vec{y}_B \right)$ rappresentazioni omogenee, sulla stessa base.
\begin{align*}
    C(x) &= A(x) + B(x)
    \rightarrow C(x_i) = A(x_i)+B(x_i)
    \rightarrow y_{C_{i}}= y_{A_{i}}+ y_{B_{i}}
\end{align*}
Per cui $C(x)$ è rappresentato da $\left( \vec{x}, \vec{y}_A+\vec{y}_B \right)$

% prodotto per punti 36.5
\subsubsection{Prodotto}
La relazione $C(x) = A(x) \cdot B(x) $ è valida per ogni $x$, quindi anche per tutti i punti $x_i$ della base: $C(x_i) = A(x_i) \cdot B(x_i) $.
Tuttavia \emph{non} è sufficiente rappresentare $C$ come
$\left( \vec{x}, \vec{y}_A \odot \vec{y}_B \right)$,
infatti sono necessari $2n-1$ punti per definire il polinomio.

Considerando le rappresentazioni estese $A(x) \equiv \left( \vec{x}^E, \vec{y}_A^E \right)$ e $B(x) \equiv \left( \vec{x}^E, \vec{y}_B^E \right)$, con $\vec{x}^E, \vec{y}_A^E,  \vec{y}_B^E \in \mathbb{C}^{2n-1}$ si ottiene la rappresentazione lecita 
$C(x) \equiv \left( \vec{x}^E, \vec{y}^E_A \odot \vec{y}^E_B \right)$.

Per eseguire il prodotto sono quindi necessari solamente $2n-1$ prodotti, posto di avere a disposizione la rappresentazione estesa, portando a una complessità di $T(n) = \Theta \left( n \right)$

\section{Conversione tra rappresentazioni}\label{sez:conversione}
% convertire in modo lento e poi veloce pag 36.9
% grafo commutativo secsi 36.9
Se si vuole interpolare o valutare un polinomio in maniera generale, non è possibile raggiungere una complessità minore di $\Theta \left( n^2 \right)$. È necessario utilizzare basi particolari per poter accelerare l'algoritmo.

\subsection{Valutazione}
\subsubsection{Valutazione naive}
% valutazione naive 37.2
Ricordando che
    \begin{equation*}
        p(x) = \sum_{j=0}^{n-1}a_j x^j
    \end{equation*}
si ottiene direttamente un algoritmo dalla definizione
\begin{algorithm}[H]
\caption{Valutazione naive}\label{alg:valnaive}
\begin{algorithmic}[1]
    \Procedure{DEF\_VAL}{$\vec{a}, \bar{x}$}
        \State $n \gets \vec{a}.length$
        \State $y \gets a_0$
        \State $pow \gets 1$
        \For{$j \gets 1 $ to $ n-1 $ }
            \State $pow \gets pow \cdot \bar{x}$
            \State $y \gets y + a_j \cdot pow$
        \EndFor
        \State return $y$
    \EndProcedure
\end{algorithmic}
\end{algorithm}
Vengono compiute $n-1$ iterazioni con una somma e due prodotti per ciclo.

% valutazione con horner 37.5
\subsubsection{Valutazione con \textit{Horner}}
Si può riscrivere un polinomio seguendo la regola di \textit{Horner} come 
    \begin{equation*}
        % p(x) = a_0 + x \left( a_1 + x \left( a_2 + \cdots + x 
        p(x) = a_0 + x ( a_1 + x ( a_2 + \cdots + x 
        % \left( a_{n-2} + x a_{n-1} \right) \right. \cdots \right)
        ( a_{n-2} + x a_{n-1} 
        \underbrace{ ) \cdots ) }_{\mathclap{ n-1 \text{ parentesi} } }
    \end{equation*}
si ottiene così l'algoritmo
\begin{algorithm}[H]
    \caption{Valutazione con \textit{Horner}}\label{alg:valhorner}
\begin{algorithmic}[1]
    \Procedure{HOR\_VAL}{$\vec{a}, \bar{x}$}
        \State $n \gets \vec{a}.length$
        \State $y \gets a_{n-1}$
        \For{$j \gets 2 $ to $ n $ }
            \State $y \gets a_{n-j} + \bar{x}y$
        \EndFor
        \State return $y$
    \EndProcedure
\end{algorithmic}
\end{algorithm}
Vengono compiute $n-1$ iterazioni con una somma e un prodotto per ciclo, portando la complessità a $T_H(n) = 2(n-1)$

% aggregato su n punti 37.9
Questo algoritmo deve essere applicato per ciascuno degli $n$ punti, per cui la complessità della valutazione risulta $\Theta \left( n^2 \right)$

\subsection{Interpolazione}
\subsubsection{Interpolazione con \textit{Gauss}}
% interpolazione gauss 38.2
La formula \ref{def:polinomio} vale $\forall x_i \in \vec{x}$ base della rappresentazione
\begin{equation*}
    p(x) = \sum_{j=0}^{n-1}a_j x^j
    \rightarrow y_i = p(x_i) = \sum_{j=0}^{n-1}a_j x_i^j
    \quad 0 \leq i \leq n-1
\end{equation*}
Si può considerare $\vec{a}$ come incognita di un sistema lineare di $n$ equazioni:
\begin{equation*}
    X \vec{a} = \vec{y}
\end{equation*}
dove
\begin{equation*}
    X = \left( 
        \begin{array}[h]{ccccccc}
            1 & x_0 & x_0^2 & \cdots & x_0^j & \cdots & x_0^{n-1} \\
            \vdots &&&&&& \vdots\\
            1 & x_i & x_i^2 & \cdots & x_i^j & \cdots & x_i^{n-1} \\
            \vdots &&&&&& \vdots\\
            1 & x_n & x_n^2 & \cdots & x_n^j & \cdots & x_n^{n-1} 
        \end{array}
    \right)
\end{equation*}
per cui $X$ è una matrice di \textit{Vandermonde}, quindi invertibile se e solo se $x_i \neq x_j, \; i\neq j$ ipotesi verificata per definizione di base.

Utilizzando il metodo dell'eliminazione di \textit{Gauss}, occorrono $n$ operazioni \textit{pivot} su $n^2$ elementi, portando ad una complessità di $\Theta \left( n^3 \right)$

\subsubsection{Interpolazione con \textit{Lagrange}}
% interpolazione lagrange 38.7
\textit{Lagrange} ha trovato una formula chiusa per il polinomio interpolante
\begin{equation*}
    p(x) = \sum_{k=0}^{n-1} \frac{\left(
            \displaystyle
            \prod_{\substack{j=0 \\ j \neq k}}^{n-1} \left( x - x_j \right)
    \right)}{\left( 
            \displaystyle
            \prod_{\substack{j=0 \\ j \neq k}}^{n-1} \left( x_k - x_j \right)
    \right)}
    = \sum_{k=0}^{n-1} \frac{y_k}{Q_k(x_k)} q_k(x)
\end{equation*}
avendo definito
\begin{equation*}
    Q_k(x) = \prod_{\substack{j=0 \\ j \neq k}}^{n-1} \left( x - x_j \right)
\end{equation*}
e si può dimostrare che la formula si valuta in $\Theta \left( n^2 \right)$ (sulle dispense online c'è un utile ESERCIZIO da leggere)

\section{Conversioni meno generali}
Senza perdita di generalità, si possono considerare basi particolari su cui valutare e interpolare un polinomio. In particolare si sceglie una famiglia di vettori di taglia $n$ legata alla radice principale dell'unità:
\begin{equation*}
    \vec{\Omega}_n = 
    \left(
        \omega_n^0=1 , \omega_n, \omega_n^2, \cdots, \omega_n^{n-1} 
    \right)
\end{equation*}
dove $\omega_n$ è la radice principale $n-esima$ di $1$ nel campo complesso.
% conversioni più veloci 39

\subsection{autocomplete}
Una bella scatola:
\begin{equation}
    \boxed{x^2+y^2 = z^2}
\end{equation}

Numeri nei casi
\begin{numcases}{T(n)=}
    2^3 \label{escaso1} \\
    2^4 \label{escaso2} 
\end{numcases}

Sotto numeri
\begin{subnumcases}{T(n)=}
    2^3 \label{escaso3} \\
    2^4 
\end{subnumcases}

\begin{itemize}[noitemsep,topsep=0pt,parsep=0pt,partopsep=0pt]
    \item qualcosa
    \item[+] qualcosa
    \item[*] qualcosa
    \item[--] qualcosa
\end{itemize}
àg
èg
ìg
òg
ùg
perché



\chapter{Programmazione dinamica}
\section{Paradigma del \emph{Dynamic Programming}}

\subsection{Introduzione}

% TODO disegno albero TD BU pag 53.2
Il problema del memorylessness del paradigma \emph{Divide and Conquer} è già stato sottolineato nella sezione \ref{sss:alberochiamate}.
Per esemplificare il problema, si introduce la sequnza di Fibonacci, definita come
\begin{equation*}
    F_n = 
    \begin{cases}
        1 & n=0,1 \\
        F_{n-1} + F_{n-2} & n>1
    \end{cases}
\end{equation*}
da cui si ricava immediatamente un algoritmo ricorsivo
\begin{algorithm}[H]
\caption{Fibonacci ricorsivo}\label{alg:rfib}
\begin{algorithmic}[1]
    \Procedure{R\_FIB}{$n$}
        \If{$ n=0 $ or $n=1$}
            \State return $1$
        \EndIf
        \State return \Call{R\_FIB}{$n-1$} + \Call{R\_FIB}{$n-2$}
    \EndProcedure
\end{algorithmic}
\end{algorithm}
% TODO disegno albero chiamate R_FIB pag 53.4
\noindent
la cui equazione di ricorrenza è
\begin{equation*}
    T_{RF}(n) = 
    \begin{cases}
        0 & n=0,1 \\
        T_{RF}(n-1) + T_{RF}(n-2) + 1 & n>1
    \end{cases}
\end{equation*}
che risulta limitata inferiormente da un esponenziale, infatti per $n>1$
\begin{align*}
    T_{RF}(n) 
    &= T_{RF}(n-1) + T_{RF}(n-2) + 1 \\
    & \geq \, 2 T_{RF}(n-2) + 1 \\
    & \geq \, 2^2 T_{RF}(n-2-2) + 2 + 1 \\
    & \geq \, 2^i T_{RF}(n-2i) + \sum_{j=0}^{i-1} 2^j \\
    \intertext{che raggiunge il caso base quando $n-2i=0$ o $n-2i=1$, ossia per $i= \left\lfloor n/2 \right\rfloor $ sia nel caso di $n$ pari sia nel caso di $n$ dispari}
    & \geq \, 2^{\left\lfloor n/2 \right\rfloor} \cancel{ T_{RF}(0 \text{ o } 1)}
    + \sum_{j=0}^{\left\lfloor n/2 \right\rfloor -1} 2^j \\
    &= 2^{\left\lfloor n/2 \right\rfloor} -1 \\
    &=  \Omega \left( 2^{n/2} \right) = \sqrt{2}^{\,n}
    \intertext{ed essendo $\sqrt{2}>1$ cresce più velocemente di ogni polinomio. Il valore esatto di $T_{RF}$ è}
    T_{RF}(n) &= \Theta \left( \left( \frac{1+\sqrt{5}}{2} \right)^n \right)
\end{align*}

Il sugo della storia è che viene calcolata un numero molto elevato di volte la stessa sottoistanza. Ispirandosi alla fase \emph{Bottom-Up} del \emph{D\&C}, e sfruttando strutture dati che contengano le informazioni necessarie, si può scrivere un algoritmo iterativo che risolve il problema.
\begin{algorithm}[H]
\caption{Fibonacci iterativo}\label{alg:itfib}
\begin{algorithmic}[1]
    \Procedure{IT\_FIB}{$n$}
        \If{$ n=0 $ or $n=1$}
            \State return $1$
        \EndIf
        \State $F[0] \gets 1$
        \State $F[1] \gets 1$
        \For{$i \gets 2 $ to $ n $ } 
            \State $F[i] \gets F[i-1] + F[i-2]$
        \EndFor
        \State return $F[n]$
    \EndProcedure
\end{algorithmic}
\end{algorithm}
\noindent
dove le soluzioni intermedie sono salvate in $F$. In realtà sarebbe sufficiente memorizzare solo gli ultimi due valori della sequenza.

La complessità di questo algoritmo è lineare: $T_{IF}(n) = \Theta \left( n \right)$.

Si può quindi introdurre il paradigma del \emph{Dynamic Programming}, basandosi su queste osservazioni.

\subsection{Paradigma del \emph{Dynamic Programming}}
% \subsection{Introduzione}
I due concetti fondamentali su cui si basa il paradigma \emph{Dynamic Programming} sono
\begin{itemize}[noitemsep,topsep=0pt,parsep=0pt,partopsep=0pt]
    \item[--] dotare di memoria l'algoritmo
    \item[--] implementare la computazione in direzione \emph{bottom-up} (tutti i dati necessari sono già stati calcolati nelle iterazioni precedenti)
\end{itemize}
In ogni caso si lavora con la proprietà di sottostruttura, generando la soluzione ad un'istanza in funzione di sottoistanze di taglia minore.

Il vantaggi sono una maggiore velocità dovuta al non replicare la computazione, gli svantaggi sono legati al dover implementare la computazione in maniera da essere sicuri di avere a disposizione le soluzioni necessarie al momento giusto, che per casi articolati non è triviale. Si può sfruttare la convenienza della fase \emph{top-down}, che genera e risolve le istanze in ordine coerente, scrivendo l'algoritmo in maniera ricorsiva e dotandolo di memoria, come descritto nella sezione seguente.

\section{Memoizzazione di un algoritmo ricorsivo}

È possibile modificare un algoritmo ricorsivo \emph{D\&C} memorizzando le soluzioni intermedie attraverso un processo detto di memoizzazione.

\subsection{Metodo generale}

Un algoritmo memoizzato è costituito da due subroutine:
\begin{enumerate}
    \item \textbf{Routine di inizializzazione} INIT\_\{NomeAlg\}
        \begin{itemize}
            \item risolve i casi di base direttamente
            \item inizializza una struttura tabellare \emph{globale} con
                \begin{itemize}
                    \item valori delle istanze di base, nelle locazioni associate alle istanze di base
                    \item valori di default in posizioni associate a istanze non di base (il valore di default deve essere scelto in modo da far capire che non è stato ancora calcolato)
                \end{itemize}
            \item invoca la seconda procedura (ricorsiva) \\ % FORMATTA
        \end{itemize}
    \item \textbf{Routine ricorsiva} REC\_\{NomeAlg\}($i$)
        \begin{itemize}
            \item controlla sulla tabella per vedere se l'istanza $i$ è già stata risolta
                \begin{itemize}
                    \item se sì la ritorna
                    \item se no
                        \begin{itemize}
                            \item la calcola con la proprietà di sottostruttura
                            \item la memorizza nella tabella
                            \item la ritorna
                        \end{itemize}
                \end{itemize}
        \end{itemize}
\end{enumerate}

\textbf{Nota:} lo spazio delle sottoistanze deve essere 
\begin{itemize}[noitemsep,topsep=0pt,parsep=0pt,partopsep=0pt]
    \item[--] piccolo
    \item[--] facilmente indicizzabile
\end{itemize}

\subsection{Algoritmo di Fibonacci memoizzato}

Per l'algoritmo di Fibonacci le due funzioni risultano
\begin{algorithm}[H]
\caption{Fibonacci memoizzato}\label{alg:fibmemoizzato}
\begin{algorithmic}[1]
    \Procedure{INIT\_FIB}{$n$}
        \If{$ n=0 $ or $n=1$}
            \State return $1$
        \EndIf
        \State $F[0] \gets 1$, $F[1] \gets 1$
        % \State $F[1] \gets 1$
        \For{$i \gets 2 $ to $ n $ } 
            \State $F[i] \gets 0$
        \EndFor
        \State return \Call{REC\_FIB}{$n$}
    \EndProcedure
    \Procedure{REC\_FIB}{$n$}
        \If{$F[i] = 0$}
            \State $F[i] \gets \Call{REC\_FIB}{i-1} + \Call{REC\_FIB}{i-2} $
        \EndIf
        \State return $F[i]$
    \EndProcedure
\end{algorithmic}
\end{algorithm}
\textbf{Note:}
$n$ è un parametro attuale, il valore che si vuole calcolare, mentre $i$ è un parametro formale, che descrive la generica sottoistanza su cui lavora l'algoritmo ricorsivo.
Nella funzione ricorsiva si testa se la struttura tabellare contiene già la soluzione della sottoistanza, e se no lo si calcola con l'algoritmo ricorsivo e lo si salva. Nelle successive chiamate il test fallisce e non è necessario calcolare nuovamente il valore.
Il \emph{caveat} nel dare memoria a un algoritmo \emph{D\&C} è che in lo spazio delle sottoistanze deve essere molto piccolo, in questo caso una sola sottoistanza per ogni valore di $n$. Inoltre perché l'algoritmo sia efficiente lo spazio delle possibili sottoistanze da risolvere deve essere facilmente indirizzabile e salvabile in una struttura dati semplice.

\textbf{Analisi della complessità:}
La ricorrenza non riesce a catturare il fatto che l'albero delle chiamate venga tagliato ogniqualvolta il valore di una sottoistanza sia stato calcolato precedentemente.
% TODO albero delle chiamate memoizzato pag 55.9
Per la fase di inizializzazione, il numero di operazioni aritmetiche che vengono eseguite è nullo.
Dall'albero delle chiamate della procedura ricorsiva con memoria, si nota che l'albero è diventato molto più snello, e sono comparse foglie dove prima erano presenti sottoalberi. Inoltre il numero di nodi interni, gli unici rispetto a cui viene fatto lavoro, sono pari al numero di sottoistanze \emph{distinte} che vanno risolte per ottenere l'istanza $n$. 
Solo nel nodo interno si esegue il conquer (la somma) associato ad ogni chiamata e la complessità sarà corrispondente al numero di nodi interni per il lavoro compiuto in ciascun nodo. C'è un nodo interno per ogni chiamata \emph{non} di base, tutte le altre sono chiamate che hanno trovato il valore in $F$ e non hanno compiuto lavoro, se non un \emph{lookup} nella tabella.
\begin{equation*}
    T_{RF} (n) = n-2+1 = n-1
\end{equation*}

\section{Paradigma generale \emph{Dynamic Programming}}

\subsection{Problemi di ottimizzazione combinatoria}

Ricordiamo la definizione di problema computazionale $\bpi{} \subseteq \bi{} \times \bs{}$

Si può definire per ogni istanza un sottoinsieme $\bs{}(i)$ formato da tutte le possibili soluzioni di quell'istanza, ricordando che una sottoistanza può ammetere più soluzioni.
\begin{equation*}
    \forall i \in \bs{} \quad S(i) = \left\{ s \in \bs{} : i \, \bpi{} \, s \right\}
\end{equation*}

Definiamo inoltre una funzione di costo che va da S in un qualche insieme totalmente ordinato.
\begin{equation*}
    c : \bs{} \to \mathbb{R}
\end{equation*}

Data $i$ un istanza generica, si vuole determinare non solo una soluzione $s \in \bs{}(i)$, ma individuare $s^*$ che massimizza (o minimizza) il criterio del costo.
\begin{equation*}
    c(s^*) = \max \left\{ c(s) : s \in \bs{}(i) \right\}
\end{equation*}

\subsection{Caratteristiche di problemi di ottimizzazione risolubili con la programmazione dinamica}

Un problema si presta ad essere risolto con la programamzione dinamica se presenta la seguente proprietà di sottostruttura ottima, che per un prolbema di ottimizzazione è una proprietà molto più ristretta rispetto a quella necessaria per il \emph{D\&C}. Infatti mette in relazione soluzioni ottime di un istanza a soluzioni ottime di sottoistanze. Questa proprietà si dice anche \emph{optimal substructure property}.

% Dare paradigmi generali che si applicano quando un problema va approcciato con la programamzione dinamica.

% Come nel dnc si sviluppa una proprietà di sottostruttura, possiamo seguire una serie di indicazioni generali che si possono istanziare volta per volta che conducono in maniera ordinata alla risuluzone di un problema con la prog dinanica. 

% Quando si decide di usare la prog dinamica?

\begin{definition}[Proprietà di sottostruttura ottima]
    Un problema gode della proprietà di sottostruttura ottima se
    \begin{enumerate}
        \item la soluzione ottima di un'istanza non di base si ottiene combinando soluzioni ottime di sottoistanze
        \item la proprietà di sottostruttura ottima genera sottoproblemi ripetuti
        \item lo spazio delle sottoistanze generate da una data istanza non è troppo grande
    \end{enumerate}
\end{definition}

Se la proprietà 2 manca, si può applicare il \emph{D\&C} classico e la programamzione dinamica non aiuterebbe.
La proprietà 3 assicura l'efficienza del paradigma, il concetto è arbitrario per ora, se non esiste una struttura dati che riesca a memorizzare in maniera efficiente i risultati intermedi, il metodo avrà risultati peggiori. In genere la dimensionalità della struttura è pari al numero di sottoistanze generate (e.g. se istanze di taglia $|i|=n$ generano $n^2$ sottoistanze, possono essere salvate in un array \emph{bi}dimensionale, $n^3$ \emph{tri}dimensionale e così via).

\subsection{Paradigma generale}

Presentiamo un paradigma generale per la programmazione dinamica, enumerando i passi da seguire per sviluppare un programma secondo il \emph{dynamic programming}, garantendo implicitamente la correttezza del codice.

\begin{enumerate}
    \item caratterizza la struttura di una soluzione ottima $s^*$ a un'istanza $i$ non di base in funzione di soluzioni ottime $s_1^*, s_2^*, \cdots, s_k^*$ di sottoistanze di $i$
    \item
        \begin{enumerate}
            \item determina una relazione di ricorrenza sui costi di istanza e sottoistanza, mettendo in relazione il costo di un'istanza come una funzione dei costi delle sottoistanze \\ $c(s^*)=f\left(c(s_1^*),c(s_2^*),\cdots,c(s_k^*)\right)$
                \label{enum:pd1}
            \item determina la minima informazione strutturale necessaria ad ottenere $s^*$ a partire da $s_1^*, s_2^*, \cdots, s_k^*$, cercando di memorizzare incrementi, e non le intere soluzioni di sottoistanze
                \label{enum:pd2}
        \end{enumerate}
    \item
        \begin{enumerate}
            \item calcola il costo $c(s^*)$ utilizzando la ricorrenza impostando la computazione
                \begin{itemize}
                    \item in maniera \emph{bottom-up} iterativa
                    \item in maniera memoizzata
                \end{itemize}
                \label{enum:pd3}
            \item calcola l'informazione addizionale strutturale per ottenere $s^*$ e memorizzarla
                \label{enum:pd4}
        \end{enumerate}
\end{enumerate}

I punti \ref{enum:pd1}, \ref{enum:pd3} sono sufficienti per ricavare il costo della soluzione ottima,
mentre i punti \ref{enum:pd2}, \ref{enum:pd4} sono necessari quando si è interessati anche al valore della soluzione ottima.

\section{Ricerca della \emph{Longest Common Subsequence}}

\subsection{Sottostringhe e sottosequenze}

\subsubsection{Notazione}

\begin{itemize}
    \item[--] $\Sigma$ un alfabeto finito di simboli
    \item[--] $\Sigma^{*}$ la sua stella di \emph{Kleene}, l'insieme infinito di concatenazioni finite di simboli in $\Sigma$
    \item[--] $X = < x_1, x_2, \cdots, x_m > \in \Sigma^{*}$ una stringa di lunghezza $m = |X|$
    \item[--] $X = \varepsilon$ la stringa vuota
\end{itemize}

\subsubsection{Prefisso, suffisso, sottostringa}

\begin{definition}[Prefisso]
    Data una stringa $X$ con $|X|=m$ 
    un prefisso proprio
    è formato dai primi $i$ simboli di $X$.
    \[
        {X_i = < x_1, \cdots, x_i >}
        \quad \text{con} \quad
        1 \leq i \leq m
    \]
    Il prefisso improprio $X_0$ è definito come la stringa vuota.
    \label{def:prefisso}
\end{definition}

\begin{definition}[Suffisso]
    Data una stringa $X$ con $|X|=m$ 
    un suffisso proprio
    è formato dagli ultimi $j$ simboli di $X$.
    \[
        {X^j = < x_j, \cdots, x_m >}
        \quad \text{con} \quad
        1 \leq j \leq m
    \]
    Il prefisso improprio $X^{m+j}$ è definito come la stringa vuota.
    \label{def:suffisso}
\end{definition}

\begin{definition}[Sottostringa]
    Data una stringa $X$ con $|X|=m$ 
    una sottostringa
    è formata da $j-i+1$ simboli contigui di $X$.
    \[
        X_{i..j} = < x_i, \cdots, x_j >
        \quad \text{con} \quad
        1 \leq i \leq j \leq m
    \]
    Se $i>j$, $X_{i..j}=\varepsilon$
    \label{def:sottostringa}
\end{definition}

Lo spazio delle sottoistanze di una stringa $|X|=m$ risulta
\begin{align*}
    1 + \sum_{i=1}^{m} \sum_{j=i+1}^{m} 1 
    &= 
    1 + \sum_{i=1}^{m} \left( m-i+1 \right)
    &
    i'=m-i+1
    \\
    &=
    1 + \sum_{i'=1}^{m} i' 
    \\
    &=
    1 + \frac{m\left( m+1 \right)}{2}
    \\
    &=
    \Theta \left( m^2 \right)
\end{align*}
dove avendo già considerata una volta la sottostringa impropria, l'indice $i$ può variare a piacimento, mentre l'indice $j$ ne è sempre maggiore.

\subsubsection{Sottosequenza}

\begin{definition}[Sottosequenza]
    Data una stringa $X$ con $|X|=m$,
    $Z = < z_1, \cdots, z_k >$ è una sottosequenza di $X$ se esiste una successione di indici crescente
    % $
    \[
    1 \leq i_1 < i_2 < \cdots < i_k \leq m
    \]
    % $
    tale che 
    $z_j = x_{i_j}$, $1 \leq j \leq k$
    % \\
    , ed è la successione di indici crescente che \emph{realizza} $Z$ in $X$
    \label{def:sottosequenza}
\end{definition}

Per esempio, sia $X = <A,B,C,B,B,D>$, una sua sottostringa è $Z_1 = X_{1..3} = <A,B,C>$ mentre una sua sottosequenza è $Z_2 = <A,C,B>$. La stessa sottosequenza può essere realizzata da più successioni di indici, infatti: $i_1=1, \, i_2=3, \, i_3=\left\{ 4,5 \right\}$

Una sottostringa è sempre anche una sottosequenza, infatti $X_{k..s}$ è realizzata da $i_1=k, \, i_2=k+1, \, \cdots, \, i_{s-k+1}=s$

Lo spazio delle sottoistanze è considerevolmente più ampio, infatti il numero di possibili sottoinsiemi ordinati di indici $S \subseteq \left\{ 1, \cdots, m \right\}$, che quindi generano una sottosequenza valida, è pari a $2^m$.

\subsection{Definizione del problema}

Si può quindi introdurre il problema della ricerca della sottosequenza comune \emph{più lunga} tra due stringhe. Formalmente:

\begin{definition}
    Data un'istanza $(X,Y) \in \Sigma^* \times \Sigma^*$, con
    $ X = < x_1, \cdots, x_m > $ e 
    $ Y = < y_1, \cdots, y_n > $,
    allora
    $ Z = < z_1, \cdots, z_k > $
    è sottosequenza comune di $X$ e $Y$ se $Z$ è sottosequenza di entrambe le stringhe.
    L'obiettivo è determinare la sottosequenza comune di massima lunghezza 
    $Z^* = LCS\left( X,Y \right)$
    \label{def:lcs}
\end{definition}
Per provare che una sottosequenza sia comune è sufficiente esibire le due successioni di indici che realizzano $Z$ in $X$ e $Y$, che sia la sottosequenza di lunghezza massima viene assicurato dall'algoritmo che risolve il problema.

Spesso invece di voler individuare $Z^*$ è sufficiente conoscerne il costo $|Z^*|$, di cui si sa che $|LCS(X,Y)| < \min\left\{ m,n \right\}$.

Si può stabilire se $Z$ è sottosequenza di $X$ in tempo $\Theta \left( |X| \right)$, da cui si ricava un algoritmo naive per la ricerca della $LCS$, generando tutte le possibili sottosequenze della stringa più corta e verificando siano sottosequenze della più lunga. Se $m \leq n$, la complessità risulta $\Theta \left( 2^{m} n \right)$.

\subsection{Proprietà di sottostruttura ottima}

Supponiamo di star analizzando la sottoistanza di taglia $m+n$ relativa alle stringhe
% $ X = < x_1, \cdots, x_m > $, $ Y = < y_1, \cdots, y_n > $, e sia
\[
    X = < x_1, \cdots, x_m > \quad  Y = < y_1, \cdots, y_n >
\]
 e sia
$ Z^* = < z_1, \cdots, z_k > $
la soluzione ottima associata a questa sottoistanza. 
Soffermiamoci sull'ultimo elemento delle due stringhe:
% $ X = < X_{m-1}, a > $ e $ Y = < Y_{n-1}, b > $.
\[
     X = < X_{m-1}, a >  \quad  Y = < Y_{n-1}, b >
\]
Si possono presentare due casi.
\begin{enumerate}
    \item $a=b$: di sicuro $z_k=a=b$ e $ Z^* = < Z_{k-1}^*, a > $: intuitivamente se $Z^*$ non terminasse con $a$, esisterebbe $\widehat{Z}$ di lunghezza massima che termina con $\widehat{z}_k \neq a$ ed è sottosequenza di $X_{m-1}^*$ e $Y_{n-1}^*$ (non terminando con $a$), ma appendendo $a$ a $\widehat{Z}$ si otterrebbe una sottosequenza più lunga, che è assurdo.
        \\
        Inoltre $Z_{k-1}^*$ sta in $X_{m-1}^*$ e $Y_{n-1}^*$, avendo già considerato l'ultimo elemento di entrambe.
        \\
        La sottoistanza generata ha taglia $m+n-2$
    \item $a \neq b$: di sicuro $z_k$ non può essere uguale ad $a$ e $b$ contemporaneamente (e può comunque essere diverso da entrambi i valori). Deve quindi verificarsi uno dei due sottocasi:
        \begin{enumerate}[label=(\roman*)]
            \item $z_k \neq a$: va risolto $LCS\left( X_{m-1}, Y \right)$
            \item $z_k \neq b$: va risolto $LCS\left( X, Y_{n-1} \right)$
        \end{enumerate}
        In entrambi i casi la sottoistanza generata ha taglia $m+n-1$
\end{enumerate}

Lo spazio dei sottoproblemi è $\left\{ \left( X_i, Y_j \right), \: 0 \leq i \leq m, \: 0 \leq j \leq n \right\}$ ed ha quindi taglia pari a $(m+1)(n+1) = \Theta (mn)$

Formalizzando la proprietà di sottostruttura ottima risulta:

\begin{lemma}[Proprietà di sottostruttura ottima per la \emph{Longest Common Subsequence}]
    Per un generico sottoproblema $\left( X_i, Y_j \right), \: 0 \leq i \leq m, \: 0 \leq j \leq n$ sia 
    $ Z^* = < z_1, \cdots, z_k > = LCS\left( X_i,Y_j \right)$ allora:
    \begin{enumerate}
        \item $(i=0) \vee (j=0) \Rightarrow z^* = \varepsilon$
            \label{psolcs:casobase}
        \item $(i>0) \wedge (j>0) \wedge (x_i = y_j) \Rightarrow $
            \begin{enumerate}
                \item $z_k = x_i = y_j$
                    \label{psolcs:caso2a}
                \item $Z_{k-1}^* = LCS(X_{i-1}, Y_{j-1})$
                    \label{psolcs:caso2b}
            \end{enumerate}
        \item $(i>0) \wedge (j>0) \wedge (x_i \neq y_j) \Rightarrow $
            \\
            $ Z^*$ è la stringa più lunga tra $LCS\left( X_{i-1}, Y_j \right)$ e $LCS\left( X_i, Y_{j-1} \right)$
            \label{psolcs:caso3}
    \end{enumerate}
    \label{lemma:psolcs}
\end{lemma}

\begin{proof}[Caso \ref{psolcs:casobase}]
    La dimostrazione del caso \ref{psolcs:casobase} è immediata, infatti l'unica sottosequenza comune è la stringa vuota $\varepsilon$.
    \[
        LCS(\varepsilon, Y_j), LCS(X_i, \varepsilon) \Rightarrow Z^*=\varepsilon 
    \]
\end{proof}

\begin{proof}[Caso \ref{psolcs:caso2a}]
    La dimostrazione del caso \ref{psolcs:caso2a} si svolge per assurdo:
    Sia 
    $ Z^* = < z_1, \cdots, z_k > $
    la soluzione ottima realizzata da 
    $1 \leq i_1 < i_2 < \cdots < i_k \leq i $ in $X_i$
    e da
    $1 \leq j_1 < j_2 < \cdots < j_k \leq j $ in $Y_j$
    .\\
    Si supponga per assurdo che
    \[
        z_k \neq x_i ( \neq y_j)
    \]
    Per $1 \leq s \leq k$ il carattere generico $z_s$ è
    \[
        z_s = x_{i_s} = y_{j_s}
    \]
    Applicando l'ipotesi assurda rispetto a  $z_s$ per $s=k$ vale
    \[
        z_k = x_{i_k} \neq x_i \quad z_k = y_{j_k} \neq y_j
    \]
    devono quindi essere diversi gli indici
    \[
        (i_k \neq i) \; \wedge \; (j_k \neq j)
    \]
    e nelle successioni di indici allora
    \[
        (i_k<i \text{ o } i_k \leq i-1) 
        \; \wedge \;
        (j_k<j \text{ o } j_k \leq j-1)
    \]
    per cui $Z^*$ è sottosequenza comune di $X_{i-1}$ e $Y_{j=i}$.
    Si consideri $\widehat{Z} = < Z^* , x_i >$, $\widehat{Z}$ è sottosequenza sia di $X_i$ sia di $Y_j$
    e le successioni di indici da esibire sono quelle di $Z^*$ con appesi $i_k < i_{k+1} = i$ e $j_k < j_{k+1} = j$.
    Quindi si trova una sottosequenza comune più lunga di $Z^*$, ma $Z^*$ era supposta come soluzione ottima, quindi è stato raggiunto un assurdo.
\end{proof}

\begin{proof}[Caso \ref{psolcs:caso2b}]
    Anche la dimostrazione del caso \ref{psolcs:caso2b} si svolge per assurdo:
    Assumendo che la soluzione ottima $Z^*$ sia realizzata dalle stesse successioni di indici della dimostrazione precedente, gli indici che realizzano
    $Z_{k-1}^* = LCS(X_{m-1}^*, Y_{n-1}^*)$
    sono i primi $k-1$ indici che realizzano $Z^*$.
    \[
        \begin{array}{c|cc}
            1 \leq i_1 < i_2 < \cdots < i_{k-1} & < \cancel{i_k} = i  & \text{ in } X_{i-1}
            \\
            1 \leq j_1 < j_2 < \cdots < j_{k-1} & < \cancel{j_k} = j  & \text{ in } Y_{j-1}
        \end{array}
    \]
    Quindi $i_{k-1} \leq i-1$ e $j_{k-1} \leq j-1$
    per cui 
    $Z_{k-1}^* $ è sottosequenza comune di $X_{i-1}^* $ e $ Y_{j-1}^*$ ed è dunque una soluzione ammissibile; si dimostra anche che è la soluzione ottima.
    \\
    Si supponga per assurdo che
    $Z_{k-1}^* \neq LCS(X_{i-1}^*, Y_{j-1}^*)$
    , allora deve esistere $\bar{Z}$ comune a $X_{i-1}^* $ e $ Y_{j-1}^*$ di lunghezza
    % \[
    $
    | \bar{Z} | > | Z_{k-1}^* | = k-1
    % \]
    $.
    Considerando $\widehat{Z} = < \bar{Z} , x_i > $, $\widehat{Z}$ è sottosequenza comune di $X_i$ e $Y_j$ di lunghezza
    \[
        | \widehat{Z} | = | \bar{Z} | + 1 > k -1 +1 = k = | Z^* |
    \]
    Il che conduce ad un assurdo, essendo $Z^*$ la soluzione ottima.
\end{proof}

\begin{proof}[Caso \ref{psolcs:caso3}]
    Anche la dimostrazione del caso \ref{psolcs:caso3} si svolge per assurdo:
    Sia $ Z^*$ la stringa più lunga tra $LCS\left( X_{i-1}, Y_j \right)$ e $LCS\left( X_i, Y_{j-1} \right)$,
    realizzata da 
    $1 \leq i_1 < i_2 < \cdots < i_k \leq i $ in $X_i$
    e da
    $1 \leq j_1 < j_2 < \cdots < j_k \leq j $ in $Y_j$.
    \\
    Si deve verificare uno dei due casi
    $( z_k \neq x_i ) \vee (z_k \neq y_j)$:
    \\
    Nel primo caso
    $ z_k \neq x_i \Rightarrow i_k<i$
    e $i_k \leq i-1$
    quindi $Z^*$ è in questo caso sottosequenza del prefisso più corto $X_{i-1}$ e come in precedenza è sottosequenza di $Y_j$. È quindi una sottosequenza comune a $X_{i-1}$ e $Y_j$ e deve essere $Z^* = LCS(X_{i-1}, Y_{j})$. Se per assurdo non lo fosse, esisterebbe $\widehat{Z}^* = LCS(X_{i-1}, Y_{j})$ con $|\widehat{Z}| > |Z^*|$, ma $\widehat{Z}$ è anche comune a $X_{i}$ e $Y_j$ che è un assurdo perché $Z^*$ è ottima.
    \\
    Nel secondo caso $Z^* = LCS(X_{i}, Y_{j-1})$, dimostrato analogamente.
    \\
    Dato che uno dei due casi deve succedere, $Z^*$ sarà la più lunga delle due $LCS$ trovate.
\end{proof}
% note su prop strutturali TODO pag 64
La funzione di costo è di due parametri $0 \leq i \leq m$ e $0 \leq j \leq n$
\begin{equation*}
    l(i,j) = 
    \begin{cases}
        0 & (i=0) \vee (j=0)
        \\
        1+l(i-1,j-1) & (i,j>0) \wedge (x_i = y_j)
        \\
        \max \left\{ l(i-1,j) , l(i,j-1) \right\}
        & \text{altrimenti}
    \end{cases}
\end{equation*}
% TODO esempio albero pag 64.5
Il caso peggiore è caratterizzato da $m=n$ quando tutti i caratteri sono diveri tra loro e $LCS=\varepsilon$ e l'albero si biforca sempre.
\begin{equation*}
    T(n,n) = 
    \begin{cases}
        0 & n=0 \\
        T(n-1, n) + T(n, n-1) + 1 & n>0
    \end{cases}
\end{equation*}
Essendo una funzione monotona
\begin{align*}
    T(n,n) &\geq 2T(n-1,n-1)+1
    \intertext{e si può procedere per unfolding:}
    & \geq 2 \cdot 2 T(n-1-1,n-1-1)+2+1
    \\
    \dots & \geq 2^j T(n-j,n-j)+\sum_{k=0}^{j-1}2^k
    \intertext{il caso base si raggiugne per $n-j=0$ ossia $j=n$}
    & \geq \sum_{j=0}^{n-1}2^k = 2^{n-1}
\end{align*}
Seguendo il paradigma \emph{D\&C} anche solo calcolare il costo della souzione ottima è esponenziale.

\subsection{Implementazione dell'algoritmo \emph{LCS} iterativo}
Per applicare il paradigma del \emph{Dynamic Programming}, occorre individuare la minima informazione addizionale per ricostruire la soluzione ottima.
In un vettore bidimensionale si può salvare quale caso è successo, memorizzando uno di tre simboli:
\begin{equation*}
    b(i,j) = 
    \begin{cases}
        ` \nwarrow \textrm{'}
        &
        LCS(X_i, Y_j)
        = < LCS(X_{i-1}, Y_{j-1}), x_i >
        \\
        ` \uparrow \textrm{'}
        &
        LCS(X_i, Y_j)
        = LCS(X_{i}, Y_{j-1})
        \\
        ` \leftarrow \textrm{'} 
        &
        LCS(X_i, Y_j)
        = LCS(X_{i-1}, Y_{j})
    \end{cases}
\end{equation*}

Si può seguire un algoritmo iterativo per calcolare il costo, avendo cura di aver già calcolato i sottoproblemi necessari al momento giusto. Per l'istanza $(i,j)$ sono necessari $(i-1,j-1)$, $(i,j-1)$ e $(i-1,j)$. Il diagramma delle dipendenze permette di visualizzare le relazioni tra sottoistanze e di pianificare l'esecuzione in maniera appropriata. La struttura del diagramma rispecchia quella della struttura dati in cui verranno memorizzate le informazioni, $L\left[ 0 \div m, 0\div n \right]$.
\begin{equation*}
    \left[ 
    \begin{array}[H]{ccccc}
        0 & \cdots & & \cdots & 0 \\
        \vdots & (i-1,j-1) & & (i-1,j) & \\
        & & \nwarrow & \uparrow & \\
        \vdots & (i,j-1) & \leftarrow & (i,j) & \\
        0 &&&&\\
    \end{array}
    \right]
\end{equation*}
Tenuti in considerazione gli elementi che vengono popolati dal caso base, dal diagramma si evince che una scansione per righe visita gli elementi nell'ordine desiderato, trovando sempre i tre elementi su cui dipende l'istanza già correttamente calcolati. La scansione non è univoca, per esempio procedendo per colonne non si commettono errori.

L'algoritmo iterativo per trovare la sottosequenza comune tra due stringhe risulta
\begin{algorithm}[H]
\caption{\emph{Longest Common Subsequence}}\label{alg:lcsit}
\begin{algorithmic}[1]
    \Procedure{LCS}{$X,Y$}
        \State $m \gets |X|$
        \State $n \gets |Y|$
        \For{$i \gets 0 $ to $ m $ }
            % \Comment{Caso base prima colonna}
            \label{alg:lcsit:cbc}
            \State $L[i,0] \gets 0$
        \EndFor
        \For{$j \gets 1 $ to $ n $ }
            % \Comment{Caso base prima riga}
            \label{alg:lcsit:cbr}
            \State $L[0,j] \gets 0$
        \EndFor
        \For{$i \gets 1 $ to $ m $ }
            \For{$j \gets 1 $ to $ n $ }
                % \Comment{L'indice interno è di colonna}
                \label{alg:lcsit:colmaj}
                \If{$x_i = y_j$}
                \label{alg:lcsit:c1}
                    \State $L[i,j] \gets 1 + L[i-1,j-1]$
                    \State $B[i,j] \gets ` \nwarrow \textrm{'}$
                \Else
                    \If{$L[i-1,j] \geq L[i,j-1]$}
                    \label{alg:lcsit:c2}
                        \State $L[i,j] \gets L[i-1,j]$
                        \State $B[i,j] \gets ` \uparrow \textrm{'}$
                    \Else
                        \State $L[i,j] \gets L[i,j-1]$
                        \State $B[i,j] \gets ` \leftarrow \textrm{'}$
                    \EndIf
                \EndIf
            \EndFor
        \EndFor
        \State return $L[m,n], B$
    \EndProcedure
\end{algorithmic}
\end{algorithm}

Dove alle righe \ref{alg:lcsit:cbc} e \ref{alg:lcsit:cbr} si sta inizializzando il caso di base, e alla riga \ref{alg:lcsit:colmaj} la tabella sta venendo popolata secondo una scansione \emph{row-major}, in quanto l'indice interno è quello di colonna.

Nel calcolo della complessità dell'algoritmo, il confronto alla riga \ref{alg:lcsit:c1} è tra elementi della stringa, quindi viene considerato costoso, mentre quello alla riga \ref{alg:lcsit:c2} è tra interi, e si può trascurare. Si conta un confronto per iterazione del ciclo interno, per un totale di 
\begin{equation*}
    T_{LCS} = mn
\end{equation*}
In realtà i sottoproblemi utili al calcolo della soluzione sono ancora meno, ed utilizzando la versione memoizzata dell'algoritmo \emph{D\&C}, solo i sottoproblemi necessari vengono calcolati, al prezzo di utilizzare un algoritmo ricorsivo, con il suo \emph{overhead} per le chiamate (e.g. \textit{call stack}).

Una volta che si ha a disposizione $B$, stampare la sottosequenza comune è semplice:
\begin{algorithm}[H]
\caption{Stampa della \emph{Longest Common Subsequence}}\label{alg:lcsprint}
\begin{algorithmic}[1]
    \Procedure{PRINT\_LCS}{$B,i,j,X$}
        \If{$\left( i=0 \right) \vee \left( j=0 \right)$}
            \State return
        \EndIf
        \If {$B[i,j] = ` \nwarrow \textrm{'}$ }
            \State \Call{PRINT\_LCS}{$B,i-1,j-1,X$}
            \State print$\left( x_i \right)$ 
        \Else
            \If {$B[i,j] = ` \leftarrow \textrm{'}$ }
                \State \Call{PRINT\_LCS}{$B,i,j-1,X$}
            \Else
                \State \Call{PRINT\_LCS}{$B,i-1,j,X$}
            \EndIf
        \EndIf
    \EndProcedure
\end{algorithmic}
\end{algorithm}

\subsection{Implementazione dell'algoritmo \emph{LCS} memoizzto}
\begin{algorithm}[H]
\caption{\emph{Longest Common Subsequence}}\label{alg:lcsmem}
\begin{algorithmic}[1]
    \Procedure{INIT\_LCS}{$X,Y$}
        \State $m \gets |X|$
        \State $n \gets |Y|$
        \If{$\left( m=0 \right) \vee \left( n=0 \right)$}
            \State return $0$
        \EndIf
        \For{$i \gets 0 $ to $ m $ }
            \State $L[i,0] \gets 0$
        \EndFor
        \For{$j \gets 1 $ to $ n $ }
            \State $L[0,j] \gets 0$
        \EndFor
        \For{$i \gets 1 $ to $ m $ }
            \For{$j \gets 1 $ to $ n $ }
                \State $L[i,j] \gets -1$
            \EndFor
        \EndFor
        \State return \Call{REC\_LCS}{$X,Y,m,n$}
    \EndProcedure
    \Procedure{REC\_LCS}{$X,Y, i,j$}
        \If{$L[i,j]=-1$}
            \If{$x_i = y_j$}
                \State $L[i,j] \gets 1 + \Call{REC\_LCS}{X,Y,i-1,j-1}$
            \Else
                \If{$\Call{REC\_LCS}{X,Y,i,j-1} \geq \Call{REC\_LCS}{X,Y,i-1,j}$}
                \label{alg:lcsmem:c1}
                    \State $L[i,j] \gets L[i-1,j]$
                \Else
                    \State $L[i,j] \gets L[i,j-1]$
                \EndIf
            \EndIf
        \EndIf
        \State return $L[i,j]$
    \EndProcedure
\end{algorithmic}
\end{algorithm}

Notiamo che dopo la riga \ref{alg:lcsmem:c1} gli elementi $L[i-1,j]$ e $L[i,j-1]$ sono inizializzati, e possono essere assegnati senza problemi.

Nel caso peggiore l'algoritmo fa un confronto per ogni coppia di prefissi, quando trova il default. La complessità migliora quando le stringhe sono simili. Se una delle due è sottostringa dell'altra, la complessità è lineare.


\section{\emph{Matrix Chain Multiplication}}

\subsection{Definizione del problema}

La moltiplicazione tra matrici è definita anche fra matrici rettangolari, a patto che il numero di colonne della matrice sinistra sia pari al numero di righe della matrice destra.
\begin{equation*}
\begingroup
\setlength{\arraycolsep}{2pt} % horizontal
\renewcommand{\arraystretch}{0.5} % vertical spacing
    \begin{array}[h]{ccccccccccc}
          & A &   & \times &   & B &   & = &   & C & \\%[-4pt]
        m &   & n &        & n &   & p &   & m &   & p
    \end{array}
\endgroup
\end{equation*}

L'algoritmo per moltiplicare matrici quadrate viene modificato lievemente:
\begin{algorithm}[H]
\caption{Prodotto di matrici rettangolari}\label{alg:mulret}
\begin{algorithmic}[1]
\Procedure{RECT\_MUL}{$A,B$}
    \State $m \gets A.rows$
    \State $n \gets A.cols$
    \State $p \gets B.rows$
    \For{$i \gets 1 $ to $ m $ }
        \For{$j \gets 1 $ to $ p $ }
            \State $c_{i,j} \gets 0$
            \For{$k \gets 1 $ to $ n $ }
                \State $c_{i,j} \gets c_{i,j} + a_{i,k} \cdot b_{k,j} $
            \EndFor
        \EndFor
    \EndFor
    \State return $C$
\EndProcedure
\end{algorithmic}
\end{algorithm}
La complessità risulta una funzione di tre parametri
\begin{equation*}
    T\left( m,n,p \right) = mnp
\end{equation*}

La moltiplicazione si generalizza a catene di matrici, sempre mantendo la compatibilità tra le dimensioni
\begin{equation*}
\begingroup
\setlength{\arraycolsep}{2pt} % horizontal
\renewcommand{\arraystretch}{0.5} % vertical spacing
    \begin{array}[h]{ccccccccccccc}
            & A_1 &     & A_2 &     & \cdots &         & A_i &     & \cdots &         & A_n &   \\%[-4pt]
        p_0 &     & p_1 &     & p_2 &        & p_{i-1} &     & p_i &        & p_{n-1} &     & p_n
    \end{array}
\endgroup
\end{equation*}
La catena di moltiplicazioni può quindi essere rappresentata da un array di $n+1$ dimensioni (massime) distinte
\begin{equation*}
    \vec{p} \in \left( \mathbb{N}^+ \right)^{n+1}
\end{equation*}
La moltiplicazione \emph{non} è commutativa, ma è associativa. Questo comporta una certa libertà nell'ordine con cui si effettuano le moltiplicazioni, che influisce molto sul numero totale di prodotti tra elementi da eseguire. Per esempio
\begin{equation*}
\begingroup
\setlength{\arraycolsep}{2pt} % horizontal
\renewcommand{\arraystretch}{0.5} % vertical spacing
    \begin{array}[h]{ccccccc}
           & A_1 &     & A_2 &   & A_3 &   \\  
        10 &     & 100 &     & 5 &     & 50
    \end{array}
\endgroup
\end{equation*}
può essere svolta nei seguenti modi 
\begin{equation*}
\left( \left( A_1 A_2 \right) A_3 \right)
\quad
\text{o}
\quad
\left( A_1 \left( A_2 A_3 \right) \right)
\end{equation*}
che risultano nei seguenti passi intermedi
\begin{equation*}
\begingroup
\setlength{\arraycolsep}{2pt} % horizontal
\renewcommand{\arraystretch}{0.5} % vertical spacing
    \begin{array}[h]{ccccccc}
           & A_{1,2} &   & A_3 &   \\  
        10 &         & 5 &     & 50
    \end{array}
\quad
\text{o}
\quad
    \begin{array}[h]{ccccccc}
           & A_{1} &     & A_{2,3} &   \\  
        10 &       & 100 &         & 50
    \end{array}
\endgroup
\end{equation*}
Nel primo caso però il numero di operazioni richieste è $10 * 100 * 5 + 10 * 5 * 50 = 7500$
mentre nel secondo è $100 * 5 * 50 + 10 * 100 * 50 = 75000$. Una buona parentesizzazione permette quindi di risparmiare un gran numero di operazioni.

\subsection{Proprietà di sottostruttura ottima}

Ad ogni parentesizzazione è associato un albero binario pieno, con $n$ foglie e $n-1$ nodi interni. Il numero di parentesizzazioni è quindi pari al numero di possibili alberi, ossia $\Theta \left( 4^n / n^{3/2} \right)$.
Risolvere il problema per enumerazione è chiaramente impraticabile.

TODO albero parentesizzazione esempio, pag 70

Il problema della \emph{Matrix Chain Multiplication} può essere quindi riscritto come:
\\
\textbf{Ingresso:} $ \vec{p} \in \left( \mathbb{N}^+ \right)^{n+1} $
\\
\textbf{Uscita:} $T^*$ parentesizzazione ottima

TODO albero parentesizzazione binaria, pag 70.5

Si nota che la parentesizzazione $T^*$ è composta da un nodo radice (che rappresenta una moltiplicazione) e due sottoalberi, anch'essi alberi binari pieni, che sono parentesizzazioni di catene più corte di matrici contigue. Si può quindi intuire la presenza di una proprietà di sottostruttura, di cui si dimostra la correttezza e l'ottimalità.

Si supponga di stare analizzando il sottoproblema generico $A_{i..j}$, con associata la parentesizzazione $T_{i..j}^{*}$, ossia la moltiplicazione ottima di $A_i, A_{i+1},\ldots, A_j$ per $1 \leq i \leq j \leq n$.
\\
Nel caso base $i=j$ la catena ha una sola matrice, $A_{i..j} = A_i$ e l'albero è composto dal solo nodo radice.
\\
Per il caso non di base $i<j$, si supponga di conoscere la parentesizzazione ottima
$T_{i..j}^*$
e ne si analizzi la struttura. L'albero è composto dal nodo radice e da due sottoalberi, 
$T_{i.. \bar{k}}$
e
$T_{\bar{k}+1..j}$
, con $i \leq \bar{k} < j$. Va dimostrata l'ottimalità delle sottoistanze, ossia
$T_{i.. \bar{k}} = T_{i.. \bar{k}}^*$ e 
$T_{\bar{k}+1..j} = T_{\bar{k}+1..j}^*$.
Mettendo in relazione il costo di $T_{i..j}^*$ con i costi di $T_{i.. \bar{k}}$ e $T_{\bar{k}+1..j}$, si ottiene:
\begin{equation*}
    c \left( 
T_{i..j}^*
    \right)
    =
    c \left( 
T_{i.. \bar{k}}
    \right)
    +
    c \left( 
T_{\bar{k}+1..j}
    \right)
    + p_{i-1} \: p_{\bar{k}} \: p_j
\end{equation*}
L'ottimalità dei sottoproblemi si dimostra per \emph{cut and paste}: se per assurdo $ T_{i.. \bar{k}} $ non fosse la parentesizzazione ottima di $A_{i..\bar{k}}$, si potrebbe sostituire il sottoalbero con quello ottimo. Ma in questo modo il costo di $ T_{i..j}^* $ diminuirebbe, che è un assurdo perché è la soluzione ottima di $A_{i..j}$.
\\
Il valore di $\bar{k}$ che minimizza il costo si ricava come
\begin{equation*}
    \bar{k}
    =
    \argmin_{1 \leq \bar{k} < j}
    \left\{ 
    c \left( 
    T_{i.. \bar{k}}^*
    \right)
    +
    c \left( 
    T_{\bar{k}+1..j}^*
    \right)
    + p_{i-1} \: p_{\bar{k}} \: p_j
    \right\}
\end{equation*}
Scrivendo il costo della soluzione come $m(i,j) = c\left( T_{i..j}^* \right)$ si ottiene:
\begin{equation*}
    m(i,j)
    =
    \begin{cases}
    0 & i=j \\    
    \displaystyle\min_{1 \leq \bar{k} < j}
    \left\{ 
    m(i, \bar{k})
    +
    m(\bar{k}+1,j)
    + p_{i-1} \: p_{\bar{k}} \: p_j
    \right\}
    & i \neq j
    \end{cases}
\end{equation*}
mentre per l'informazione addizionale per ricostruire la soluzione è sufficiente memorizzare il punto dove si spezza l'albero:
\begin{equation*}
    S(i,j)
    =
    \bar{k} =
    \displaystyle\argmin_{1 \leq \bar{k} < j}
    \left\{ 
    m(i, \bar{k})
    +
    m(\bar{k}+1,j)
    + p_{i-1} \: p_{\bar{k}} \: p_j
    \right\}
\end{equation*}
La presenza di problemi ripetuti si verifica facilmente: l'istanza $(1,4)$ genera, nel momento dell'accumulazione del massimo:
per $k=1$ i problemi $(1,1)$ e $(2,4)$,
per $k=2$ i problemi $(1,2)$ e $(3,4)$,
per $k=3$ i problemi $(1,3)$ e $(4,4)$.
Il sottoproblema $(2,4)$ a sua volta genera le coppie di sottoproblemi 
$(2,2)$ e $(3,4)$,
$(2,3)$ e $(4,4)$,
per cui si nota già la ripetizione dell'istanza $(3,4)$.
L'approccio con la programmazione dinamica al problema può quindi migliorare l'efficienza rispetto al \emph{D\&C}.

\subsection{Implementazione dell'algoritmo \emph{MCM} iterativo}

L'ultimo problema da risolvere per poter implementare l'algoritmo in maniera \emph{bottom-up} è stabilire l'ordine di risoluzione delle sottoistanze.
Osservando la matrice dei costi $m[1..n, 1..n]$ \ref{eq:mcmmatcosti}, si nota che i casi base $i=j$ sono lungo la diagonale principale, e i casi non di base $i<j$ sono nella parte superiore della matrice.
Supponendo di star calcolando $m(i,j)$, i sottoproblemi che è necessario avere a disposizione sono: per la parte $m(i,k)$, quando $k=i$, il problema è sulla diagonale principale, e per valori crescenti di $k$, i problemi sono sulla riga $i$. Per l'addendo $m(k+1,j)$, quando $k=i$ è necessario il problema subito sotto, ossia $m(i+1,j)$, e per i successivi valori di $k$ devono essere a disposizione i valori sulla colonna $j$, fino ad arrivare alla diagonale principale. Le frecce indicano valori crescenti di $k$.
\begin{equation}
    \left[ 
        \begin{array}[H]{ccccccc}
            0 & & & & & & \\
            & 0 & & & & & \\
            & & i,i & i+1,i & \rightarrow & i,j & \\
            & & & 0 & & i+1,j & \\
            & & & & 0 & \downarrow & \\
            & & & & & j,j & \\
            & & & & & & 0\\
        \end{array}
    \right]
    \label{eq:mcmmatcosti}
\end{equation}
Procedendo per diagonali parallele alla principale, e riducendone la dimensione fino ad arrivare all'angolo, si possono calcolare gli elementi in un ordine coerente.
L'equazione di una generica diagonale è $l'=j'-i'+1$, che per $l'=1$ risulta la diagonale principale: $1=j'-i'+1 \rightarrow i'=j'$.
Il primo indice di iterazione sarà quindi $l$ crescente da $1$ a $n$. Fissato un generico $l$, gli indici della diagonale generica risultano
$(1,l)$, $(2,l+1)$ e così via, in generale: $(\bar{i}, \bar{i}+l-1)$. L'ultimo elemento viene raggiunto per $\bar{i}+l-1=n \rightarrow \bar{i}=n-l+1$.
Il valore di $\bar{j}$ discende da $l$ ed $\bar{i}$.

L'algoritmo per calcolare la parentesizzazione ottima di una catena di matrici risulta
\begin{algorithm}[H]
\caption{\emph{Matrix Chain Multiplication}}\label{alg:mcmit}
\begin{algorithmic}[1]
\Procedure{MCM}{$\vec{p}$}
    \State $n \gets \vec{p}.len$
    \For{$i \gets 1 $ to $ n $ }
        \State $m[i,i] \gets 0$
    \EndFor
    \For{$l \gets 2 $ to $ n $ }
        \For{$i \gets 2 $ to $ n-l+1 $ }
            \State $m[i,j] \gets +\infty$
            \label{alg:mcmit:initmij}
            \State $j \gets i+l-1$
            \For{$k \gets i $ to $ j-1 $ }
                \State $q \gets m[i,k]+m[k+1,j]+ p_{i-1} \, p_k \, p_j$
                \If{$q < m[i,j]$}
                    \State $m[i,j] \gets q$
                    \State $S[i,j] \gets k$
                \EndIf
            \EndFor
        \EndFor
    \EndFor
    \State return $m[1,n]$, $\vec{S}$
\EndProcedure
\end{algorithmic}
\end{algorithm}

Anche scorrendo la matrice secondo il \emph{reversed column major} gli elemento vengono visitati in ordine corretto:
\begin{algorithmic}[1]
    \For{$i \gets 1 $ to $ n $ }
        \For{$j \gets i $ down to $ 1 $ }
        \EndFor
    \EndFor
\end{algorithmic}

La complessità dell'algoritmo risulta:
\begin{align*}
    T_{MCM}(n) 
    &= \sum_{l=2}^{n} \sum_{i=1}^{n-l+1} \sum_{k=i}^{i+l-2} 2
    & i+l-2-i+1 = l-1 &\\
    &= \sum_{l=2}^{n} \sum_{i=1}^{n-l+1} 2 (l-1)
    & n-l+1-1+1 = n-l+1 &\\
    &= 2 \sum_{l=2}^{n} (l-1) (n-l+1)
    & l'=l-1 &\\
    &= 2 \sum_{l'=1}^{n-1} l' (n-l')
    \\
    &= 2 \left[ n \sum_{l'=1}^{n-1} l' - \sum_{l'=1}^{n-1} (l')^2\right]
    \labeq{A}
    \\
    &= 2 \left[ n \frac{n(n-1)}{2} - \frac{(n-1)n(2n-1)}{6}\right]
    \\
    &= n(n-1) \left[ n - \frac{2n-1}{3}\right]
    \\
    &= \frac{n(n-1)(n+1)}{3}
    \\
    &= \Theta \left( n^3 \right)
\end{align*}
Dove in $\labeq{A}$ è stata usata, istanziandola per $n-1$,
\begin{equation*}
    \sum_{i=1}^{n} i^2 = \frac{n(n+1)(2n+1)}{6}
\end{equation*}

Avendo a disposizione $S$, scrivere la parentesizzazione è semplice, osservando che
\begin{equation*}
    str\left( T_{i..j}^* \right) = < `(\textrm{'} , 
    str\left( T_{i..k}^* \right) ,
    str\left( T_{k+1..j}^* \right)
    , `)\textrm{'} > 
\end{equation*}
con $str\left( T_{i..i}^* \right) = `A_i\textrm{'} $ 
\begin{algorithm}[H]
\caption{\emph{Stampa della parentesizzazione}}\label{alg:mcmprint}
\begin{algorithmic}[1]
\Procedure{PRINT\_P}{$i,j,\vec{S}$}
    \If{$i = j$}
        \State \Call{print}{$`A_i\textrm{'}$}
        \State return
    \EndIf
    \State \Call{print}{`(\textrm{'}}
    \State \Call{PRINT\_P}{$i, S[i,j], \vec{S}$}
    \State \Call{PRINT\_P}{$S[i,j]+1, j, \vec{S}$}
    \State \Call{print}{`)\textrm{'}}
\EndProcedure
\end{algorithmic}
\end{algorithm}
Il costo della stampa si misura nel numero di caratteri che vengono stampati: essendo la parentesizzazione un albero pieno con $n$ foglie e $n-1$ nodi interni, contando due parentesi per nodo, verranno stampati $n+2(n-1) = \Theta(n)$ caratteri.

L'algoritmo \ref{alg:mcmit} viene utilizzato nel calcolo effettivo di una catena di moltiplicazioni tra matrici compatibili $\vec{A}=\left( A_1, \dots, A_n \right)$, descritta da $\vec{p}$, nel seguente modo, strutturato in maniera simile al PRINT\_P, utilizzando anche RECT\_MUL (\ref{alg:mulret}):
\begin{algorithm}[H]
\caption{\emph{Moltiplicazione ottima di matrici rettangolari}}\label{alg:mcmmul}
\begin{algorithmic}[1]
    \State $m, \vec{S} \gets \Call{MCM}{\vec{p}}$
    \Procedure{MCM\_MUL}{$\vec{A},\vec{S},i,j$}
    \If{$i = j$}
        \State return $A_i$
    \EndIf
    \State \Call{MCM\_MUL}{$\vec{A},\vec{S},i, S[i,j]$}
    \State \Call{MCM\_MUL}{$\vec{A},\vec{S},S[i,j]+1,j$}
    \State return \Call{RECT\_MUL}{$X,Y$}
\EndProcedure
\end{algorithmic}
\end{algorithm}
Il costo sarà proprio $m$, calcolato in MCM.


\chapter{Paradigma \emph{Greedy}}
\section{Paradigma \emph{Greedy}}

\subsection{Introduzione}

Il problema del memorylessness del paradigma \emph{Divide and Conquer} è stato risolto utilizzando il paradigma del \emph{Dynamic Programming}. Risolvendo un problema con la programmazione dinamica, però, la soluzione viene costruita componendo le soluzioni di sottoistanze, scegliendo di volta in volta la sottosoluzione migliore. Questa scelta può avvenire solo dopo aver calcolato \emph{tutte} le soluzioni alle sottoistanze.

Il paradigma \emph{Greedy} seleziona ad ogni iterazione la scelta più promettente, e calcola la soluzione alla sottoistanza relativa \emph{solo} a quella scelta.

Occorre dimostrare che la scelta non comprometta l'ottimalità della soluzione.

\subsection{Definizione}

Il paradigma \emph{Greedy} agisce in tre passi:
\begin{enumerate}
    \item Scelta \emph{Greedy}: compie una scelta che sembra essere quella più promettente, localmente ottima, che non comprometta la soluzione: la soluzione ottima conterrà quella scelta.
    \item \emph{Clean up}: l'istanza viene ripulita, in accordo con la scelta effettuata.
    \item \emph{Tail recursion}: viene risolta l'\emph{unica} istanza generata, come ultimo comando della funzione. Questo tipo di ricorsione può sempre essere scritto in maniera iterativa.
\end{enumerate}

Vanno quindi dimostrate due proprietà:
\begin{enumerate}
    \item la scelta \emph{Greedy} (SG) non compromette l'ottimalità della soluzione locale: \\
        $\exists S^*$ che contiene la scelta \emph{Greedy}
    \item $\exists S^*$ che, oltre alla scelta \emph{Greedy}, contiene la soluzione della sottoistanza ottenuta dal \emph{clean up}, detta sottoistanza residua
\end{enumerate}

\section{Selezione di attività}

\subsection{Definizione del problema}

Un problema che si presta ad essere risolto con il paradigma \emph{Greedy} è la selezione di attività. Data una risorsa condivisa e un insieme di attività che la utilizzano, come si può selezionare il massimo numero di attività compatibili? Formalmente si definiscono:
\begin{itemize}
    \item[--] Risorsa condivisa
    \item[--] Insieme di attività $S = \left\{ a_1, \cdots, a_n \right\}$ con $a_i = [s_i, f_i)$
    \item[--] Attività compatibili: $a_i$ è compatibile con $a_j$ se
        $[s_i, f_i) \cap [s_j, f_j) = \emptyset$
        o
        $(f_i \leq s_j ) \vee (f_j \leq s_i)$
\end{itemize}
L'obiettivo è determinare il sottoinsieme $S^* \subseteq S$ di attività mutualmente compatibili (intervalli a coppie disgiunti) di cardinalità massima.

\subsection{Soluzione \emph{Greedy}}

Senza perdita di generalità, si può assumere che le attività siano ordinate per tempo di fine non decrescente:
$f_1 \leq f_2 \leq \cdots \leq f_n$

L'intuizione è di selezionare di volta in volta l'istanza che finisce prima, in modo da lasciare il più tempo possibile \emph{compatto} per selezionare le altre. Nella fase di \emph{clean-up}, si scorrono le attività in ordine, eliminando quelle non compatibili. È sufficiente fermarsi alla prima istanza compatibile che si trova: successive istanze che non sono compatibili con quella selezionata, non saranno compatibili neppure con quella trovata per prima, e saranno eliminate successivamente.

\begin{algorithm}[H]
\caption{Selezione di attività, implementazione ricorsiva}\label{alg:asrec}
\begin{algorithmic}[1]
    \Procedure{REC\_AS}{$\vec{s}, \vec{f}, g$}
        \State $n \gets \vec{s}.len$
        \If{$g > n$}
        \Comment{Problema vuoto}
            \State return
        \EndIf
        \State $SG \gets \left\{ a_g \right\}$
        \Comment{Prima attività, scelta \emph{Greedy}}
        \State $i \gets g+1$
        \While{ $\left( s_i < f_g \right) \wedge \left( i \leq n \right)$}
            \State $i \gets i+1$
            \Comment{\emph{Clean-up}}
        \EndWhile
        \State return $SG \cup \Call{REC\_AS}{\vec{s}, \vec{f}, i}$
        \Comment{\emph{Tail recursion}}
    \EndProcedure
\end{algorithmic}
\end{algorithm}

\begin{algorithm}[H]
\caption{Selezione di attività, implementazione iterativa}\label{alg:asiter}
\begin{algorithmic}[1]
    \Procedure{GREEDY\_AS}{$\vec{s}, \vec{f}$}
        \State $n \gets \vec{s}.len$
        \State $g \gets 1$
        \State $A \gets \left\{ a_g \right\}$
        \Comment{Insieme di attività scelte $A$}
        \For{$i \gets 2 $ to $ n $ }
        \Comment{Alterna selezione e \emph{clean-up}}
            \If{$ s_i \geq f_g $}
                \State $g \gets i$
                \State return $A \cup \left\{ a_g \right\}$
            \EndIf
        \EndFor
        \State return $A$
    \EndProcedure
\end{algorithmic}
\end{algorithm}

\subsection{Dimostrazione della correttezza}

\subsubsection{Proprietà di scelta \emph{Greedy}}

Occorre dimostrare che esiste una soluzione ottima $A^*$ che contiene la scelta \emph{Greedy} ${a_1}$. Si procede per \emph{cut and paste}: Sia $\widehat{A}$ una soluzione ottima. Se ${a_1} \in \widehat{A}$ si conclude, se no si sostituisce un'attività in $\widehat{A}$ con la scelta \emph{Greedy} e si dimostra che l'insieme ottenuto è ancora ottimo e contiene la scelta \emph{Greedy}.
Sia $\widehat{i} =  \argmin\limits_i \{ a_i \in \widehat{A} \} $, e sia
\[
A^* = \widehat{A} 
\;
\underbracket[1pt]{
\setminus \left\{ a_{\widehat{i}} \right\} 
}_{\text{\emph{cut}}}
\;
\underbracket[1pt]{
\vphantom{ \setminus \left\{ a_{\widehat{i}} \right\} }
\cup \left\{ a_1 \right\}
}_{\text{\emph{paste}}}
\]

Va dimostrato che $a_1$ è compatibile con ogni $a_j$, ossia 
$\forall a_j \in \widehat{A} \setminus \left\{ a_{\widehat{i}} \right\} \Rightarrow f_1 \leq s_j$
\\
Vale $f_1 \leq f_{\widehat{i}}$ perché $a_1$ è la scelta \emph{Greedy}, e $f_{\widehat{i}} \leq s_j$ perché $\widehat{A}$ contiene attività compatibili, per cui $f_1 \leq s_j$, e $A^*$ è ammissibile. Inoltre $|A^*| = |\widehat{A}|$, quindi è anche soluzione ottima. Per costruzione $A^*$ contiene la scelta \emph{Greedy}, e si conclude.

\subsubsection{Proprietà di sottostruttura ottima}

Va dimostrata l'esistenza di una soluzione ottima $A^*$ che, oltre ad $a_1$, contiene una soluzione ottima al sottoproblema residuo $S_r$. 
Il sottoproblema residuo si ottiene effettuando il \emph{clean-up}, eliminando tutte le attività $a_j$, con $j>1$, per cui $s_j < t_1$, fermandosi o per $g : f_1 \leq s_g$, o quando tutte le attività sono state eliminate. $S_r$ è quindi un suffisso di attività che parte da $g$.

Si possono verificare due casi:

$S_r = \emptyset $: tutte le attività sono in conflitto con $a_1 \Rightarrow A^*=\left\{ a_1 \right\}$
e la soluzione al $S_r$ è $sol( A^* \setminus \left\{ a_1 \right\} ) = sol ( \emptyset ) = \emptyset$.
La soluzione al $S_r$ è vuota, e $A^*$ oltre alla scelta \emph{Greedy} non contiene nulla.

$S_r = \left\{ a_g, a_{g+1}, \cdots, a_n \right\}, g>1$: almeno un'attività è compatibile con $a_1$, la soluzione ottima dovrà contenere almeno due attività. Si deve dimostrare che $A^* \setminus \left\{ a_1 \right\}$ è soluzione ammissibile e ottima per $S_r$.

Per dimostrare l'ammissibilità vanno mostrate due proprietà: le attività in $A^* \setminus \left\{ a_1 \right\}$ devono essere mutualmente compatibili, e lo sono perché questo è un sottoinsieme di attività compatibili; $A^* \setminus \left\{ a_1 \right\} \subseteq S_r$, devono essere attività contenute in $S_r$, e lo sono, perché $S_r$ è stato ottenuto proprio eliminando $a_1$ e tutte le attività non compatibili con questa, ossia $a_2, \cdots, a_{g-1}$, e nessuna delle attività eliminate per ottenere $S_r$ potrebbe convivere con $a_1$.

Per dimostrare l'ottimalità di $A^* \setminus \left\{ a_1 \right\}$ per $S_r$ si procede per assurdo, supponendo che $A^* \setminus \left\{ a_1 \right\}$ \emph{non} sia ottima per $S_r$. Deve quindi esistere un sottoinsieme $\widehat{A} \subseteq S_r$ di attività compatibili
contenente $a_g$, la scelta \emph{Greedy} per $S_r$,
con $|\widehat{A}| > |A^*|-1$, e se questo fosse vero si troverebbe una soluzione a $S$ migliore di $A^*$, che era ottima.
Si può aggiungere a questo insieme $a_1$ che di sicuro non è già presente, non essendo in $S_r$, ottenendo $|\widehat{A} \cup \left\{ a_1 \right\} | > |A^*|$. 
Tutte le attività in $\widehat{A}$ sono compatibili con $a_1$, infatti $f_1 < s_g$ per costruzione del problema residuo, $s_g < f_g$ per ogni attività e $f_g < s_i$ per ogni attività in $\widehat{A}$, per la compatibilità di $a_g$.
Questo conduce ad un assurdo, perché $A^*$ era la soluzione ottima.

\section{Compressione di un file}

\subsection{Definizione del problema}

Un file può essere visto come stringa di caratteri di un alfabeto, $F \in \Sigma^*$. A ciascun carattere è associata una frequenza: $\forall c \in \Sigma : f(c) \in [0,1]$. Il file usa solo un sottoalfabeto $C \subseteq \Sigma : c \in C \Rightarrow f(c) > 0$.

La compressione è una mappa che associa ad ogni carattere una stringa di bit:
\[
    % l_F : \Sigma \rightarrow \left\{ 0,1 \right\}^* \\
    e : C \rightarrow \left\{ 0,1 \right\}^*
\]

Le $e(c)$ sono dette parole di codice, o \emph{codeword}.
Le mappe di codifica sono memorizzate in una struttura detta \emph{trie}, un albero binario etichettato.
Per ricostruire il file originale, è sufficiente mantenere due puntatori, uno sul trie e uno sul file codificato, scorrendo il file e scendendo l'albero, tornando alla radice ogni volta che si arriva ad una foglia.

Un esempio di file, codificato con caratteri ASCII a 8 bit, è caratterizzato dalle frequenze dei caratteri che compaiono nel file:
\begin{equation*}
    \begin{array}[H]{rcccccc}
        \texttt{C:} & a & b & c & d & e & f \\
        \texttt{freq rel:} & 0.45 & 0.13 & 0.12 & 0.16 & 0.09 & 0.05
    \end{array}
\end{equation*}
Se $|F| = 100M$ caratteri, per memorizzarlo senza compressione sono necessari $800 Mbit$. Il file usa solo 6 dei 256 caratteri possibili, sono quindi sufficienti meno bit per memorizzarli. In generale, per memorizzare $k$ caratteri sono necessari $\left\lceil \log_2 k \right\rceil$ bit. In questo caso, 3 bit sono sufficienti, e il file si può quindi memorizzare con $300 Mbit$ utilizzando una codifica a lunghezza costante (\emph{fixed-length encoding}).
\begin{equation*}
    \begin{array}[H]{rcccccc}
        \texttt{C:} & a & b & c & d & e & f \\
        \texttt{codifica:} & 000 & 001 & 010 & 011 & 100 & 101
    \end{array}
\end{equation*}
Oltre al file compresso occorre memorizzare anche la mappa di codifica, ma la sua dimensione è generalmente irrisoria rispetto a quella del file.
In \rfig{tree:lungcost} è mostrato il \emph{trie} della codifica a lunghezza costante.
\begin{figure}[h]
\begin{center}
\begin{tikzpicture}[
        level/.style={sibling distance=60mm/#1},
        redge/.style={right,draw=none},
        ledge/.style={left,draw=none},
        inner/.style={draw,circle,fill=black,inner sep=0pt,minimum width=0pt},
        % inner/.style={inner sep=0pt,minimum width=0pt},
        % inner/.style={inner sep=0pt,minimum width=-1pt},
        leaf/.style={circle,draw,minimum width=20pt},
        ]
    \node (z) [inner] {}
    child {
        node [inner] (a) {}
        child {
            node [inner] (b) {}
                child {
                    node [leaf] (b1) {a}
                    edge from parent node[ledge] {0}
                }
                child {
                    node [leaf] (b2) {b}
                    edge from parent node[redge] {1}
                }
            edge from parent node[ledge] {0}
        }
        child {
            node [inner] (e) {}
                child {
                    node [leaf] (e2) {c}
                    edge from parent node[ledge] {0}
                }
                child {
                    node [leaf] (e2) {d}
                    edge from parent node[redge] {1}
                }
            edge from parent node[redge] {1}
        }
        edge from parent node[ledge] {0$\;\;$} % edge da (a)
    }
    child {
        node [inner] (c) {}
        child {
            node [inner] (d) {}
                child {
                    node [leaf] (d1) {e}
                    edge from parent node[ledge] {0}
                }
                child {
                    node [leaf] (d2) {f}
                    edge from parent node[redge] {1}
                }
            edge from parent node[ledge] {0}
        }
        child {
            node [inner] (c2) {}
            % edge from parent node[redge] {1}
        }
        edge from parent node[redge] {$\;\;$1}
    }
    ; % end of the node
\end{tikzpicture}
\end{center}
\caption{\emph{Trie} della codifica a lunghezza costante}
\label{tree:lungcost}
\end{figure}
Questo \emph{trie} è chiaramente non ottimo: una delle foglie non è utilizzata.
Assegnare ancora meno bit ad alcune lettere può condurre ad una compressione ancora maggiore, ma bisogna assicurarsi che la codifica risultante sia libera da prefissi (\emph{prefix code}). Per esempio la seguente codifica
\begin{equation*}
    \begin{array}[H]{rcccccc}
        \texttt{C:} & a & b & c & d & e & f \\
        \texttt{codifica:} & 0 & 00 & 01 & 1 & 10 & 11
    \end{array}
\end{equation*}
non permette di distinguere $e(`ad \textit{'})$ e $e(`c \textit{'})$.
Questa codifica però fornisce il limite inferiore alla grandezza possibile del file compresso:
\begin{equation*}
    |F| \left( 0.45*1 + 0.13*2 + \dots \right) = 153M
\end{equation*}
Per assicurare la decodificabilità deve valere
\begin{equation*}
    \nexists \, c_1, c_2 : e(c_1) \text{ è prefisso di } e(c_2)
\end{equation*}
Si definisce profondità di un carattere $c$ in un \emph{trie} la quantità $d_T(c) = |e(c)|$.
\\
Il costo di una codifica è
\begin{equation*}
    \sum_{c \in C}\left( |F| f(c) \right) d_T(c) = 
    |F| \sum_{c \in C} f(c) d_T(c) = 
    |F| B(T)
\end{equation*}
dove $B(T)$ è la somma della profondità delle foglie del \emph{trie} pesata con le frequenze del carattere. $B(T)$ è anche la lunghezza di \emph{codeword} media.
\\
L'obiettivo è minimizzare $B(T)$ su tutti i possibili \emph{trie} per l'insieme di caratteri $C$.

\subsection{Soluzione \emph{Greedy}}

\subsubsection{Proprietà di un \emph{trie} ottimo}

Prima di poter arrivare alla soluzione, si deve dimostrare la seguente proprietà: un \emph{trie} ottimo $T^*$ per $\{C,f\}$ è sempre un albero binario pieno.

La dimostrazione si svolge per assurdo, ipotizzando che esista un albero ottimo \emph{non} pieno. In questo caso deve esistere un nodo $u$ con un solo figlio $v$. Questo nodo è radice di un sottoalbero, che ha come foglia $c'$ (se fosse composto solo da $v$, $c'$ è scelto coincidente con $v$).
\\
Si possono unire i due nodi $u$ e $v$ in un singolo nodo, ottenendo l'albero $\widehat{T}$. Questo è un albero ammissibile per $\{C,f\}$, infatti non scompaiono foglie, e vale
\begin{equation*}
    \forall c \in C : d_{T^*} (c) \geq d_{\widehat{T}} (c)
\end{equation*}
Infatti al di fuori del sottoalbero resta tutto inalterato, mentre per le foglie del sottoalbero la profondità scende di 1, e questo succede per \emph{almeno} una foglia $c'$: $d_{\widehat{T}} (c') = d_{T^*} (c') -1$
\begin{align*}
    B \left( \widehat{T} \right)
    &= \sum_{c \in C} d_{\widehat{T}} (c) f(c) \\
    &= \sum_{c \in C \setminus \left\{ c' \right\}} d_{\widehat{T}} (c) f(c) + d_{\widehat{T}} (c') f(c') \\
    &\leq \sum_{c \in C \setminus \left\{ c' \right\}} d_{T^*} (c) f(c) 
    + d_{T^*} (c') f(c') - f(c') \\
    &= \sum_{c \in C} d_{T^*} (c) f(c) - f(c') \\
    &= B \left( T^* \right) - f(c')
\end{align*}
Ma questo è assurdo perché $f(c')>0$, avendo scartato i caratteri a costo nullo, e $T^*$ era l'albero ottimo.

Si può quindi riscrivere l'albero a lunghezza fissa come

\begin{figure}[h]
\begin{center}
\begin{tikzpicture}[
        level/.style={sibling distance=60mm/#1},
        redge/.style={right,draw=none},
        ledge/.style={left,draw=none},
        inner/.style={draw,circle,fill=black,inner sep=0pt,minimum width=0pt},
        % inner/.style={inner sep=0pt,minimum width=0pt},
        % inner/.style={inner sep=0pt,minimum width=-1pt},
        leaf/.style={circle,draw,minimum width=20pt},
        ]
    \node (z) [inner] {}
    child {
        node [inner] (a) {}
        child {
            node [inner] (b) {}
                child {
                    node [leaf] (b1) {a}
                    edge from parent node[ledge] {0}
                }
                child {
                    node [leaf] (b2) {b}
                    edge from parent node[redge] {1}
                }
            edge from parent node[ledge] {0}
        }
        child {
            node [inner] (e) {}
                child {
                    node [leaf] (e2) {c}
                    edge from parent node[ledge] {0}
                }
                child {
                    node [leaf] (e2) {d}
                    edge from parent node[redge] {1}
                }
            edge from parent node[redge] {1}
        }
        edge from parent node[ledge] {0$\;\;$} % edge da (a)
    }
    child {
        node [inner] (c) {}
            child {
                node [leaf] (d1) {e}
                edge from parent node[ledge] {0}
            }
            child {
                node [leaf] (d2) {f}
                edge from parent node[redge] {1}
            }
        edge from parent node[redge] {$\;\;$1}
    }
    ; % end of the node
\end{tikzpicture}
\end{center}
\caption{\emph{Trie} subottimo ricavato dalla codifica a lunghezza costante}
\label{tree:lungcostimproved}
\end{figure}

\subsubsection{Soluzione \emph{Greedy}}
Si è quindi definito un problema di ottimizzazione molto chiaro: va determinato un albero pieno ottimo che minimizzi $B(T)$:
\begin{equation*}
    T^* = \argmin \left\{ B(T) : T \right\}
\end{equation*}
L'albero ottimo viene generato compiendo successive operazioni di \emph{merge} tra sottoalberi, creando di volta in volta un nuovo nodo interno.
Si inizia con $k=n=|C|$ alberi, e ad ogni iterazione si passa da $k$ a $k-1$ alberi. Dopo $n-1$ iterazioni si conclude, avendo ottenuto un singolo albero con $n$ foglie.

L'intuizione su cui si basa la scelta degli alberi da unire si basa sul fatto che quando si compie un \emph{merge} la profondità delle foglie dei sottoalberi coinvolti viene incrementata di 1: ha quindi probabilmente senso scegliere ogni volta le due con frequenza minore.
\\
Se $x$ e $y$ sono le due foglie con frequenza minore ($f(x)<f(y)<\dots$), dopo l'unione condividono tutto il prefisso e sono differenziate dall'ultimo carattere. Si può quindi considerare un unico pseudocarattere $z$ associato al prefisso, con $f(z) = f(x)+f(y)$. In questo modo si hanno $n-2$ caratteri non modificati e un carattere nuovo, generando un problema di taglia $n-1$.
\\
Per convenzione, si associa l'etichetta $0$ alla foglia con frequenza minore, e si chiama quel ramo \emph{sinistro}, indipendentemente dalla rappresentazione grafica.

Memorizzando le frequenze in una coda di priorità l'operazione di ricerca del minimo viene eseguita in $O(1)$ mentre l'inserimento di un nuovo nodo si compie in $O(\log n)$.

La classe nodo ha i seguenti attributi: $left$ e $right$ per memorizzare i nodi figli; per le foglie, viene memorizzato il valore speciale $nil$. $f$ indica la frequenza combinata del nodo; $char$, per le foglie, memorizza il carattere.

Si può quindi procedere alla scrittura dell'algoritmo di \emph{Huffman}.

\begin{algorithm}[H]
\caption{Algoritmo di \emph{Huffman}}\label{alg:huff}
\begin{algorithmic}[1]
    \Procedure{HUFFMAN}{$C, f$}
        \State $n \gets |C|$
        \State $Q \gets \emptyset$
        \ForAll{$c \in C$}
            \State $z \gets \Call{newnode}{ }$
            \State $z.f \gets f(c)$
            \State $z.char \gets c$
            \State $z.left \gets z.right \gets nil$
            \State $Q.insert(z)$
            \Comment{Le priorità in $Q$ sono le frequenze}
        \EndFor
        \For{$i \gets 1$ to $n-1$}
            \State $x \gets Q.\Call{extractmin}{ }$
            \State $y \gets Q.\Call{extractmin}{ }$
            \State $z \gets \Call{newnode}{ }$
            \State $z.left \gets x$
            \State $z.right \gets y$
            \State $z.f \gets x.f + y.f$
            \State $Q.\Call{insert}{z}$
        \EndFor
        \State return $Q.\Call{extractmin}{ }$
    \EndProcedure
\end{algorithmic}
\end{algorithm}

\begin{figure}[h]
\centering % to center the subfigures
% \begin{center}
\begin{subfigure}[b]{0.9\textwidth}
\centering % to center the tikzpicture
\begin{tikzpicture}[
        level/.style={sibling distance=60mm/#1},
        redge/.style={right,draw=none},
        ledge/.style={left,draw=none},
        inner/.style={draw,circle,fill=black,inner sep=0pt,minimum width=0pt},
        subtree/.style={draw,circle,minimum width=20pt},
        % inner/.style={inner sep=0pt,minimum width=0pt},
        % inner/.style={inner sep=0pt,minimum width=-1pt},
        leaf/.style={circle,draw,minimum width=20pt},
        ]
    \node (a) [subtree, label=below right:{$0.45$}] at (0,0) {a};
    \node (b) [subtree, label=below right:{$0.15$}] at (2,0) {b};
    \node (c) [subtree, label=below right:{$0.40$}] at (4,0) {c};
    % \node (d) [subtree, label=below right:{$0.16$}] at (6,0) {d};
    % \node (e) [subtree, label=below right:{$0.09$}] at (8,0) {e};
    % \node (f) [subtree, label=below right:{$0.05$}] at (10,0) {f};
\end{tikzpicture}
\caption{Iterazione 0}
\label{fig:huffit0}
\end{subfigure}
\\
\begin{subfigure}[b]{0.9\textwidth}
\centering % to center the tikzpicture
\begin{tikzpicture}[
        % level/.style={sibling distance=60mm/#1},
        % redge/.style={right,draw=none},
        % ledge/.style={left,draw=none},
        % inner/.style={draw,circle,fill=black,inner sep=0pt,minimum width=0pt},
        subtree/.style={draw,circle,minimum width=20pt},
        % inner/.style={inner sep=0pt,minimum width=0pt},
        % inner/.style={inner sep=0pt,minimum width=-1pt},
        % leaf/.style={circle,draw,minimum width=20pt},
        ]
    \node (a) [subtree, label=below right:{$0.45$}] at (0,0) {a};
    \node (b) [subtree, label=below right:{$0.13$}] at (2,0) {b};
    \node (c) [subtree, label=below right:{$0.12$}] at (4,0) {c};
    \node (d) [subtree, label=below right:{$0.16$}] at (6,0) {d};
    \node (ef) [subtree] at (8,0) {$0.15$}
        child {
            node (e) [subtree, label=below right:{$0.09$}] {e}
        }
        child {
            node (e) [subtree, label=below right:{$0.05$}] {f}
        }
    ; % end of ef
\end{tikzpicture}
\caption{Iterazione 1}
\label{fig:huffit1}
\end{subfigure}
\\
\begin{subfigure}[b]{0.9\textwidth}
\centering % to center the tikzpicture
\begin{tikzpicture}[
        % level/.style={sibling distance=60mm/#1},
        % redge/.style={right,draw=none},
        % ledge/.style={left,draw=none},
        % inner/.style={draw,circle,fill=black,inner sep=0pt,minimum width=0pt},
        subtree/.style={draw,circle,minimum width=20pt},
        % inner/.style={inner sep=0pt,minimum width=0pt},
        inner/.style={inner sep=0pt,minimum width=9pt},
        % leaf/.style={circle,draw,minimum width=20pt},
        ]
    \node (a) [subtree, label=below right:{$0.45$}] at (0,0) {a};
    \node (bc) [subtree] at (2,0) {$0.25$}
    % \node (bc) [inner, label=right:{$0.25$}] at (2,0) { }
        child {
            node (b) [subtree, label=below right:{$0.13$}] {b}
        }
        child {
            node (c) [subtree, label=below right:{$0.12$}] {c}
        }
    ; % end of bc
    \node (d) [subtree, label=below right:{$0.16$}] at (6,0) {d};
    \node (ef) [subtree] at (8,0) {$0.15$}
        child {
            node (e) [subtree, label=below right:{$0.09$}] {e}
        }
        child {
            node (e) [subtree, label=below right:{$0.05$}] {f}
        }
    ; % end of ef
\end{tikzpicture}
\caption{Iterazione 2}
\label{fig:huffit2}
\end{subfigure}
% \end{center}
\caption{Alberi di Huffman}
\label{tree:treehuff}
\end{figure}
\subsection{Dimostrazione della correttezza}

\subsubsection{Proprietà di scelta \emph{Greedy}}

\subsubsection{Proprietà di sottostruttura ottima}

\subsection{autocomplete}
Snippet di \LaTeX{} che tornano spesso utili

Una bella scatola:
\begin{equation}
    \boxed{x^2+y^2 = z^2}
\end{equation}

Numeri nei casi
\begin{numcases}{T(n)=}
    2^3 \label{escaso1} \\
    2^4 \label{escaso2} 
\end{numcases}

Sotto numeri
\begin{subnumcases}{T(n)=}
    2^3 \label{escaso3} \\
    2^4 
\end{subnumcases}

Liste compatte
\begin{itemize}[noitemsep,topsep=0pt,parsep=0pt,partopsep=0pt]
    \item qualcosa
    \item[+] qualcosa
    \item[*] qualcosa
    \item[--] qualcosa
\end{itemize}

Parole in libertà per l'autocomplete: 
à
è
ì
ò
ù
perché
così
sì
può
più

Viva vim se scrivi \verb|<C-k>`e| o \verb|<C-k>e`| in insert mode mette una è

% delirio doppio di vim se scrivi \verb|<C-k>da| in insert mode mette ``Hiragana letter DA'' che purtroppo non posso mostrarvi %だ
% insomma i digraph sono tanti e belli

Spazietti fra equazioni
\begin{equation*}
    A^{[0]}(x) = \sum_{j=0}^{\frac{n}{2}-1} a_{2j}x^j
    \quad \text{ e } \quad
    A^{[1]}(x) = \sum_{j=0}^{\frac{n}{2}-1} a_{2j+1}x^j
\end{equation*}

Un gustoso algoritmo
\begin{algorithm}[H]
\caption{Divide and Conquer}\label{alg:dncmock}
\begin{algorithmic}[1]
    \Procedure{D\&C}{$i$}
        \If{$|i| \leq n_0$}                             \Comment{BASE}
            \State *risolvo direttamente*
        \EndIf
        \State $<i_1, i_2, \dots, i_k> \gets A_D(i)$    \Comment{DIVIDE}
        \For{$j \gets 1 $ to $ k $ }                    \Comment{RECURSE}
            \State $s_j \gets $ \Call{D\&C}{$i_j$}
        \EndFor
        \State $s \gets A_C(<s_1, s_2, \dots, s_k>)$    \Comment{CONQUER}
        \State return $s$
    \EndProcedure
\end{algorithmic}
\end{algorithm}

Un alberello
\begin{center}
\begin{tikzpicture}[
        level/.style={sibling distance=60mm/#1},
        redge/.style={right,draw=none},
        ledge/.style={left,draw=none},
        inner/.style={draw,circle,fill=black,inner sep=0pt,minimum width=0pt},
        % inner/.style={inner sep=0pt,minimum width=0pt},
        % inner/.style={inner sep=0pt,minimum width=-1pt},
        leaf/.style={circle,draw},
        ]
    \node (z) [inner] {}
    child {
        node [inner] (a) {}
        child {
            node [inner] (b) {}
                child {
                    node [leaf] (b1) {left}
                    edge from parent node[ledge] {0}
                }
                child {
                    node [leaf] (b2) {righ}
                    edge from parent node[redge] {1}
                }
            edge from parent node[ledge] {0}
        }
        child {
            node [inner] (e) {}
                child {
                    node [leaf] (e2) {left}
                    edge from parent node[ledge] {0}
                }
                child {
                    node [leaf] (e2) {righ}
                    edge from parent node[redge] {1}
                }
            edge from parent node[redge] {1}
        }
        edge from parent node[ledge] {0} % edge da (a)
    }
    child {
        node [inner] (c) {}
        child {
            node [inner] (d) {}
                child {
                    node [leaf] (d1) {left}
                    edge from parent node[ledge] {0}
                }
                child {
                    node [leaf] (d2) {righ}
                    edge from parent node[redge] {1}
                }
            edge from parent node[ledge] {0}
        }
        child {
            node [inner] (f) {}
                child {
                    node [leaf] (f1) {left}
                    edge from parent node[ledge] {0}
                }
                child {
                    node [leaf] (f2) {righ}
                    edge from parent node[redge] {1}
                }
            edge from parent node[redge] {1}
        }
        edge from parent node[redge] {1}
    }
  ; % end of the node
\end{tikzpicture}
\end{center}


\cleardoublepage
% \fancyfoot[LO,RE]{Bibliography}
\lfoot{Bibliography}
\bibliographystyle{unsrt}
\bibliography{biblio}
\cleardoublepage

\appendix
% \fancyfoot[LO,RE]{Appendix \thechapter}
\lfoot{Appendice \thechapter}

\chapter{Appendice}
Appendice con info utili

Dovrebbe chiamarsi Appendice F


\end{document}

%% ci sono cose interessanti su fancyfoot in thesis-example
