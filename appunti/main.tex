\documentclass[a4paper,oneside]{book}
\usepackage[utf8]{inputenc}
\usepackage[italian]{babel}
\usepackage{amsmath,amssymb,amsfonts,amsthm}
\usepackage{mathtools} % dcases
\usepackage{marginnote}
\usepackage[top=3cm, bottom=3cm, heightrounded, marginparwidth=2.5cm, marginparsep=1.5cm]{geometry}
\usepackage{graphicx}
\graphicspath{ {images/} }
\usepackage[nottoc]{tocbibind}	% to generate Bibliography entry in toc
\usepackage{lipsum}
\usepackage{bm}
\usepackage{fancyhdr}
\usepackage{hyperref}	% enable hyperlinks to referenced elements
\hypersetup{
	colorlinks=true,
	linkcolor=cyan,
	citecolor=cyan,
	linkcolor=cyan
}
\usepackage{enumitem} % more control over lists: stackoverflow.com/a/4974583
\usepackage{indentfirst} % YES indent after section/chapter heading
\usepackage[Algoritmo]{algorithm}
\usepackage[noend]{algpseudocode} % algoritmi decenti, senza end
\usepackage[makeroom]{cancel} % strike math
\usepackage{nicefrac} % nice inline fractions
\allowdisplaybreaks % allow breaking equations over pages tex.stackexchange.com/a/431236
\usepackage{cases} % numbering eq inside case env
\usepackage{upquote} % display '` in verbatim

\usepackage{tikz} % for tikzpicture
\usetikzlibrary{shapes} % node shapes


%%   CUSTOM THEOREM/DEF  %%

\theoremstyle{definition}
\newtheorem{definition}{Definizione}[section]

\theoremstyle{theorem}
\newtheorem{theorem}{Teorema}[section]

\theoremstyle{theorem}
\newtheorem{lemma}[theorem]{Lemma}
% \newtheorem{lemma}{Lemma}[section]

%%         END           %%

\title{Appunti di Algoritmi per l'Ingengeria}
\author{lezioni tenute da\\ Geppino Pucci \\ trascritte da \\ Pietro}
% a sto punto se fai un libro crea una pagina di titolo decente
\date{\today}

% magia
% \fancyhead{}
\makeatletter 
\pagestyle{fancy}
% \fancyhead[LE,RO]{\@title}
% \fancyfoot{}
% \fancyfoot[LE,RO]{\thepage}
% \fancyfoot[LO,RE]{Chapter \thechapter}
% \fancyfoot[CO,CE]{\@author}
\lhead{\@title}
\chead{}
\rhead{}
\lfoot{Capitolo \thechapter}
\cfoot{\@author}
\rfoot{\thepage}
\makeatother 

%%   COMANDI    %%
\newcommand{\bpi}{\bm{\Pi}}
% \newcommand{\bi}{I}
\newcommand{\bi}{\mathcal{I}}
\newcommand{\bs}{\mathcal{S}}
\newcommand{\ba}{S}
\newcommand{\bb}{S}
\newcommand{\bc}{S}

\newcommand{\mleq}{\overset{?}{\leq}}
\newcommand{\mgeq}{\overset{?}{\geq}}
\newcommand{\mles}{\overset{?}{<}}
\newcommand{\mges}{\overset{?}{>}}
\newcommand{\meq}{\overset{?}{=}}

\newcommand{\circola}[1]{\mbox{\textcircled{\footnotesize #1}}}
\newcommand{\labeq}[1]{\overset{\circola{#1}}{=}}

\newcommand{\rfig}[1]{Figura~\ref{#1}}
%% MATH OPERATOR %%

\DeclareMathOperator*{\argmin}{arg\,min}

% numerazione sana degli algoritmi
\renewcommand{\thealgorithm}{\arabic{chapter}.\arabic{algorithm}} 

%% FINE COMANDI %%

\begin{document}

\pagestyle{plain}
\pagenumbering{gobble}

% \input{titlepage}
\maketitle
% titolo più serio in thesis-example

\cleardoublepage

\frontmatter % non esiste perche` e` un articolo e non un libro

\chapter*{Ringraziamenti}
\addcontentsline{toc}{chapter}{Ringraziamenti}
Si ringrazia tanta gente


Si cita il CLRS, che è proprio un mattonazzo \cite{Cormen:2009:IAT:1614191} ben fatto


\tableofcontents

\mainmatter

\pagestyle{fancy}

\chapter{Introduzione agli algoritmi}
\section{Introduzione}

Il corso vuole insegnare a progettare ed analizzare algoritmi efficienti.

\begin{description}
    \item{\textbf{Progetto}} ideare una strategia di risoluzione secondo paradigmi generali:
        \begin{itemize}
            \item divide and conquer
            \item dynamic programming
            \item greedy
        \end{itemize}
    \item{\textbf{Analisi}} si valuta la correttezza e la complessità degli algoritmi con prove matematiche
    \item{\textbf{Algoritmi}} si vedranno applicazioni notevoli, negli ambiti del calcolo matematico e la manipolazione di stringhe
    \item{\textbf{Efficienti}} la complessità degli algoritmi è legata alla teoria riguardante la NP-completezza dei problemi
\end{description}

\section{Problema computazionale}

\begin{definition}[Algoritmo]\label{def:alg}
    Un algoritmo è una procedura computazionale finita (terminante) e deterministica, specificata come una sequenza di passi elementari (istruzioni) estratte da un insieme standard associato a un modello computazionale (astrazione di un computer) che trasforma in maniera univoca un ingresso in un uscita.
\end{definition}


\begin{definition}[Problema computazionale]\label{def:probcomp}
    Un problema computazionale $\bpi{}$ è una relazione tra un insieme di istanze $\bi{}$ e un insieme di soluzioni $\bs{}$: $\bpi{} \subseteq \bi{} \times \bs{}$
    \\
    Un problema computazionale definisce la specifica astratta. % ma de che?
\end{definition}

Note:
% \begin{itemize}[noitemsep,topsep=0pt,parsep=0pt,partopsep=0pt]
\begin{itemize}[noitemsep,parsep=0pt,partopsep=0pt]
    \item[--] le seguenti notazioni sono usate in maniera equivalente: $i_1 \bpi{} s_1 \iff (i_1 , s_1 ) \in \bpi{}$
    \item[--] la relazione non è univoca, ad un'istanza possono essere associate più soluzioni
    \item[--] nella specifica astratta si assume esistano le soluzioni per ogni istanza: $\forall i \in \bi{},\exists s \in \bs{} | i \bpi{} s$
\end{itemize}

Un algoritmo $A_{\bpi{}}$ è un algoritmo per $\bpi{}$, ossia risolve $\bpi{}$ se, quando i suoi ingressi sono elementi di $\bi{}$, le sue uscite sono gli elementi di $\bs{}$ in relazione all'ingresso, formalmente:

$$ A: \bi{} \to \bs{} \quad \textrm{e} \quad A(i)=s \iff i \, \bpi{} \, s $$

L'algoritmo realizza in modo procedurale la relazione specificata in modo astratto dal problema computazionale, ossia risolve istanze di un problema computazionale. Più precisamente, un algoritmo è scritto per un modello di computazione, non lavora con istanze astratte ma con le loro codifiche.

\section{Analisi di correttezza e complessità}

\subsection{Analisi di correttezza}

Occorre provare il teorema di correttezza:
$$ \forall i \in \bi{} \: : \: i \, \bpi{} \, A(i) $$

\subsection{Analisi di complessità}

\subsubsection{Modello di costo}
Ad ogni istruzione elementare viene associato un costo. Scegliendo $c \in \mathbb{R} \cup \{ 0 \} $ l'analisi risulta eccessivamente complessa. I possibili costi vengono quindi limitati a $c \in \{ 0,1 \} $, assegnando $1$ solo alle operazioni che determinano effettivamente la complessità. Chiaramente vanno compiute scelte ragionevoli.

La complessità di eseguire l'algoritmo $A_{\bpi{}}$ su $i \in \bi{}$ si definisce come
$$t_{A_{\bpi{},i}}=\;somma\;dei\;costi\;associati\;alle\;istruzioni\;eseguite\;per\;ottenere\;A_{\bpi{}}(i)\;$$

\subsubsection{Taglia di un'istanza}
Calcolare la complessità per ogni singola istanza risulta nuovamente troppo complesso. L'insieme delle instanze viene quindi partizionato raggruppando tutte le istanze di taglia uguale.

\begin{definition}[Taglia di un'istanza]\label{def:taglia}
    La taglia di un'istanza è una misura intera, non negativa, ragionevole, della lunghezza associata a una qualche codifica ragionevole di una data istanza.\\
    $$|\cdot|:\bi{}\to \mathbb{N} \cup \{0\}$$
\end{definition}

Le istanze di taglia uguale vengono raccolte, partizionando l'insieme $\bi{}$
$$ \bi{}_{n} = \left\{ i \in \bi{} : |i|=n \right\} \quad n \geq 0$$ 

Si definisce una metrica sintetica, la funzione di complessità, che è associata alla taglia delle istanze.
$$ T_{A_{\bpi{}}} : \mathbb{N} \cup \{0\} \to \mathbb{R}^{+} \cup \{0\}$$

\textbf{Nota bene:} l'ingresso della funzione complessità deve essere intero.

Si possono definire diverse funzioni complessità:

\begin{align*}
\textrm{\textbf{Caso peggiore:}} &\quad T_{A_{\bpi{}}}^{WORST} (n) = \sup_{i \in \bi{}_{n}} \{t_{A_{\bpi{},i}} \} \\
\textrm{\textbf{Caso migliore:}} &\quad T_{A_{\bpi{}}}^{BEST} (n) = \inf_{i \in \bi{}_{n}} \{t_{A_{\bpi{},i}} \} \\
\textrm{\textbf{Caso medio:}} &\quad T_{A_{\bpi{}}}^{AVE} (n) = \mathbb{E} \left[t_{A_{\bpi{},i}} \right]
\end{align*}



\chapter{Paradigma Divide and Conquer}
% \section{Paradigma divide and conquer}

Seguendo il paradigma \textit{Divide and Conquer}, si cerca una soluzione ad una data istanza in funzione delle soluzioni a determinate istanze \textbf{più piccole}, dette sottoistanze.

% \subsection{Stella di Kleene}
% FORMATTA potrebbe non essere una sezione
\section{Stella di Kleene}
\begin{definition}[Stella di Kleene]\label{def:kleene}
    La stella di Kleene viene definita come l'insieme di tutte le sequenze finite di elementi di $A$.
    $$ A^* = \left\{ < a_1, a_2, \dots, a_k > \: : k \geq 0, \: a_i \in A, \: 1 \leq i \leq k \right\}$$
    La sequenza vuota viene indicata con $ < \; > \: = \epsilon $
\end{definition}

% \subsection{Proprietà di sottostruttura}
\section{Proprietà di sottostruttura}

La proprietà di sottostruttura è una proprietà del problema $\bpi{} \subseteq \bi{} \times \bs{} $ che permette di definire il divide and conquer. Devono esistere due funzioni per avere la proprietà di sottostruttura:

\begin{description}
    \item{Funzione di \textbf{Divisione}} $ D : \bi{} \to \bi{}^*$
    \item{Funzione di \textbf{Ricombinazione}} $ C : \bs{}^* \to \bs{}$
\end{description}

E devono valere le due seguenti proprietà:
\begin{itemize}
    \item $\exists n_0 \in \mathbb{N} \cup \{0\} : \forall i \in \bi{}, |i| > n_0 $
        \begin{itemize}
            \item[--] $ D(i)=<i_1, i_2, \dots, i_k>:|i_j|<|i|, 1 \leq j \leq k$ \\
                (le sottoistanze sono strettamente più piccole dell'istanza originale)
            \item[--] data $ <s_1, s_2, \dots, s_k> \in \bs{}^* $ con $i_j \bpi{} s_j,  1 \leq j \leq k \Rightarrow i \bpi{} C(<s_1, s_2, \dots, s_k>)$ \\
                (la ricombinazione delle soluzioni alle sottoistanze è in relazione con l'istanza originale)
        \end{itemize}
    \item è possibile la risoluzione diretta delle istanze $ i \in \bi{} : |i| \leq n_0 $
\end{itemize}

Note:
% \begin{itemize}
\begin{itemize}[noitemsep,parsep=0pt,partopsep=0pt]
    \item[--] il numero di sottoistanze può variare, in base all'istanza considerata
    \item[--] gli algoritmi che implementano (con sequenze di istruzioni elementari) le funzioni, sono, formalmente, oggetti matematici diversi \\
        $ A_D : i \mapsto <i_1, i_2, \dots, i_k> \sim D(i)$ e $ A_C : <s_1, s_2, \dots, s_k> \mapsto s \sim C(<s_1, s_2, \dots, s_k>) $
\end{itemize}

% \subsubsection{Problema del Sorting di interi - definizione}
\subsection{Problema del Sorting di interi - definizione}

Introduciamo il problema dell'ordinamento di una sequenza di numeri interi, che verrà usato anche in seguito per esemplificare i concetti teorici esposti.

Come insiemi definiamo $\bi{} = \mathbb{Z}^* $ e $\bs{} = \mathbb{Z}^* $. Non tutti gli elementi di $\bs{}$ saranno associati ad un'istanza.

Quando $ (i,s) \in \bpi{}_{SORT} $ ? A parte il caso degenere $i=s=\epsilon$, deve esistere una corrispondenza biunivoca fra gli indici per cui
\begin{itemize}[noitemsep,parsep=0pt,partopsep=0pt]
    \item[--] $i = <i_1, i_2, \dots, i_k> $ e $ s = <s_1, s_2, \dots, s_k> $ con $k \geq 1$
    \item[--] $ \exists \;\phi:\{1, \dots, n\} \leftrightarrow \{1, \dots, n\} \;|\; s_i = i_{\phi(i)}$
    \item[--] $ s_1 \leq s_2 \leq \dots \leq s_k$ 
\end{itemize}

% \subsection{Paradigma Divide and Conquer}
\section{Prototipo Divide and Conquer}

\begin{algorithm}[H]
\caption{Divide and Conquer}\label{alg:dnc}
\begin{algorithmic}[1]
    \Procedure{D\&C}{$i$}
        \If{$|i| \leq n_0$}                             \Comment{BASE}
            \State *risolvo direttamente*
        \EndIf
        \State $<i_1, i_2, \dots, i_k> \gets A_D(i)$    \Comment{DIVIDE}
        \For{$j \gets 1 $ to $ k $ }                    \Comment{RECURSE}
            \State $s_j \gets $ \Call{D\&C}{$i_j$}
        \EndFor
        \State $s \gets A_C(<s_1, s_2, \dots, s_k>)$    \Comment{CONQUER}
        \State return $s$
    \EndProcedure
\end{algorithmic}
\end{algorithm}

% \subsubsection{Albero delle chiamate}
\subsection{Albero delle chiamate}
Nel corso del D\&C, si alternano chiamate all'algoritmo di Divide, e quindi una fase di espansione, generazione \textit{top-down} delle sottoistanze, e chiamate all'algoritmo di Conquer, associato a fasi di contrazione, risoluzione \textit{bottom-up}. L'albero non viene generato tutto contemporaneamente, ma viene creato (e distrutto) nel corso dell'algoritmo, seguendo il cammino di una visita anticipata (\textit{depth-first search}).

Il fatto che nella fase di Conquer venga eliminata la porzione di albero generato per risolvere una sottoistanza, è uno dei difetti principali di questo paradigma. Il non ricordare le soluzioni parziali trovate lo rende un processo computazionale \textit{memoryless}.

TODO pagina 4, albero delle chiamate, disegnini e grafichetti

% \subsubsection{Esempio algoritmo D\&C, ricerca del massimo}
\subsection{Esempio algoritmo D\&C: ricerca del massimo}

Come esempio di algoritmo D\&C, viene presentata una procedura che trova il massimo intero in una sequenza di interi $A[1 \dots n] \in \mathbb{Z}^*$

\begin{algorithm}[H]
\caption{Massimo}\label{alg:max}
\begin{algorithmic}[1]
    \Procedure{MAX}{$A,i,j$}                            
        \If{$i=j$}                             \Comment{BASE}
            \State return $A[i]$
        \EndIf
        \State $k \gets \left\lfloor \frac{i+j}{2} \right\rfloor$    \Comment{DIVIDE - punto di mezzo discreto}
        \State $m_1 \gets $ \Call{MAX}{$A,i,k$} \Comment{RECURSE}
        \State $m_1 \gets $ \Call{MAX}{$A,k+1,j$}
        \If{$m_1 \geq m_2$}              \Comment{CONQUER}
            \State return $m_1$
        \Else
            \State return $m_2$
        \EndIf
    \EndProcedure
\end{algorithmic}
\end{algorithm}

Nota: Nella firma della procedura, sono presenti tutti i parametri necessari a identificare la \textit{generica} sottostruttura

% \subsection{Correttezza del DnC}
\section{Correttezza del DnC}

La correttezza di un algoritmo Divide and Conquer viene provata con un'induzione sulla taglia delle istanze.

\begin{description}
    \item{\textbf{Base:}} provare la correttezza della soluzione diretta per $|i| \leq n_0$
    \item{\textbf{HP:}} formulare l'ipotesi induttiva, assunta come corretta per  $|i| < n$, con $n > n_0$
    \item{\textbf{TH:}} sfruttando l'ipotesi induttiva, dimostrare che la tesi è valida per $|i| = n$.
        La dimostrazione si articola in più parti, in cui viene anche provata la proprietà di sottostruttura.
        \begin{enumerate}[noitemsep,topsep=0pt,parsep=0pt,partopsep=0pt]
            \item $\forall i, D(i) = <i_1, \dots, i_k> \Rightarrow |i_j| < |i|$ \\
                verificare che il divide contragga davvero
            \item date $s_j \Rightarrow i_j \bpi{} s_j, 1 \leq j \leq k$ (vero per l'ipotesi induttiva)
            \item data $s = A_C (s_1, \dots, s_k) \Rightarrow i \bpi{} s$ \\
                correttezza del Conquer
        \end{enumerate}
\end{description}

\subsection{Esempio: correttezza dell'algoritmo MAX}

Per prima cosa va identificata la taglia di una generica sottoistanza in funzione dei parametri che la caratterizzano: $n=j-i+1$


\begin{description}
    \item{\textbf{Base:}} $n=1 \rightarrow j-i+1=1 \rightarrow j=i \rightarrow return \: A[i]$: corretto, $max\{a\}=a$
    \item{\textbf{HP:}} MAX viene considerato corretto per i casi $j-i+1 < n$
    \item{\textbf{TH:}} $j-i+1 = n$, $n>1$ e quindi anche $i<j$
        \begin{enumerate}[noitemsep,topsep=0pt,parsep=0pt,partopsep=0pt]
            \item sottoistanze di taglia minore e non vuote
                % $k= \left\lfloor \frac{i+j}{2}\right\rfloor$, $|i_1| = k-i+1$, $|i_2| = j-(k+i)+1 = j-k$
                % \begin{align*}
                    % k&= \left\lfloor \frac{i+j}{2}\right\rfloor & |i_1| &= k-i+1 \\
                     % &                                          & |i_2| &= j-(k+i)+1 = j-k
                % \end{align*}
                \begin{align*}
                    k&= \left\lfloor \frac{i+j}{2}\right\rfloor & |i_1| &= k-i+1  & |i_2| &= j-(k+i)+1 = j-k
                \end{align*}
                si possono scrivere maggiorazioni e minorazioni su $k$, sfruttando $i \leq j-1$
                    \begin{gather*}
                        k= \left\lfloor \frac{i+j}{2}\right\rfloor \leq
                        \left\lfloor \frac{(j-1)+j}{2}\right\rfloor = 
                        \left\lfloor \frac{2j-1}{2}\right\rfloor \leq j-1 \\
                        i= \left\lfloor \frac{2i+1}{2}\right\rfloor
                        = \left\lfloor \frac{i+(i+1)}{2}\right\rfloor 
                        \leq \left\lfloor \frac{i+j}{2}\right\rfloor 
                        \leq k
                    \end{gather*}
                da cui 
                \begin{equation*}
                    \begin{split}
                    i \leq &k \leq j-1 \\
                    i-i+1 \leq &k-i+1 \leq j-1-i+1 \\
                    0 < 1 \leq &k-i+1 \leq j-i < j-i+1 = n\\
                    0 < &|i_1| < n
                    \end{split}
                \end{equation*}
            \item 
                $m_1 = MAX(A,i,k) = max(\{A(x):i \leq x \leq k \}) $ \\
                $m_2 = MAX(A,k+1,j) = max(\{A(x):k+1 \leq x \leq j \}) $ \\
                considerati corretti per l'ipotesi induttiva
            \item sfruttando la proprietà $ A = A_1 \cup A_2 \Rightarrow max\{A\} = max\{max\{A_1\}, max\{A_2\}\}$,
                basta verificare che le due sottoistanze coprano tutto A, e lo fanno: $(i,k)(k+1,j)$
        \end{enumerate}
\end{description}

% \subsection{Analisi della complessità}
\section{Analisi della complessità}

Relazione di ricorrenza, pag 6

Induzione parametrica - guess, pag 7.5

Esempio guess MAX, pag 9.5

Esempio guess errata, pag 10

Esempio guess cn più complessa, pag 10.5

Verifica per induzione correttezza cn, pag 11.5 (esercizio)

In versione cn+d, pag 11.5 (esercizio)

Complessità dall'albero delle ricorrenze
% forse nel prossimo capitolo

% \subsubsection{autocomplete}
\subsection{autocomplete}

Il \textit{merge}, cuore dell'algoritmo \ref{alg:mergesort}, avviene alla riga \ref{alg:mergeline}, quindi guarda a \algref{alg:mergesort}{alg:mergeline} per saperne di più.

\begin{algorithm}[h]
\caption{MergeSort}\label{alg:mergesort}
\begin{algorithmic}[1]
    \Procedure{MergeSort}{$A,p,r$}
    \If{$p<r$}
        \State $ q \gets \left\lfloor \frac{p+r}{2} \right\rfloor $ 
        \State \Call{MergeSort}{$A,p,q$} \Comment{Ordina prima}
        \State \Call{MergeSort}{$A,q+1,r$} \Comment{Ordina dopo}
        \State
            \Call{Merge}{$A,p,q,r$}
            \label{alg:mergeline}
            \Comment{Merge in $\Theta(n)$}
    \EndIf
    \EndProcedure
\end{algorithmic}
\end{algorithm}

àg
èg
ìg
òg
ùg


\chapter{Master Theorem}
\section{Contrazioni}
% definizione contrazioni, pag 13.5
\begin{definition}[Contrazione]\label{def:contraz}
    Una contrazione è una funzione $f: \mathbb{N} \rightarrow \mathbb{N}$, per cui vale
    \[ f(n) < n \quad \forall n>n_0 \]
\end{definition}

\subsection{Iterate}
% definizione iterata, pag 13.6
\begin{definition}[Iterata]\label{def:iterata}
    Ad una contrazione sono associate le iterate di $f(n)$
    \[
    \begin{cases} 
        f^{(0)} (n) = n      &  i = 0 \\
        f^{(i)} (n) = f \left( f^{(i-i)} (n) \right)   &  i > 0 \\
        
    \end{cases}
    \]
\end{definition}
Nota: le iterate formano una successione decrescente $ f^{(0)} (n) > f^{(1)} (n) > \cdots > f^{(i)} (n) $ \\
Nella prossima sezione, l'iterata $0$ sarà associata alla radice dell'abero e all'istanza generale, l'iterata $i-esima$ rappresenterà la taglia al livello $i$ dell'albero.

\subsection{Ausiliaria}
% Def ausiliaria, pag 14.6
\begin{definition}[Ausiliaria]\label{def:ausiliaria}
    Alle iterate di una funzione si associa anche una funzione ausiliaria, che indica il maggior indice di iterata per cui il valore è ancora maggiore del valore di base. Nell'albero delle ricorrenze, questo indicherà l'ultimo livello.
    \[ f^*(n, n_0) = \max \{ i>0 : f^{(i)}(n) > n_0 \} \]
    La funzione è definita solo per $n>n_0$, per convenzione assume il valore $f^*(n,n_0)=-1$ se $ n \leq n_0$
\end{definition}

\subsection{Esempi}\label{es_iter_aus}

\subsubsection{$\bm{f(n) = n/2}$}
% pag 13.9, 14.8
Calcoliamo la forma esplicita dell'iterata e la funzione ausiliaria per
\[ f(n) = \frac{n}{2} \quad \text{con} \quad n=2^k \]
Forma esplicita iterata:
\begin{align*}
    & f^{(0)}(n) = n \\
    & f^{(1)}(n) = f(n) = \frac{n}{2} \\
    & f^{(2)}(n) = f \left( f^{(1)}(n) \right) = f \left( \frac{n}{2} \right) = \frac{n}{4} = \frac{n}{2^2} \\
    & f^{(i)}(n) = \frac{n}{2^i} 
\end{align*}
Dove la generalizzazione  è valida perché gli argomenti restano potenze di due: $n/2^i = 2^{k-i}$ \\
Calcolo funzione ausiliaria:
\begin{align*}
    & f^{(i)}(n) \mgeq n_0 
        & \text{per quali $i\:$?}\\
    & \frac{n}{2^i} \mges n_0 \\
    & 2^i \mles \frac{n}{n_0} \\
    & i < \log_2 \left( \frac{n}{n_0} \right)
        & \text{il vincolo è risolto} \\
    \rightarrow \quad & f^*(n, n_0) = \log_2 \left( \frac{n}{n_0} \right) - 1
        & \text{ne prendo il massimo} \\
    n_0 = 1 \rightarrow \quad & f^*(n, 1) = \log_2 \left( n \right) - 1
        & \text{se $n_0=1$}
\end{align*}

\subsubsection{$\bm{f(n) = n-1}$}
% pag 14, 15
Calcoliamo la forma esplicita dell'iterata e la funzione ausiliaria per
\[ f(n) = n-1 \]
Forma esplicita iterata:
\begin{align*}
    & f^{(0)}(n) = n \\
    & f^{(1)}(n) = f(n) = n-1 \\
    & f^{(2)}(n) = f \left( f^{(1)}(n) \right) = f \left( n-1 \right) = n-1-1 = n-2 \\
    & f^{(i)}(n) = n-i
\end{align*}
Calcolo funzione ausiliaria:
\begin{align*}
    & f^{(i)}(n) \mgeq n_0 
        & \text{per quali $i\:$?}\\
    & n-i \mges n_0 \\
    & i < n - n_0
        & \text{il vincolo è risolto} \\
    \rightarrow \quad & f^*(n, n_0) = n - n_0 - 1
        & \text{ne prendo il massimo} \\
    n_0 = 1 \rightarrow \quad & f^*(n, 1) = n - 2
        & \text{se $n_0=1$}
\end{align*}

\subsubsection{$\bm{f(n) = \sqrt{n}}$}
% pag 14.2, 15.2
Calcoliamo la forma esplicita dell'iterata e la funzione ausiliaria per
\[ f(n) = \sqrt{n} \quad \text{con} \quad n=2^{2^k} \]
Forma esplicita iterata:
\begin{align*}
    & f^{(0)}(n) = n \\
    & f^{(1)}(n) = f(n) = \sqrt{n} = n^{\nicefrac{1}{2}} \\
    & f^{(2)}(n) = f \left( f^{(1)}(n) \right) = f \left( n^{\nicefrac{1}{2}} \right) = n^{\nicefrac{1}{2^2}} \\
    & f^{(i)}(n) = n^{\nicefrac{1}{2^i}} \\
\end{align*}
Calcolo funzione ausiliaria:
\begin{align*}
    & f^{(i)}(n) \mgeq n_0 
        & \text{per quali $i\:$?}\\
    & n^{\nicefrac{1}{2^i}} \mges n_0 \\
    & \log_2 n^{\nicefrac{1}{2^i}} \mges \log_2 n_0 \\
    & \frac{1}{2^i} \log_2 n \mges \log_2 n_0 \\
    & 2^i \mles \frac{\log_2 n}{\log_2 n_0} \\
    & i < \log_2 \frac{\log_2 n}{\log_2 n_0} 
        & \text{il vincolo è risolto} \\
    & i < \log_2 \log_2 n + \log_2 \log_2 n_0 \\
    \rightarrow \quad & f^*(n, n_0) = \log_2 \log_2 n + \log_2 \log_2 n_0 - 1
        & \text{ne prendo il massimo} \\
    n_0 = 2 \rightarrow \quad & f^*(n, 2) = \log_2 \log_2 n - 1
        & \text{se $n_0=2$}
\end{align*}
Il vincolo su $n$ può essere generalizzato a $n=a^{2^k}$, in questo caso $n_0 = a$

\subsubsection{$\bm{f(n) = \left\lfloor n/2 \right\rfloor}$}
Nel caso $f(n) = \left\lfloor n/2 \right\rfloor$, si può dimostrare che
$ f^{(2)}(n) = \left\lfloor \left\lfloor n/2 \right\rfloor / 2 \right\rfloor = \left\lfloor n / 2^2 \right\rfloor $
ma non è per niente banale.

\section
[Modifica al \textit{Divide and Conquer}]
{Modifica al \textit{Divide and Conquer} \\ 
{ \large Risoluzione di una classe generale di ricorrenze\\ associate ad algoritmi \textit{Divide and Conquer} } }
% titolo meno informativo della storia
% sottotitolo già meglio
\subsection{Meta-algoritmo}
% Metaalgoritmo DnC modificato, pag 12.5
\begin{algorithm}[H]
\caption{Divide and Conquer modificato}\label{alg:dncm}
\begin{algorithmic}[1]
    \Procedure{D\&C}{$i$}
        \State $n=|i|$
        \If{$n \leq n_0$}                             \Comment{BASE}
            \State *risolvo direttamente*
        \EndIf
        \State $<i_1, i_2, \dots, i_{S(n)}> \gets A_D(i)$    \Comment{DIVIDE}
        \For{$j \gets 1 $ to $ {S(n)} $ }                    \Comment{RECURSE}
            \State $s_j \gets $ \Call{D\&C}{$i_j$}
        \EndFor
        \State $s \gets A_C(<s_1, s_2, \dots, s_{S(n)}>)$    \Comment{CONQUER}
        \State return $s$
    \EndProcedure
\end{algorithmic}
\end{algorithm}
% Parametri noti, pag 13
L'algoritmo ha dei vincoli in più rispetto al prototipo generico, in particolare sul numero e dimensione delle sottoistanze generate. I parametri che lo descrivono sono:
\begin{itemize}[noitemsep,topsep=0pt,parsep=0pt,partopsep=0pt]
    \item[--] $S(n): \mathbb{N} \rightarrow \mathbb{N}$ \\
        istanze di taglia $n$ nella fase di \textit{Divide} generano sempre lo \textbf{stesso} numero di sottoistanze
    \item[--] $ |i_j| = f(n) < n \quad \forall j : 1 \leq j \leq S(n) $ \\
        le sottoistanze sono sempre grandi uguali
    \item[--] $ w(n) = T_{A_D}(n) + T_{A_C}(n)$ \\
        lavoro compiuto per ogni istanza per dividere e ricomporre
    \item[--] $n_0$ taglia del caso base
    \item[--] $T_0$ tempo al caso peggiore di risoluzione diretta quando $|i| \leq n_0$
\end{itemize}
% Nuova eq di ricorrenza, pag 13.3
L'equazione di ricorrenza diventa
\[
    T(n) = 
    \begin{cases} 
        T_0      &  n \leq n_0 \\
        S(n) \, T(f(n)) + w(n) & n > n_0
    \end{cases}
\]
Ad ogni passo vengono fatte $S(n)$ chiamate ricorsive su istanze grandi $f(n)$

\subsubsection{esempio parametri MAX}
% parametri max, pag 13.5
I parametri del $MAX$ sono, con la nuova notazione:\\
$S(n)=2$, $f(n)=n/2$, $w(n)=1$, $T_0=0$, $n_0=1$

\subsection{Equazione delle ricorrenze generica}
% Alberone, pag 15.8
% { \centering
\begin{center}
\begin{tikzpicture}[
  level/.style={sibling distance=45mm},
  every node/.style={rectangle,draw,solid},
  dotme/.style={edge from parent/.style={loosely dotted,draw}},
  norm/.style={edge from parent/.style={solid,draw}}
  ]
  \node [label=right:{$\rightarrow w(n)$}] (z){$n = f^{(0)}(n)$}
    child { node (a) {$f^{(1)}(n)$}
      child { node (c) {$f^{(2)}(n)$}
      }
      child { node [label=right:{$\rightarrow w\left(f^{(2)}(n)\right)$}] (d) {$f^{(2)}(n)$}
        child [dotme] { node [label=right:{$\rightarrow w\left(f^{(l)}(n)\right)$}] 
                (d1) {$f^{(l)}(n)$}
          child [norm] { node (e) {$f^{(l+1)}(n)$}
            child [dotme] { node [label=right:{$\rightarrow
                            % w\left(f^{\left(\overline{l}\right)}(n)\right)$}] 
                            % nessuno dei due è decente
                            w\left(f^{(\overline{l})}(n)\right)$}]
                            (e1) {$f^{(\overline{l})}(n)$}
              child [norm] { node (g) {$\leq n_0$}
              }
              child [norm]{node[label=right:{$\rightarrow T_0$}]
                       (h) {$\leq n_0$}
              }
            }
          }
          child [norm] { node [label=right:{$\rightarrow w\left(f^{(l+1)}(n)\right)$}]
                   (f) {$f^{(l+1)}(n)$}
          }
        }
      }
    }
    child { node [label=right:{$\rightarrow w\left(f^{(1)}(n)\right)$}] (b) {$b f^{(1)}(n)$}
    }
  ; % end of the node
  \path (a) -- (b) node [draw=none,midway] {$\cdots S\left(f^{(0)}(n)\right) \cdots$ };
  \path (c) -- (d) node [draw=none,midway] {$\cdots S\left(f^{(1)}(n)\right) \cdots$ };
  \path (e) -- (f) node [draw=none,midway] {$\cdots S\left(f^{(l)}(n)\right) \cdots$ };
  \path (g) -- (h) node [draw=none,midway] {$\cdots S\left(f^{(l)}(n)\right) \cdots$ };
  % \draw (e) -- (f) node [draw=none,midway,loosely dotted,above=10pt] {$S(f^{(l)}(n)) $ };
\end{tikzpicture}
\end{center}
% } % end centering, LASCIA la riga sopra questa parentesi (serve un \par)
% \usetikzlibrary{positioning}
% \node[nome,right] (p2) [below = of p1] {text text};

Nell'albero la taglia delle istanze è indicata nel nodo, il lavoro associato (uguale per ogni istanza del livello) è indicato a fianco del livello. Il numero di sottoistanze generate da ciascuna istanza di un livello è indicato tra i nodi figli. Si indica con $l \leq f^*(n,n_0)$ un generico livello interno e con $\overline{l} = f^*(n,n_0)$ l'ultimo livello interno. I nodi foglia sono tutti di dimensione minore del caso base, ed è a loro associato il lavoro $T_0$ necessario a risolverli direttamente.

Da questo albero delle ricorrenze si può ricavare il lavoro totale, sommando i contributi di ciascun nodo.
% Formulone, pag 16
\begin{align*}
    T(n) & = \sum_{l=0}^{f^*(n,n_0)}\text{\# nodi nel livello}\cdot\text{\# $w$ per nodo del livello}+\text{contributo foglie}\\
    &= \sum_{l=0}^{f^*(n,n_0)} \left[ \prod_{j=0}^{l-1} S \left( f^{(j)}(n) \right) \cdot w \left( f^{(l)}(n) \right) \right] +
        \prod_{j=0}^{f^*(n,n_0)} S \left( f^{(j)}(n) \right) \cdot T_0
\end{align*}

Da qualche parte le convenzioni sugli operatori, pag 15.5; insieme alle cose a fine capitolo DnC -$>$ Appendice

\subsubsection{Esempio formula}
% Esempio, pag 16.5
Risolviamo l'equazione di ricorrenza utilzzando la formula appena ottenuta.
\[
    T(n) = 
    \begin{cases} 
        4      &  n = 1 \\
        6 \, T\left(\frac{n}{3}\right) + n^2-n & n > 1 \quad \text{assumendo } n=3^k
    \end{cases}
\]
Da cui ricaviamo i valori dei parametri:
\begin{align*}
    S(n)&=6 & f(n)&=\frac{n}{3} & w(n)&=n^2-n & T_0&=4 & n_0&=1
\end{align*}
Le formule di iterata $i-esima$ e ausiliaria si ricavano come spiegato nella sezione \ref{es_iter_aus}:
\begin{align*}
    f^{(i)}(n) &= \frac{n}{3^i} & f^*(n, 1) &= \log_3 \left( n \right) - 1
\end{align*}
Calcoliamo le componenti interne della formula:
\begin{gather*}
    \prod_{j=0}^{l-1} S \left( f^{(j)}(n) \right) = \prod_{j=0}^{l-1} 6 = 6^{l-1+1} = 6^l \\
    w\left(f^{(l)}(n) \right) =  w \left( \frac{n}{3^l} \right) = \frac{n^2}{3^{2l}} - \frac{n}{3^l} = \frac{n^2}{9^l}-\frac{n}{3^l} \\
    \prod_{j=0}^{f^*(n,n_0)} S \left( f^{(j)}(n) \right) = \prod_{j=0}^{\log_3 n-1} 6 = 6^{\log_3 n} =
    n^{\log_3 6} = n^{\log_3 (2\cdot3)} = n^{\log_3 2 + \log_3 3} = n^{\log_3 2 + 1}
\end{gather*}
La formula risulta quindi:
\begin{align*}
    T(n) & = \sum_{l=0}^{\log_3 n-1} 6^l \left( \frac{n^2}{9^l}-\frac{n}{3^l} \right) + 4 n^{\log_3 2 + 1} \\
    & = n^2\,\sum_{l=0}^{\log_3 n-1} \left( \frac{2}{3} \right)^l - n\,\sum_{l=0}^{\log_3 n-1} 2^l+ 4 n^{\log_3 2+1} \\
    & = n^2 \left( \frac{1- \left( 2/3 \right)^{\log_3 n} }{1 - 2/3 } \right) - n(2^{\log_3 n} -1) + 4 n^{\log_3 2+1} 
    % & \left[ \left( \frac{2}{3} \right)^{\log_3 n} &= \frac{2^{\log_3 n}}{3^{\log_3 n}} = \frac{n^{\log_3 2}}{n}\right]\\
    &
    \left[
    \left( \frac{2}{3} \right)^{\log_3 n}
    \right. % tutto questo per avere le parentesi quadre grandi
    &=      % attorno a sto maledetto align
    \left.  % che avrei anche potuto spostare 
    \frac{2^{\log_3 n}}{3^{\log_3 n}} = \frac{n^{\log_3 2}}{n}
    \vphantom{\left( \frac{2}{3} \right)^{\log_3 n}}
    \right] % ma grandi quanto voglio io
    \\
    & = 3n^2 \left( 1- \frac{n^{\log_3 2}}{n} \right) - n \left( n^{\log_3 2}-1 \right) + 4 n^{\log_3 2+1} \\
    & = 3n^2 -3n^{1+\log_3 2} - n^{1+\log_3 2} + n + 4 n^{1+\log_3 2} \\
    & = 3n^2+n
\end{align*}
È una buona idea verificare la correttezza per induzione:\\
Base: 
\[ T(1)=4 \meq \left( 3n^2+n \right) \Big|_{n=1} = 4 \rightarrow \text{vera} \]
Ipotesi: Considero corretta la formula per i valori $s<n$ nella forma $s=3^k$ \\
Tesi:
\begin{align*}
    T(n) &= 6T \left( \frac{n}{3} \right) + n^2 -n \\
    & = 6 \left( 3 \frac{n^2}{9} + \frac{n}{3} \right) \\
    & = 3n^2+n\rightarrow \text{vera}
\end{align*}

\subsubsection{Esempio formula}
% Esempio, pag 17.5
\[
    T(n) = 
    \begin{cases} 
        1      &  n = 2 \\
        2 \, T\left( \sqrt{n}\right) + 1 & n > 2 \quad \text{assumendo } n=2^{2^k} 
    \end{cases}
\]
Le formule di iterata $i-esima$ e ausiliaria si ricavano come spiegato nella sezione \ref{es_iter_aus}:
\begin{align*}
    T(n) &= \sum_{l=0}^{\log_2 \log_2 n-1} 2^l \cdot 1 + 2^{\log_2 \log_2 n} \cdot 1 \\
    &= 2^{\log_2 \log_2 n} -1 + 2^{\log_2 \log_2 n} 
    && \left[ 2^{\log_2 \log_2 n} = \log_2 n ^{\cancel{\log_2 2}} \right] \\
    &= 2 \log_2 n - 1
\end{align*}

\section{\textit{Master Theorem}}
\subsection{Ipotesi}
% Ipotesi, pag 18
Il \textit{Master Theorem} è valido per sottoclassi di ricorrenze in cui:
\begin{enumerate}
    \item $S(n) = a$ con $a$ costante, $a \geq 1$ (almeno un'istanza viene generata)
    \item $f(n) = \frac{n}{b}$ con $b$ costante, $b > 1$ (deve essere una contrazione)
    \item $w(n): \mathbb{N}^+ \rightarrow \mathbb{N}^+$% \setminus \{0\}$ % ???
    \item $T_0 > 1$ ($ T_0 = 0$ è ragionevole ma non vale la dimostrazione)
    \item $n_0 \in \mathbb{N}$
\end{enumerate}
La formula di ricorrenza è quindi nella forma:
\begin{equation}
    T(n) = 
    \begin{cases} 
        T_0      &  n \leq n_0 \\
        a \, T\left( \frac{n}{b} \right) + w(n) & n > n_0 \quad \text{assumendo } n=b^k
    \end{cases}
    \label{eq:masterricorrenza1}
\end{equation}

\subsection{Tesi}
% Tesi, pag 18.5
% \[
% T(n) = 
% % \begin{numcases}
% \begin{cases}
    % \Theta (n^{\log_b a}) & \text{se } \exists \varepsilon > 0 : w(n) = O(n^{\log_b a-\varepsilon})\\
    % \Theta (n^{\log_b a} \, \log_n) & \text{se } w(n) = \Theta(n^{\log_b a})\\
    % \Theta (w(n)) & \text{se }
    % \begin{cases}
    % % \begin{numcases}
        % \exists \varepsilon > 0 : w(n) = \Omega(n^{\log_b a+\varepsilon})\\
        % \exists 0<c<1 : \forall n \in \mathbb{N}-\{0\} : a w \left( \frac{n}{b} \right) < c w(n)
    % % \end{numcases}
    % \end{cases}
% % \end{numcases}
% \end{cases}
% \]

% \begin{numcases}{T(n)=}
    % \Theta (n^{\log_b a}) & se $\exists \varepsilon > 0 : w(n) = O(n^{\log_b a-\varepsilon})$\\
    % \Theta (n^{\log_b a} \, \log_n) & se $w(n) = \Theta(n^{\log_b a})$\\
    % \Theta (w(n)) & condition
    % % \begin{subnumcases}{se}
        % % \exists \varepsilon > 0 : w(n) = \Omega(n^{\log_b a+\varepsilon})\\
        % % \exists 0<c<1 : \forall n \in \mathbb{N}-\{0\} : a w \left( \frac{n}{b} \right) < c w(n)
    % % \end{subnumcases}
% \end{numcases}
% \textbf{Caso 1:} 
% \begin{equation}
    % T(n) = \Theta (n^{\log_b a})  \text{ se } \exists \varepsilon > 0 : w(n) = O(n^{\log_b a-\varepsilon})
    % \label{eq:mastercaso1}
% \end{equation}

Ci sono tre casi possibili:\\
\begin{description}
    \item[\textbf{Caso 1:}] 
        \begin{equation}
            T(n) = \Theta (n^{\log_b a})  \text{ se } \exists \varepsilon > 0 : w(n) = O(n^{\log_b a-\varepsilon})
            \label{eq:mastercaso1}
        \end{equation}
    \item[\textbf{Caso 2:}] 
        \begin{equation}
            T(n) = \Theta (n^{\log_b a} \, \log_n) \text{se } w(n) = \Theta(n^{\log_b a})
            \label{eq:mastercaso2}
        \end{equation}
    \item[\textbf{Caso 3:}] 
        \begin{subnumcases}{T(n) = \Theta (w(n)) \text{ se }}
            \exists \varepsilon > 0 : w(n) = \Omega(n^{\log_b a+\varepsilon})
            \label{eq:mastercaso3a}\\
            \exists 0<c<1 : \forall n \in \mathbb{N} \setminus \{0\} : a w \left( \frac{n}{b} \right) < c w(n)
            \label{eq:mastercaso3b}
        \end{subnumcases}
\end{description}

Note:
\begin{itemize}
    \item[--] $n^{\log_b a}$ è detta funzione di soglia, o \textit{treshold}
    \item[--] la condizione \ref{eq:mastercaso3b} è detta condizione di regolarità, indica che $w(n)$ è non decrescente, e quando cresce lo fa in modo uniforme
\end{itemize}

\subsection{Dimostrazione}
\subsubsection{Considerazioni sui parametri}
% Considerazioni sull'asintotico, pag 18.9, 19
Il parametro $n_0$, con $T_0>0$, rende alcuni passaggi della dimostrazione più difficili. Lo si può assumere come il primo elemento della famiglia $b^k=b^0=1$? Considerando i nodi foglia dell'albero delle ricorrenze, hanno tutti taglia $n_0$ e lavoro associato $T_0$. Continuando a dividere i nodi finché raggiungono taglia $1$, si generano dei sotto-alberi alti $\log_b \left( n_0 \right)$, con nodi foglia di taglia $1$, a ciascuno dei quali è associato ancora il lavoro $T_0$, essendo il lavoro al caso peggiore. Il lavoro associato ai sotto-alberi è quindi $O(1)$ (costante), e senza perdita di generalità si può considerare $n_0=1$.

\subsubsection{Considerazioni sulle relazioni asintotiche}
% Considerazioni sull'asintotico, pag 18.9, 19
Per funzioni intere positive $g: \mathbb{N} \setminus \{0\} \rightarrow  \mathbb{N} \setminus \{0\}$, si può modificare la definizione delle maggiorazioni e minorazioni asintotiche.
\begin{description}
    \item[$\bm{T(n)=O \left( g(n) \right)}$:]
        $T$ è \textit{O-grande} di $g$ ($T$ è asintoticamente minore di $g$) se vale
        $$T(n) = O\left(g(n)\right)\text{ se }\exists c>0,\overline{n}:
            \forall n>\overline{n}\Rightarrow T(n)\leq c\cdot g(n) \qquad \circola{A}$$
        mostriamo che per le funzioni intere positive, $\overline{n}=1$
        $$T(n) = O\left(g(n)\right)\text{ se }\exists c'>0:
            \forall n>1 \Rightarrow T(n)\leq c'\cdot g(n) \qquad \circola{B}$$
        dimostriamo $ \circola{A} \Rightarrow \circola{B}$
        $$ \text{sia } \overline{c} = \max \left\{ c, T(1), T(2), \dots, T( \overline{n} ) \right\} $$
        verifico che $ \forall n>1 \rightarrow T(n) \leq \overline{c} g(n) $
        \begin{itemize}[noitemsep,topsep=0pt,parsep=0pt,partopsep=0pt]
            \item $1 \leq n \leq \overline{n} \rightarrow T(n) \leq \overline{c} \leq \overline{c}g(n) $ \\
                la prima valida per definizione di $ \overline{c} $, la seconda perché $g$ è intera positiva
            \item $ n > \overline{n} \rightarrow T(n) \leq cg(n) \leq \overline{c}g(n)$ \\
                la prima valida perché $T=O(G)$, la seconda per definizione di $\overline{c}$
        \end{itemize}
    \item[$\bm{T(n)=\Omega \left( g(n) \right)}$:]
        $T$ è \textit{omega} di $g$ ($T$ è asintoticamente maggiore di $g$) se vale
        $$T(n) = \Omega \left(g(n)\right)\text{ se }\exists c>0,\overline{n}:
            \forall n>\overline{n}\Rightarrow T(n)\geq c\cdot g(n) \qquad \circola{A}$$
        mostriamo che per le funzioni intere positive, $\overline{n}=1$
        $$T(n) = \Omega \left(g(n)\right)\text{ se }\exists c'>0:
            \forall n>1 \Rightarrow T(n)\geq c'\cdot g(n) \qquad \circola{B}$$
        dimostriamo $ \circola{A} \Rightarrow \circola{B}$
        $$ \text{sia } \overline{c} = \min \left\{ c, \frac{T(1)}{g(1)}, \frac{T(2)}{g(2)} ,
            \dots, \frac{T( \overline{n})}{ g( \overline{n}) } \right\} $$
        verifico che $ \forall n>1 \rightarrow T(n) \geq \overline{c} g(n) $
        \begin{itemize}[noitemsep,topsep=0pt,parsep=0pt,partopsep=0pt]
            \item $1 \leq n \leq \overline{n} \rightarrow T(n) = \frac{T(n)}{g(n)}g(n) \geq \overline{c}g(n) $ \\
                per definizione di $ \overline{c} $ come minimo dei valori
            \item $ n > \overline{n} \rightarrow T(n) \geq cg(n) \geq \overline{c}g(n)$ \\
                la prima valida perché $T=\Omega(G)$, la seconda per definizione di $\overline{c}$
        \end{itemize}
    \item[$\bm{T(n)=\Theta \left( g(n) \right)}$:] $T=\Omega(G)$ e $T=O(g)$ implicano $T=\Theta(g)$
\end{description}

\subsubsection{Costo delle foglie}
Ricordiamo che 
\begin{align*}
    f(n)&= \frac{n}{b} & f^{(j)}(n) &= \frac{n}{b^j} & f^*(n,1) &= \log_b n -1
\end{align*}

Per cui la formula risulta:
\begin{equation}
    \begin{aligned}[b]
        T(n) &= \sum_{l=0}^{\log_b n -1} \left[ a^l \cdot w \left( \frac{n}{b^l} \right) \right] + a^{\log_b n} \cdot T_0 \\
        &= \sum_{l=0}^{\log_b n -1} \left[ a^l \cdot w \left( \frac{n}{b^l} \right) \right] + n^{\log_b a} \cdot T_0
        \label{eq:masterchiusa1}
    \end{aligned}
\end{equation}

In cui possiamo notare la presenza della funzione di soglia come numero di foglie dell'albero. In più, avendo imposto (per ora) $T_0>0$, è triviale dedurre \begin{equation}
    n^{\log_b a} \cdot T_0 = \Theta \left( n^{\log_b a} \right) 
    \label{eq:masterfoglie}
\end{equation}

\subsubsection{Dimostrazione caso 1}
% Dimostrazione, pag 20
Vogliamo dimostrare il caso $1$:
\begin{align*}
    \exists \varepsilon > 0 &: w(n) = O(n^{\log_b a-\varepsilon})
    \Rightarrow
    T(n) = \Theta \left( n^{\log_b a} \right) 
    \intertext{ossia per ipotesi}
    \exists \varepsilon > 0 , \exists c>0 &: \forall n>1 \rightarrow w(n) \leq cn^{\log_b a-\varepsilon} \\
    \intertext{maggioriamo il primo termine della formula \ref{eq:masterchiusa1}, ricavandone il limite superiore asintotico cercato:}
    \sum_{l=0}^{\log_b n -1} a^l \cdot w \left( \frac{n}{b^l} \right)
    & \leq c \sum_{l=0}^{\log_b n -1} a^l \cdot \left( \frac{n}{b^l} \right)^{\log_b a - \varepsilon} 
    & w=O \left( n^{\log_b a - \varepsilon} \right) & \\
    & = c n^{\log_b a - \varepsilon} \sum_{l=0}^{\log_b n -1} \frac{a^l}{b^{l^{\log_b a}} b^{l^{-\varepsilon}}}
    & b^{l^{\log_b a}} = b^{\log_b a^{l}} = a^l & \\
    & = c n^{\log_b a - \varepsilon} \sum_{l=0}^{\log_b n -1} \frac{\cancel{a^l}}{\cancel{a^l} b^{-\varepsilon^l}} \\
    & = c n^{\log_b a - \varepsilon} \sum_{l=0}^{\log_b n -1} b^{\varepsilon^l}
    & \text{se } b>1, \, b^{\varepsilon}>1 & \\
    & = c n^{\log_b a - \varepsilon} \left( \frac{b^{\varepsilon^{\log_b n}} - 1}{b^{\varepsilon} - 1} \right) \\
    & = c n^{\log_b a - \varepsilon} \left( \frac{n^{\varepsilon} - 1}{b^{\varepsilon} - 1} \right) \\
    & \leq c n^{\log_b a - \varepsilon} \left( \frac{n^{\varepsilon} }{b^{\varepsilon} - 1} \right) \\
    & \leq c\frac{ n^{\log_b a - \varepsilon + \varepsilon} }{b^{\varepsilon} - 1} \\
    & \leq \frac{ c }{b^{\varepsilon} - 1} n^{\log_b a}
    & c, b, \varepsilon \text{ sono costanti} &\\
    \intertext{quindi la sommatoria è $O-grande$ di $n^{\log_b a}$}
    \sum_{l=0}^{\log_b n -1} a^l \cdot w \left( \frac{n}{b^l} \right)
    & = O \left( n^{\log_b a} \right)
    \intertext{combinando questo risultato con \ref{eq:masterfoglie} si ottiene}
    T(n) & = O \left( n^{\log_b a} \right) + \Theta \left( n^{\log_b a} \right) 
    \intertext{per le leggi sui vincoli asintotici, $O$ è implicato da $\Theta$}
    T(n) & = \Theta \left( n^{\log_b a} \right) 
    \intertext{Nota: se $T_0=0$, \ref{eq:masterfoglie} non è valida, e risulta nullo il contributo delle foglie}
    T(n) & = O \left( n^{\log_b a} \right)
\end{align*}
In questo caso la maggior parte del lavoro è svolto dai nodi foglia.

\subsubsection{Dimostrazione caso 2}
% Dimostrazione, pag 21
Vogliamo dimostrare il caso $2$:
\begin{align*}
    w(n) = \Theta(n^{\log_b a})
    & \Rightarrow
    T(n) = \Theta \left( n^{\log_b a} \log n \right) 
    \intertext{ossia per ipotesi}
    \exists c_1, c_2 : \forall n \geq 1 & \rightarrow c_1 n^{\log_b a} \leq w(n) \leq n^{\log_b a}
    \intertext{usiamo la seconda disequazione per maggiorare $w$}
    \sum_{l=0}^{\log_b n -1} a^l \cdot w \left( \frac{n}{b^l} \right)
    & \leq c_2 \sum_{l=0}^{\log_b n -1} a^l \left( \frac{n}{b^l} \right)^{\log_b a} \\
    % & = c_2 \sum_{l=0}^{\log_b n -1} \frac{\cancel{a^l}}{\cancel{a^l}} n^{log_b a}
    & = c_2 \sum_{l=0}^{\log_b n -1} \frac{a^l n^{\log_b a}}{a^l} 
    & b^{l^{\log_b a}} = a^l &\\
    & = c_2 n^{\log_b a} \sum_{l=0}^{\log_b n -1} 1 \\
    & = c_2 n^{\log_b a} \log_b n \\
    \intertext{quindi la sommatoria è $O-grande$ di $n^{\log_b a} \log n$}
    \sum_{l=0}^{\log_b n -1} a^l \cdot w \left( \frac{n}{b^l} \right)
    & = O \left( n^{\log_b a} \log n \right) \\
    \intertext{in modo analogo si minora con $c_1$}
    \sum_{l=0}^{\log_b n -1} a^l \cdot w \left( \frac{n}{b^l} \right)
    & \geq c_1 \sum_{l=0}^{\log_b n -1} a^l \left( \frac{n}{b^l} \right)^{\log_b a} = c_1 n^{\log_b a} \log_b n \\
    \intertext{quindi la sommatoria è $\Omega-grande$ di $n^{\log_b a} \log n$}
    \sum_{l=0}^{\log_b n -1} a^l \cdot w \left( \frac{n}{b^l} \right)
    & = \Omega \left( n^{\log_b a} \log n \right) \\
    \intertext{combinando i due risultati}
    \sum_{l=0}^{\log_b n -1} a^l \cdot w \left( \frac{n}{b^l} \right)
    & = \Theta \left( n^{\log_b a} \log n \right) \\
    \intertext{combinando questo risultato con \ref{eq:masterfoglie} si ottiene}
    T(n) & = \Theta \left( n^{\log_b a} \right) + \Theta \left( n^{\log_b a} \log n \right) \\
    \intertext{per le leggi sui vincoli asintotici, il primo contributo è trascurabile}
    T(n) & = \Theta \left( n^{\log_b a} \log n \right) \\
\end{align*}
In questo caso la maggior parte del lavoro è compiuta all'interno dell'albero, ogni livello compie lo stesso lavoro cumulativo, e ci sono $\log n$ livelli. L'esempio principale del caso 2 è il \textit{Merge-Sort}, descritto da $T(n)=2T(n/2)+n$, dove ogni livello compie $n^{\log_2 2}=n$ operazioni e la complessità è $\Theta (n \log n)$.

\subsubsection{Dimostrazione caso 3}
% Dimostrazione, pag 22
Prima di dimostrare il caso 3, dimostriamo il seguente lemma sulla condizione di regolarità estesa
\begin{lemma}
    Se una funzione è regolare $\left( \exists c, 0 \leq c \leq 1 : \forall n \rightarrow aw\left( \frac{n}{b} \right) \leq cw(n) \right) $
    vale \[ \forall l \geq 1 \quad a^l w\left( \frac{n}{b^l} \right) \leq c^l w(n) \]
    \label{teo:regestesa}
\end{lemma}
\begin{proof}
    Dimostriamo il lemma per induzione. La base, per $l=1$, coincide con l'ipotesi di regolarità. Ipotizzando il lemma sia valido per $l'<l$, dimostriamo che vale per $l$.
    \begin{align*}
        a^l w\left( \frac{n}{b^l} \right)
        &= a^{l-1} a w \left( \frac{n/b^{l-1}}{b} \right)
        & \text{per la regolarità } a w \left( \frac{n/b^{l-1}}{b} \right) \leq cw \left( \frac{n}{b^{l-1}} \right)&\\
        & \leq a^{l-1} cw \left( \frac{n}{b^{l-1}} \right)
        & \text{per l'HP induttiva } a^{l-1} \left( \frac{n}{b^{l-1}} \right) \leq c^{l-1}w(n)&\\
        & \leq cc^{l-1}w(n) \\
        & = c^lw\left( n \right)
    \end{align*}
\end{proof}
Dimostriamo ora il caso $3$, di cui riportiamo l'enunciato:
    \begin{align*}
        a) \quad &\exists \varepsilon > 0 : w(n) = \Omega(n^{\log_b a+\varepsilon}) \\
        b) \quad &\exists c, 0 \leq c \leq 1 : \forall n \rightarrow aw\left( \frac{n}{b} \right) \leq cw(n) \\
        \Rightarrow \quad & T(n) = \Theta \left( w(n) \right)
    \end{align*}
\begin{proof}
\begin{align*}
    \sum_{l=0}^{\log_b n -1} a^l \cdot w \left( \frac{n}{b^l} \right)
    & \leq \sum_{l=0}^{\log_b n -1} c^l \cdot w(n) \\
    & \leq w(n) \sum_{l=0}^{+\infty} c^l & c<1 &\\
    & = \frac{1}{1-c}w(n)
    \intertext{quindi la sommatoria è $O-grande$ di $w(n)$}
    \sum_{l=0}^{\log_b n -1} a^l \cdot w \left( \frac{n}{b^l} \right)
    & = O \left (w(n) \right) \\
    \intertext{combinando questo risultato con \ref{eq:masterfoglie} si ottiene}
    T(n) & = \Theta \left( n^{\log_b a} \right) + O\left( w(n) \right)
    \intertext{per le leggi sui vincoli asintotici, il primo contributo è trascurabile}
    T(n) & = O\left( w(n) \right)
    \intertext{per $n>1$ osserviamo}
    T(n) &= aT\left( \frac{n}{b} \right) w(n) \geq w(n)
    & \text{$a$ e $T$ positive} &\\
    T(n) & = \Omega \left( w(n) \right) 
    \intertext{da cui ricaviamo}
    T(n) & = \Theta \left( w(n) \right)
\end{align*}
\end{proof}

\subsection{autocomplete}
Una bella scatola:
\begin{equation}
    \boxed{x^2+y^2 = z^2}
\end{equation}

Numeri nei casi
\begin{numcases}{T(n)=}
    2^3 \label{escaso1} \\
    2^4 \label{escaso2} 
\end{numcases}

Sotto numeri
\begin{subnumcases}{T(n)=}
    2^3 \label{escaso3} \\
    2^4 
\end{subnumcases}

\begin{itemize}[noitemsep,topsep=0pt,parsep=0pt,partopsep=0pt]
    \item qualcosa
    \item[+] qualcosa
    \item[*] qualcosa
    \item[--] qualcosa
\end{itemize}
àg
èg
ìg
òg
ùg
perché

\subsubsection{Caso 2}
% Dimostrazione, pag 21
\begin{proof}
    \begin{align*}
        w(n) = \Theta(n^{\log_b a})
        & \Rightarrow
        T(n) = \Theta \left( n^{\log_b a} \log n \right) 
        \intertext{ossia per ipotesi}
        \exists c_1, c_2 : \forall n \geq 1 & \rightarrow c_1 n^{\log_b a} \leq w(n) \leq n^{\log_b a}
        \intertext{usiamo la seconda disequazione per maggiorare $w$}
        \sum_{l=0}^{\log_b n -1} a^l \cdot w \left( \frac{n}{b^l} \right)
        & \leq c_2 \sum_{l=0}^{\log_b n -1} a^l \left( \frac{n}{b^l} \right)^{\log_b a} \\
    % & = c_2 \sum_{l=0}^{\log_b n -1} \frac{\cancel{a^l}}{\cancel{a^l}} n^{log_b a}
        & = c_2 \sum_{l=0}^{\log_b n -1} \frac{a^l n^{\log_b a}}{a^l} 
        & b^{l^{\log_b a}} = a^l &\\
        & = c_2 n^{\log_b a} \sum_{l=0}^{\log_b n -1} 1 \\
        & = c_2 n^{\log_b a} \log_b n \\
        \intertext{quindi la sommatoria è $O-grande$ di $n^{\log_b a} \log n$}
        \sum_{l=0}^{\log_b n -1} a^l \cdot w \left( \frac{n}{b^l} \right)
        & = O \left( n^{\log_b a} \log n \right) \\
        \intertext{in modo analogo si minora con $c_1$}
        \sum_{l=0}^{\log_b n -1} a^l \cdot w \left( \frac{n}{b^l} \right)
        & \geq c_1 \sum_{l=0}^{\log_b n -1} a^l \left( \frac{n}{b^l} \right)^{\log_b a} = c_1 n^{\log_b a} \log_b n \\
        \intertext{quindi la sommatoria è $\Omega-grande$ di $n^{\log_b a} \log n$}
        \sum_{l=0}^{\log_b n -1} a^l \cdot w \left( \frac{n}{b^l} \right)
        & = \Omega \left( n^{\log_b a} \log n \right) \\
        \intertext{combinando i due risultati}
        \sum_{l=0}^{\log_b n -1} a^l \cdot w \left( \frac{n}{b^l} \right)
        & = \Theta \left( n^{\log_b a} \log n \right) \\
        \intertext{combinando questo risultato con \ref{eq:masterfoglie} si ottiene}
        T(n) & = \Theta \left( n^{\log_b a} \right) + \Theta \left( n^{\log_b a} \log n \right) \\
        \intertext{per le leggi sui vincoli asintotici, il primo contributo è trascurabile}
        T(n) & = \Theta \left( n^{\log_b a} \log n \right) \\
    \end{align*}
\end{proof}
In questo caso la maggior parte del lavoro è compiuta all'interno dell'albero, ogni livello compie lo stesso lavoro cumulativo, e ci sono $\log n$ livelli. L'esempio principale del caso 2 è il \textit{Merge-Sort}, descritto da $T(n)=2T(n/2)+n$, dove ogni livello compie $n^{\log_2 2}=n$ operazioni e la complessità è $\Theta (n \log n)$.


\chapter{Operazioni fra matrici}
\section{Introduzione - Notazione}
% notazione, pag 25.5
Saranno principalmente considerate matrici quadrate $\mathbb{R}^{n \times n}$, con taglia associata $n$, il numero di righe, o $N=n^2$, il numero di elementi.\\
Verranno considerate operazioni costose solo somme, sottrazioni, prodotti e divisioni tra scalari reali. A volte verrà posta particolare attenzione a prodotti e divisioni, che sono più computazionalmente intensi.\\
Una matrice $A$ è composta da elementi $a_{i,j}$ e si indica
\[ A = \left( a_{i,j} \right)_{1 \leq i,j \leq n} \]

\section{Somma e sottrazione}
% sum sub, pag 25.8, 26
La somma (e, in tutta la sezione, in modo analogo la sottrazione) è una delle operazioni definita fra le matrici.
\begin{definition}
    La somma fra matrici è definita come
    \[
    C = A + B \rightarrow
    c_{i,j} = a_{i,j} + b_{i,j}
    \]
    \label{def:somma}
\end{definition}
\begin{algorithm}[H]
\caption{Somma tra matrici}\label{alg:sum}
\begin{algorithmic}[1]
\Procedure{SUM}{$A,B$}
    \State $n \gets A.rows$
        \For{$i \gets 1 $ to $ n $ }
            \For{$j \gets 1 $ to $ n $ }
                \State $c_{i,j} \gets a_{i,j} + b_{i,j} $
            \EndFor
        \EndFor
    \State return $C$
\EndProcedure
\end{algorithmic}
\end{algorithm}
La correttezza dell'algoritmo discende in maniera diretta dalla definizione.

La complessità dell'algoritmo è $T(n) = n^2$

\section{Moltiplicazione di matrici}
\subsection{Definizione}
% definizione, pag 26.4
La moltiplicazione tra matrici è definita secondo il prodotto righe per colonne.
\begin{definition}
    Il prodotto righe per colonne è definito come
    \[
    C = A \times B \rightarrow
    c_{i,j} = \sum_{k=1}^{n}a_{i,k} b_{k,j}
    \]
    \label{def:prodrc}
\end{definition}
\begin{algorithm}[H]
\caption{Prodotto righe per colonne secondo la definizione}\label{alg:mulrcdef}
\begin{algorithmic}[1]
\Procedure{MUL}{$A,B$}
    \State $n \gets A.rows$
        \For{$i \gets 1 $ to $ n $ }
            \For{$j \gets 1 $ to $ n $ }
                \State $c_{i,j} \gets a_{i,j} \cdot b_{i,j} $
                \For{$k \gets 2 $ to $ n $ }
                \State $c_{i,j} \gets c_{i,j} + a_{i,j} \cdot b_{i,j} $
                \EndFor
            \EndFor
        \EndFor
    \State return $C$
\EndProcedure
\end{algorithmic}
\end{algorithm}
La correttezza dell'algoritmo discende in maniera diretta dalla definizione.

La complessità dell'algoritmo è $T(n) = \Theta (n^3)$, infatti trasformando i cicli $for$ in sommatorie, si ottiene la formula
\[
    T(n) = \sum_{i=1}^{n} \sum_{j=1}^{n} \left( 1 + \sum_{k=2}^{n} 2 \right) = 
    = \sum_{i=1}^{n} \sum_{j=1}^{n} \left( 2n-2+1 \right) 
    = \left( 2n-1 \right) \sum_{i=1}^{n} \sum_{j=1}^{n} 1 
    = (2n-1)n^2 = 2n^3 -n^2
\]

\subsection{Implementazione ricorsiva}
% moltiplicazione ricorsiva, pag 27.3, 28
Nel corso del capitolo ci limiteremo a studiare matrici di taglia $n=2^k$, semplificando notevolmente l'analisi asintotica. Questo non comporta una perdita di generalità, infatti $\forall m \neq 2^k$ è sufficiente scegliere $n=2^{\left\lceil \log_2 m \right\rceil} < 2m$ e si può procedere con gli algoritmi che saranno presentati, avendo al più quadruplicato il numero di elementi.

TODO disegni delle matrici pag 27.5

Il prodotto righe per colonne ha la proprietà frattale, ovvero lo si può risolvere dal prodotto righe per colonne dei suoi blocchi.
% TODO l'italiano quà è brutto forte
Infatti dividendo le matrici in blocchi di dimensione $\frac{n}{2} \times \frac{n}{2}$ vale
\[
    1 \leq r \leq 2,\, 1 \leq s \leq 2 \rightarrow C_{r,s} = A_{r,1} B_{1,s} + A_{r,2} B_{2,s}
\]

TODO divisione in blocchi di C, pag 27.6

TODO altri disegni pag 27.9

La proprietà frattale descritta, che permette di ottenere il prodotto righe per colonne di due matrici $n \times n$ tramite prodotti righe per colonne di matrici $\frac{n}{2} \times \frac{n}{2}$ che poi vengono combinati è una proprietà di sottostruttura. Il caso di base è semplicemente il prodotto di due scalari. Quando $n>1$ si calcolano ricorsivamente tutti i prodotti delle sottomatrici.

\begin{algorithm}[H]
\caption{Prodotto righe per colonne ricorsivo}\label{alg:mulrcric}
\begin{algorithmic}[1]
\Procedure{RMUL}{$A, B$}
    \State $n \gets A.rows$
    \If{$n = 1$}                             \Comment{BASE}
        \State return $\left( a_{11} \cdot b_{11} \right)$
    \EndIf
    \State *divido $A, B$ in blocchi $A_{11}, \cdots, B_{22}$*    \Comment{DIVIDE}
    \For{$r \gets 1 $ to $ 2 $ }                                    \Comment{RECURSE e CONQUER sono simultanei}
        \For{$s \gets 1 $ to $ 2 $ }
        \State $C_{r,s} \gets 
        \Call{SUM}{}
        % \Call{SUM}{ % boh si offende un sacco
        \left(
            \Call{RMUL}{A_{r,1}, B_{1,s}}, 
            \Call{RMUL}{A_{r,2}, B_{2,s}} 
        \right)
        $
        % } $
        \EndFor
    \EndFor
    \State return $C$
    \EndProcedure
\end{algorithmic}
\end{algorithm}
Osservando che il \textit{Divide} non comporta alcun costo, il \textit{Recurse} comporta $2$ chiamate ricorsive per ciascuna delle quattro iterazioni dei cicli $for$ e il \textit{Conquer} necessita di $4$ somme su matrici di dimensione $\frac{n}{2} \times \frac{n}{2}$, si può scrivere l'equazione di ricorrenza
\begin{equation}
    T(n) = 
    \begin{cases}
        1 & n=1 \\
        8 T \left( \frac{n}{2} \right) + 4 \left( \frac{n}{2} \right)^2 & n>1
    \end{cases}
    \label{eq:rcric}
\end{equation}
con parametri
\begin{align*}
    a&=8 & b&=2 & w(n)&=n^2 & n^{log_2 8}&=n^3
\end{align*}
per cui siamo nel caso uno del \textit{Master Theorem}, e risulta
\begin{equation*}
    T(n) = \Theta\left( n^{log_2 8} \right) = \Theta\left( n^{3} \right)  
\end{equation*}
TODO applica per ESERCIZIO la formula generale per ricavare le costanti $2n^3-n^2$

\subsection{Algoritmo di \textit{Strassen}}
% moltiplicazione Strassen, pag 29
Il \textit{Master Theorem}, oltre ad essere strumento di analisi, può essere usato per la progettazione di algoritmi. Si può notare dall'equazione \ref{eq:rcric} che il costo delle chiamate ricorsive è molto maggiore rispetto al costo di ricombinazione. Cercando un algoritmo che riduca il primo, eventualmente a scapito del secondo, si può provare ad abbassare la complessità dell'algoritmo.

L'algoritmo di \textit{Strassen} segue proprio questa strada, mostrando una maniera per ottenere il prodotto di due matrici $n \times n$ usando solo $7$ prodotti di matrici $\frac{n}{2} \times \frac{n}{2}$ e un numero costante($18$) di somme e sottrazioni tra matrici $\frac{n}{2} \times \frac{n}{2}$.

\begin{algorithm}[H]
    \caption{Algoritmo di \textit{Strassen}}\label{alg:strassen}
\begin{algorithmic}[1]
% \Procedure{$D_S$}{$A, B$}
\Procedure{D\textsubscript{S}}{$A, B$}
    \State $A_1 \gets A_{11}$; $B_1 \gets \Call{SUB}{B_{12}, B_{22} }$
    \State $A_2 \gets \Call{SUM}{A_{11}, A_{12}}$; $B_2 \gets B_{22}$
    \State $A_3 \gets \Call{SUM}{A_{21}, A_{22}}$; $B_3 \gets B_{11}$
    \State $A_4 \gets A_{22}$; $B_4 \gets \Call{SUB}{B_{21}, B_{11} }$
    \State $A_5 \gets \Call{SUM}{A_{11}, A_{22}}$; $B_5 \gets \Call{SUM}{B_{11}, B_{22} }$
    \State $A_6 \gets \Call{SUM}{A_{12}, A_{22}}$; $B_6 \gets \Call{SUM}{B_{21}, B_{22} }$
    \State $A_7 \gets \Call{SUB}{A_{11}, A_{21}}$; $B_7 \gets \Call{SUB}{B_{11}, B_{12} }$
    \State return $A_1, \cdots, A_7, B_1, \cdots, B_7$
\EndProcedure
\Procedure{R\textsubscript{S}}{$A_1, \cdots, A_7, B_1, \cdots, B_7$}
    \For{$i \gets 1 $ to $ 7 $ }
    \State $P_i \gets \Call{SMUL}{A_{i}, B_{i} }$
    \EndFor
    \State return $P_1, \cdots, P_7$
\EndProcedure
\Procedure{C\textsubscript{S}}{$P_1, \cdots, P_7$}
    \State $C_{11} \gets \Call{SUB}{} \left( \Call{SUM}{P_4, P_5}, \Call{SUB}{P_2, P_6} \right) $
    \State $C_{12} \gets \Call{SUM}{P_1, P_2}$
    \State $C_{21} \gets \Call{SUM}{P_3, P_4}$
    \State $C_{22} \gets \Call{SUB}{} \left( \Call{SUM}{P_1, P_5}, \Call{SUB}{P_3, P_7} \right) $
    \State return $C_1, \cdots, C_4$
\EndProcedure
\Procedure{SMUL}{$A, B$}
    \State $n \gets A.rows$
    \If{$n = 1$}                             \Comment{BASE}
        \State return $\left( a_{11} \cdot b_{11} \right)$
    \EndIf
    \State $\left( A_1, \cdots, A_7, B_1, \cdots, B_7 \right) \gets \Call{D\textsubscript{S}}{A, B}$
    \Comment{DIVIDE}
    \State $\left( P_1, \cdots, P_7 \right) \gets \Call{R\textsubscript{S}}{A_1, \cdots, A_7, B_1, \cdots, B_7}$
    \Comment{RECURSE}
    \State $C  \gets \Call{C\textsubscript{S}}{P_1, \cdots, P_7}$
    \Comment{CONQUER}
    \State return $C$
    \EndProcedure
\end{algorithmic}
\end{algorithm}

La correttezza dell'algoritmo si prova mostrando la correttezza dei singoli blocchi di $C$, per esempio per $C_{12}$
\begin{equation*}
    C_{12} = P_1+P_2 = A_{11} \left( B_{12}-B_{22} \right) + \left( A_{11}+A_{12} \right) B_{22} =
    A_{11}B_{12} - \cancel{A_{11}B_{22}} + \cancel{A_{11}B_{22}} + A_{12}B_{22}
\end{equation*}

L'equazione di ricorrenza associata all'algoritmo è
\begin{equation}
    T(n) = 
    \begin{cases}
        1 & n=1 \\
        7 T \left( \frac{n}{2} \right) + 18 \left( \frac{n}{2} \right)^2 & n>1
    \end{cases}
    \label{eq:ricstrassen}
\end{equation}
per cui la funzione di soglia è diventata $n^{log_2 7} \approx n^{2.80}$, si è ancora nel caso uno e vale
\begin{equation*}
    T_S(n) = \Theta \left( n^{log_2 7} \right)
\end{equation*}
TODO verifica per ESERCIZIO il valore alle costanti, deve venire $T_S(n)=7n^{log_2 7}-6n^2$

Per quanto l'esponente sia diminuito rispetto all'implementazione ricorsiva base, è aumentato il coefficiente da $2$ a $7$. Per valori piccoli di $n$ quindi, la complessità dell'algoritmo di \textit{Strassen} è maggiore, e solo per $n>1024$ diventa conveniente.

\subsection{Algoritmo ibrido}
Si può notare come l'algoritmo di Strassen sia asintoticamente veloce, ma con costanti elevate, mentre l'algoritmo diretto sia asintoticamente più lento, ma abbia costanti minori. Utilizzando la tecnica dell'ibridizzazione, si possono combinare i due algoritmi, mantenendo l'andamento asintotico del primo e diminuendo le costanti.

Il primo approccio, immediato, si limita a selezionare uno dei due algoritmi in base alla taglia della matrice


\begin{algorithm}[H]
    \caption{Algoritmo ibrido naive}\label{alg:mulrcnaive}
\begin{algorithmic}[1]
\Procedure{Naive\_Hybrid\_MUL}{$A, B$}
    \State $n \gets A.rows$
    \If{$n < 1024 $}
        \State return $\Call{MUL}{A,B}$
        \Else
        \State return $\Call{SMUL}{A,B}$
    \EndIf
    \EndProcedure
\end{algorithmic}
\end{algorithm}

L'equazione di ricorrenza associata all'algoritmo è
\begin{equation*}
    T_{NH}(n) = 
    \begin{cases}
        2n^3 - n^2 & n<1024 \\
        7 n^{log_2 7} - 6n^2& n \geq 1024
    \end{cases}
    \label{eq:ricnaivehybrid}
\end{equation*}
e si può notare non è che sia cambiata molto, in particolare per $n \geq 1024 $ è rimasta identica.

Innestiamo quindi in modo più efficace l'algoritmo veloce, modificando il caso base di \textit{Strassen}, cambiando $n_0$ e bloccando prima la ricorsione.

TODO albero pag 31.7

\begin{algorithm}[H]
    \caption{Algoritmo ibrido}\label{alg:mulrcibrido}
\begin{algorithmic}[1]
\Procedure{Hybrid\_SMUL}{$A, B$}
    \State $n \gets A.rows$
    \If{$n \leq n_0 $}
        \State return $\Call{MUL}{A,B}$
        \Comment{BASE}
    \EndIf
    \State $\left( A_1, \cdots, A_7, B_1, \cdots, B_7 \right) \gets \Call{D\textsubscript{S}}{A, B}$
    \Comment{DIVIDE}
    \State $\left( P_1, \cdots, P_7 \right) \gets \Call{R\textsubscript{HS}}{A_1, \cdots, A_7, B_1, \cdots, B_7}$
    \Comment{RECURSE}
    \State $C  \gets \Call{C\textsubscript{S}}{P_1, \cdots, P_7}$
    \Comment{CONQUER}
    \State return $C$
    \EndProcedure
\end{algorithmic}
\end{algorithm}

Nota:
\begin{itemize}[noitemsep,topsep=0pt,parsep=0pt,partopsep=0pt]
    \item[--] nell'algoritmo \ref{alg:mulrcibrido} la chiamata ricorsiva è stata modificata, in modo da chiamare l'algoritmo ibrido
    \item[--] l'analisi di correttezza va modificata, la base è valida per la correttezza di MUL, il resto dell'induzione è analogo
\end{itemize}

L'equazione di ricorrenza associata all'algoritmo è
\begin{equation}
    T(n, n_0) = 
    \begin{cases}
        2n^3 - n^2 & n<n_0 \\
        7 T \left( \frac{n}{2}, n_0 \right) + \frac{9}{2}n^2 & n \geq n_0
    \end{cases}
    \label{eq:richybrid}
\end{equation}

Risolvendo l'equazione ricorsiva, si ottiene una complessità in funzione di $n_0$ e si può minimizzare la funzione

Utilizzando la formula generale \ref{eq:formachiusaric}, con $n=2^k$ e scegliendo $n_0 = 2^k$, si ricava per il caso base
\begin{align*}
    T_0 &= 2 n_0^3 - n_0^2
    \intertext{e per l'ultimo livello prima dei nodi foglia $f^*\left( n, n_0 \right)$}
    l \text{ generale } &\rightarrow \text{ taglia } \frac{n}{2^l} \text{ che pongo uguale a } n_0 \\
    \frac{n}{2^l} &= n_0 \rightarrow 2^l = \frac{n}{n_0} \rightarrow l = log_2 \left( \frac{n}{n_0} \right)
    \intertext{come livello delle foglie, per cui}
    f^*\left( n, n_0 \right) &= log_2 \left( \frac{n}{n_0} \right) -1
    \intertext{che si comporta come previsto, al crescere di $n_0$ il numero di livelli diminuisce; ricordando inoltre}
    S(n) = 7 \quad f(n) &= \frac{n}{2} \quad w(n) = \frac{9}{2} n^2
\end{align*}
risulta
\begin{align*}
    T(n, n_0) &=
    \sum_{l=0}^{ log_2 \left( \frac{n}{n_0} \right) -1} 7^l \frac{9}{2} \left( \frac{n}{2^l} \right)^2
    + 7^{log_2 \frac{n}{n_0}} \left( s n_0^3 - n_0^2 \right) \\
    &= \ldots \text{TODO ESERCIZIO} \\
    &= \left( \frac{2n_0+5}{n_0^{log_2 7}-2} \right)n^{log_2 7} - 6n^2
\end{align*}
Per cui la relazione con $n_0$ risulta solo nella costante moltiplicativa, e possiamo minimizzare $g(n_0)$, vincolato a $n_0 = 2^k$, in generale la derivata che si deve studiare è
\begin{equation}
    \frac{\partial}{\partial n_0} T(n, n_0)
\end{equation}

% \begin{align*}
\begin{gather*}
    \left( \frac{d}{d n_0} g(n_0) \right) n^{log_2 7} \quad \text{di cui studio il segno} \\
    \text{sign} \left( \frac{d}{d n_0} g(n_0) \right) 
    = \text{sign} \left( 2 n_0 - \left( 2n_0+5 \right)\left( log_2 7-2 \right) \right) \\
    % TODO parentesi più grandi
    \frac{d}{d n_0} g(n_0) \geq 0  \rightarrow n_0 \geq \frac{5\left( log_2 7 - 2 \right)}{2\left( 3-log_2 7 \right)} \approx 10,48
    \intertext{per cui il massimo vincolato a $n_0=2^k$ vale}
    n_0' = 8 \rightarrow g(8) \approx 3,92 \\
    n_0'' = 16 \rightarrow g(16) \approx 3,94 \\
% \end{align*}
\end{gather*}

L'equazione di ricorrenza associata all'algoritmo è
\begin{equation}
    T(n, 8) 
    \begin{cases}
        = 2n^3 - n^2 & n \leq 8 \\
        \approx 3,92 n^{log_2 7} + 6n^2 & n > 8
    \end{cases}
    \label{eq:richybrid}
\end{equation}


\subsection{autocomplete}
Una bella scatola:
\begin{equation}
    \boxed{x^2+y^2 = z^2}
\end{equation}

Numeri nei casi
\begin{numcases}{T(n)=}
    2^3 \label{escaso1} \\
    2^4 \label{escaso2} 
\end{numcases}

Sotto numeri
\begin{subnumcases}{T(n)=}
    2^3 \label{escaso3} \\
    2^4 
\end{subnumcases}

\begin{itemize}[noitemsep,topsep=0pt,parsep=0pt,partopsep=0pt]
    \item qualcosa
    \item[+] qualcosa
    \item[*] qualcosa
    \item[--] qualcosa
\end{itemize}
àg
èg
ìg
òg
ùg
perché



\chapter{Trasformata veloce di \textit{Fourier}}
\section{Rappresentazione di polinomi}
% polinomi definizione 33.7
In generale, avere diverse rappresentazioni di un oggetto permette di eseguire operazioni diverse su rappresentazioni diverse, dove risulta più comodo. Chiaramente la rappresentazione è legata a come si possono fare le operazioni, e sono inoltre necessarie operazioni efficienti di conversione.

Introduciamo quindi un nuovo dominio applicativo, i polinomi, su cui verrà costruito l'algoritmo per la trasformata veloce di \textit{Fourier}.

\begin{definition}[Polinomio]
    Un polinomio è una funzione $p: \mathbb{C} \rightarrow \mathbb{C}$ definita su un'indeterminata e un insieme di coefficienti, come somma di monomi.
    \begin{equation*}
        p(x) = a_0 + a_1 x + a_2 x^2 + \dots + a_{n-1} x^{n-1} = \sum_{j=0}^{n-1}a_j x^j
    \end{equation*}
    \label{def:polinomio}
\end{definition}

\begin{definition}[Grado di un polinomio]
    Il grado di un polinomio è definito come l'indice massimo del coefficiente non nullo
    \begin{equation*}
        deg(p(x)) = \max \left\{ i: a_i \neq 0 \right\}
    \end{equation*}
    \label{def:poligrado}
\end{definition}

% grado limitato da n, pag 33.9
\begin{definition}[Polinomio di grado limitato da $n$]
    Un polinomio si dice di grado limitato da $n$ se il suo massimo grado può essere $n-1$
    \label{def:polilimitato}
\end{definition}

\subsection{Rappresentazione per coefficienti}
% rapp coeff 33.8
Un polinomio può essere rappresentato a $n$ coefficienti
\begin{equation*}
    p(x) \equiv \vec{a} \in \mathbb{C}^n
\end{equation*}
La rappresentazione può essere facilmente estesa con un'operazione di padding
\begin{equation*}
    p(x) \equiv \left( \vec{a}, 0_m \right) \in \mathbb{C}^{n+m}
\end{equation*}

\subsection{Rappresentazione per punti}
% forse non serve una subsubsection
\begin{theorem}[Teorema di interpolazione]
    Date $n$ coppie di punti $\left( x_i, y_i \right) \in \mathbb{C}^2 \text{ con }
    % x_i~\neq~x_j
    x_i \neq x_j
    % \forall~i~\neq~j
    \; \forall i \neq j
    \; \exists ! p(x)
    $
    di grado limitato da $n$ per cui
    $p(x_i) = y_i$
    detto polinomio interpolante.
    \label{teo:interpolazione}
\end{theorem}
C'è quindi una corrispondenza tra $n-uple$ di punti e un \emph{singolo} polinomio, quindi una $n-upla$ di punti è una rappresentazione di un polinomio di grado limitato da $n$
\begin{equation*}
    p(x) \equiv \left( \vec{x}, \vec{y}\right) \quad
    \vec{x}, \vec{y} \in \mathbb{C}^{n}, x_i \neq x_j \; \forall i \neq j
\end{equation*}
dove $\vec{x}$ si dice base della rappresentazione

Anche la rappresentazione per punti si può estendere
\begin{equation*}
    \left( \vec{x}, \vec{y}\right) \rightarrow
    \left( \vec{x}^E, \vec{y}^E\right) \quad
    \vec{x}^E, \vec{y}^E \in \mathbb{C}^{m}, x_i \neq x_j \; \forall i \neq j, \text{ con } m>n
\end{equation*}
Per estendere la rappresentazione occorre valutare il polinomio in $m-n$ punti aggiuntivi

\subsection{Conversione tra rappresentazioni}
Se si dispone della rappresentazione per coefficienti, è sufficiente valutare il polinomio in un certo numero di punti per ricavare la rappresentazione per punti, mentre se si dispone di una tabulazione occorre interpolare il polinomio. Eseguire questa conversione in maniera efficiente sarà argomento della sezione \ref{sez:conversione}

\section{Operazioni tra polinomi}

\subsection{Operazioni utilizzando la rappresentazione per coefficienti}
% somma sottrazione coefficienti 34.8
\subsubsection{Somma e sottrazione}
Siano $A(x) \equiv \vec{a}$ e $B(x) \equiv \vec{b}$, con $\vec{a}, \vec{b} \in \mathbb{C}^n$ e $C(x) \equiv \vec{c}$ omogeneo, con $\vec{c} \in \mathbb{C}^n$
\begin{align*}
    C(x) &= A(x) + B(x)
    = \sum_{j=0}^{n-1} a_j x^j + \sum_{j=0}^{n-1} b_j x^j 
    = \sum_{j=0}^{n-1} \left( a_j + b_j \right) x^j 
    = \sum_{j=0}^{n-1} c_j x^j 
\end{align*}
Per cui $C(x)$ è rappresentato dalla somma vettoriale delle rappresentazioni: $\vec{c} = \vec{a} + \vec{b}$

% prodotto coefficienti 35
\subsubsection{Prodotto}
Siano $A(x) \equiv \vec{a}$ e $B(x) \equiv \vec{b}$, con $\vec{a}, \vec{b} \in \mathbb{C}^n$, in questo caso $C(x)$ è un polinomio di grado limitato da $2n-1$, infatti $(n-1)+(n-1)+1$ (il limite di grado è di $1$ superiore al grado massimo) quindi $C(x) \equiv \vec{c}$, con $\vec{c} \in \mathbb{C}^{2n-1}$
\begin{align*}
    C(x) &= A(x) \cdot B(x)
    = \left( \sum_{j=0}^{n-1} a_j x^j \right) \cdot \left( \sum_{j=0}^{n-1} b_j x^j \right) 
\end{align*}
Da cui, per $ 0 \leq j \leq 2n-2$
\begin{align*}
    c_j &= \sum_{ \substack{k,h \\ k+h=j \\ 0 \leq k,h \leq n-1} } a_k b_h
    \intertext{che semplifichiamo notando che $h=j-k$}
    c_j &= \sum_{k=0}^{n-1} a_k b_{j-k}
    \intertext{ma questa sommatoria è valida solo sfruttando una convenzione per cui se $j-k$ è $<0$ o $\geq n$, il coefficiente $b_{j-k}$ assume valore $0$}
\end{align*}
Si possono cercare vincoli più stretti su $k$
\begin{equation*}
    \left\{ 
        \begin{array}[h]{l}
            0 \leq k \leq n-1 \\
            0 \leq j-k \leq n-1
        \end{array}
    \right.
    \rightarrow
    \left\{ 
        \begin{array}[h]{l}
            0 \leq k \leq n-1 \\
            j-n+1 \leq k \leq j
        \end{array}
    \right.
\end{equation*}
ossia
\begin{equation*}
    \max\left\{ 0, j-n+1 \right\} \leq k \leq \min \left\{ n-1, j \right\}
\end{equation*}
\begin{definition}[Convoluzione lineare]
    La convoluzione lineare è un operatore vettoriale definito come
    \begin{equation*}
        \vec{c}=\vec{a} * \vec{b} \rightarrow 0\leq j \leq 2n : c_j = \sum_{k = \max\left\{ 0, j-n+1 \right\}}^{\min \left\{ n-1, j \right\}} a_k b_{j-k}
        % ho dubbi su quel \leq 2n, cioè per j=2n k va da n+1 a n-1 quindi la somma è vuota, ed è corretto, ma comunque si è allungato senza dire niente a nessuno e mi da fastidio
    \end{equation*}
    \label{def:convlin}
\end{definition}
TODO ESERCIZIO: supponi $\vec{b} \in \mathbb{C}^m$, ricava i vincoli, dovrebbe venirti $\max \left\{ 0, j-m+1 \right\}$

% magia delle più nere questo calcolo di complessità
La complessità per calcolare direttamente la convoluzione risulta, contando il numero di operazioni: $n^2$ prodotti $+n^2$ somme $-(2n-1)$ somme che risparmi, per cui $T(n)=2n^2-2n+1$ o alternativamente $T(n)=n^2+(n-1)^2$, in cui si mette in evidenza il numero di prodotti $\left[ n^2 \right]$ e di somme $\left[ (n-1)^2 \right]$.

% algoritmo convoluzione lineare 36
L'algoritmo naive per implementare la convoluzione lineare risulta
\begin{algorithm}[H]
\caption{Convoluzione lineare}\label{alg:convlinnaive}
\begin{algorithmic}[1]
    \Procedure{DEF\_LIN\_CONV}{$\vec{a}, \vec{b}$}
        \State $n \gets \vec{a}.length$
        \For{$j \gets 0 $ to $ 2n-2 $ }
            \State $c_j \gets 0$
            \For{$k \gets \max\left\{ 0, j-n+1 \right\} $ to $ \min \left\{ n-1, j \right\} $ }
                \State $c_j \gets c_j + a_k b_{j-k}$
            \EndFor
        \EndFor
        \State return $ \vec{c}$
    \EndProcedure
\end{algorithmic}
\end{algorithm}
In quest'implementazione una somma è stata sprecata inizializzando $c_j=0$, si potrebbe inizializzare a $c_j=a_{\max\left\{ 0, j-n+1 \right\}}b_{j-\max\left\{ 0, j-n+1 \right\}}$

La complessità dell'algoritmo risulta $T_{DLC} = \Theta \left( n^2 \right)$
\subsection{Operazioni utilizzando la rappresentazione per punti}

% somma per punti 36.3
\subsubsection{Somma e sottrazione}
Siano $A(x) \equiv \left( \vec{x}, \vec{y}_A \right)$ e $B(x) \equiv \left( \vec{x}, \vec{y}_B \right)$ rappresentazioni omogenee, sulla stessa base.
\begin{align*}
    C(x) &= A(x) + B(x)
    \rightarrow C(x_i) = A(x_i)+B(x_i)
    \rightarrow y_{C_{i}}= y_{A_{i}}+ y_{B_{i}}
\end{align*}
Per cui $C(x)$ è rappresentato da $\left( \vec{x}, \vec{y}_A+\vec{y}_B \right)$

% prodotto per punti 36.5
\subsubsection{Prodotto}
La relazione $C(x) = A(x) \cdot B(x) $ è valida per ogni $x$, quindi anche per tutti i punti $x_i$ della base: $C(x_i) = A(x_i) \cdot B(x_i) $.
Tuttavia \emph{non} è sufficiente rappresentare $C$ come
$\left( \vec{x}, \vec{y}_A \odot \vec{y}_B \right)$,
infatti sono necessari $2n-1$ punti per definire il polinomio.

Considerando le rappresentazioni estese $A(x) \equiv \left( \vec{x}^E, \vec{y}_A^E \right)$ e $B(x) \equiv \left( \vec{x}^E, \vec{y}_B^E \right)$, con $\vec{x}^E, \vec{y}_A^E,  \vec{y}_B^E \in \mathbb{C}^{2n-1}$ si ottiene la rappresentazione lecita 
$C(x) \equiv \left( \vec{x}^E, \vec{y}^E_A \odot \vec{y}^E_B \right)$.

Per eseguire il prodotto sono quindi necessari solamente $2n-1$ prodotti, posto di avere a disposizione la rappresentazione estesa, portando a una complessità di $T(n) = \Theta \left( n \right)$

\section{Conversione tra rappresentazioni}\label{sez:conversione}
% convertire in modo lento e poi veloce pag 36.9
% grafo commutativo secsi 36.9
Se si vuole interpolare o valutare un polinomio in maniera generale, non è possibile raggiungere una complessità minore di $\Theta \left( n^2 \right)$. È necessario utilizzare basi particolari per poter accelerare l'algoritmo.

\subsection{Valutazione}
\subsubsection{Valutazione naive}
% valutazione naive 37.2
Ricordando che
    \begin{equation*}
        p(x) = \sum_{j=0}^{n-1}a_j x^j
    \end{equation*}
si ottiene direttamente un algoritmo dalla definizione
\begin{algorithm}[H]
\caption{Valutazione naive}\label{alg:valnaive}
\begin{algorithmic}[1]
    \Procedure{DEF\_VAL}{$\vec{a}, \bar{x}$}
        \State $n \gets \vec{a}.length$
        \State $y \gets a_0$
        \State $pow \gets 1$
        \For{$j \gets 1 $ to $ n-1 $ }
            \State $pow \gets pow \cdot \bar{x}$
            \State $y \gets y + a_j \cdot pow$
        \EndFor
        \State return $y$
    \EndProcedure
\end{algorithmic}
\end{algorithm}
Vengono compiute $n-1$ iterazioni con una somma e due prodotti per ciclo.

% valutazione con horner 37.5
\subsubsection{Valutazione con \textit{Horner}}
Si può riscrivere un polinomio seguendo la regola di \textit{Horner} come 
    \begin{equation*}
        % p(x) = a_0 + x \left( a_1 + x \left( a_2 + \cdots + x 
        p(x) = a_0 + x ( a_1 + x ( a_2 + \cdots + x 
        % \left( a_{n-2} + x a_{n-1} \right) \right. \cdots \right)
        ( a_{n-2} + x a_{n-1} 
        \underbrace{ ) \cdots ) }_{\mathclap{ n-1 \text{ parentesi} } }
    \end{equation*}
si ottiene così l'algoritmo
\begin{algorithm}[H]
    \caption{Valutazione con \textit{Horner}}\label{alg:valhorner}
\begin{algorithmic}[1]
    \Procedure{HOR\_VAL}{$\vec{a}, \bar{x}$}
        \State $n \gets \vec{a}.length$
        \State $y \gets a_{n-1}$
        \For{$j \gets 2 $ to $ n $ }
            \State $y \gets a_{n-j} + \bar{x}y$
        \EndFor
        \State return $y$
    \EndProcedure
\end{algorithmic}
\end{algorithm}
Vengono compiute $n-1$ iterazioni con una somma e un prodotto per ciclo, portando la complessità a $T_H(n) = 2(n-1)$

% aggregato su n punti 37.9
Questo algoritmo deve essere applicato per ciascuno degli $n$ punti, per cui la complessità della valutazione risulta $\Theta \left( n^2 \right)$

\subsection{Interpolazione}
\subsubsection{Interpolazione con \textit{Gauss}}
% interpolazione gauss 38.2
La formula \ref{def:polinomio} vale $\forall x$ e in particolare $\forall x_i \in \vec{x}$ base della rappresentazione
\begin{equation*}
    p(x) = \sum_{j=0}^{n-1}a_j x^j
    \rightarrow y_i = p(x_i) = \sum_{j=0}^{n-1}a_j x_i^j
    \quad 0 \leq i \leq n-1
\end{equation*}
Si può considerare $\vec{a}$ come incognita di un sistema lineare di $n$ equazioni:
\begin{equation*}
    X \vec{a} = \vec{y}
\end{equation*}
dove
\begin{equation*}
    X = \left( 
        \begin{array}[h]{ccccccc}
            1 & x_0 & x_0^2 & \cdots & x_0^j & \cdots & x_0^{n-1} \\
            \vdots &&&&&& \vdots\\
            1 & x_i & x_i^2 & \cdots & x_i^j & \cdots & x_i^{n-1} \\
            \vdots &&&&&& \vdots\\
            1 & x_n & x_n^2 & \cdots & x_n^j & \cdots & x_n^{n-1} 
        \end{array}
    \right)
\end{equation*}
per cui $X$ è una matrice di \textit{Vandermonde}, quindi invertibile se e solo se $x_i \neq x_j \; \forall i\neq j$, ipotesi verificata per definizione di base.

Utilizzando il metodo dell'eliminazione di \textit{Gauss}, occorrono $n$ operazioni \textit{pivot} su $n^2$ elementi, portando ad una complessità di $\Theta \left( n^3 \right)$

\subsubsection{Interpolazione con \textit{Lagrange}}
% interpolazione lagrange 38.7
\textit{Lagrange} ha trovato una formula chiusa per il polinomio interpolante
\begin{equation*}
    p(x) = \sum_{k=0}^{n-1} \frac{\left(
            \displaystyle
            \prod_{\substack{j=0 \\ j \neq k}}^{n-1} \left( x - x_j \right)
    \right)}{\left( 
            \displaystyle
            \prod_{\substack{j=0 \\ j \neq k}}^{n-1} \left( x_k - x_j \right)
    \right)}
    = \sum_{k=0}^{n-1} \frac{y_k}{Q_k(x_k)} q_k(x)
\end{equation*}
avendo definito
\begin{equation*}
    Q_k(x) = \prod_{\substack{j=0 \\ j \neq k}}^{n-1} \left( x - x_j \right)
\end{equation*}
e si può dimostrare che la formula si valuta in $\Theta \left( n^2 \right)$ (sulle dispense online c'è un utile ESERCIZIO da leggere)

\subsection{Conversioni con la \textit{Discrete Fourier Transform}}
% conversioni più veloci 39
Senza perdita di generalità, si possono considerare basi particolari su cui valutare e interpolare un polinomio. In particolare si sceglie una famiglia di vettori di taglia $n$ legata alla radice principale dell'unità:
\begin{equation*}
    \vec{\Omega}_n = 
    \left(
        \omega_n^0=1 , \omega_n, \omega_n^2, \cdots, \omega_n^{n-1} 
    \right)
\end{equation*}
dove $\omega_n$ è la radice principale $n-esima$ di $1$ nel campo complesso.
Basandosi su questa base si può definire la trasformata veloce di \textit{Fourier}

\begin{definition}[\textit{Discrete Fourier Transform}]
    Dato un polinomio $A(x)$ e la sua rappresentazione per punti $A(x) \equiv \vec{a} \in \mathbb{C}^n$, si definisce la trasformata veloce di \textit{Fourier} come la valutazione di $A(x)$ sui punti di $\vec{\Omega}_n$
    \begin{equation*}
        \vec{y} = DFT_n \left( \vec{a} \right) \quad \text{ con } \quad y_i = \sum_{j=0}^{n-1} a_j \left( \omega_n^{i} \right)^{j}
    \end{equation*}
    per cui $\vec{y} \in \mathbb{C}^n$ e $A(x) \equiv \left( \vec{\Omega}_n , \vec{y} \right)$ è una rappresentazione per punti di $A(x)$
    \label{def:dft}
\end{definition}
% TODO piccole note a pag 40.1
Gli $n$ vincoli introdotti nella definizione sono $n$ valutazioni di $A(x)$ nei punti della base $\vec{\Omega}_n$ e possono essere raccolti in una matrice di \textit{Vandermonde} detta \emph{matrice di \emph{Fourier} di ordine $n$}
\begin{equation}
    V\left( \vec{\Omega}_n \right) = \left( 
        \begin{array}[h]{ccccccc}
            1 & \omega_n^0 = 1 & 1 & \cdots & 1 & \cdots & 1 \\
            \vdots &&&&&& \vdots\\
            1 & \omega_n & \omega_n^2 & \cdots & \omega_n^j & \cdots & \omega_n^{n-1} \\
            \vdots &&&&&& \vdots\\
            1 & \omega_n^{i} & \omega_{n}^{i2} & \cdots & \omega_{n}^{ij} & \cdots & \omega_n^{i\left( n-1 \right)} \\
            \vdots &&&&&& \vdots\\
            1 & \omega_n^{n-1} & \omega_{n}^{\left( n-1 \right) 2} & \cdots & \omega_{n}^{\left( n-1 \right) j} & \cdots & \omega_n^{\left( n-1 \right) \left( n-1 \right)} \\
        \end{array}
    \right) = F_n
    \label{eq:matricefourier}
\end{equation}
Si può riscrivere la $DFT$ come operatore lineare
\begin{equation*}
    DFT_n\left( \vec{a} \right) = F_n \vec{a} = \vec{y}
\end{equation*}
Ricordando che $\vec{\Omega}_n$ è una base composta da punti distinti e $F_n$ è una matrice di \textit{Vandermonde}, questa è anche invertibile e si può definire la trasformata inversa
\begin{equation*}
    DFT_n^{-1} \left( \vec{y} \right) = \vec{a} = F_n^{-1} \vec{y}
\end{equation*}
che risulta l'operazione di interpolazione.

Nota: la $DFT_n\left( \vec{a} \right) $ appena definita è un operatore algebrico lineare, una descrizione astratta del problema computazionale, mentre $FFT \left( \vec{a} \right)$ è l'algoritmo per calcolarla. Mostreremo che $T_{FFT} \left( n \right) = \Theta \left( n \log_n \right)$

\section{Proprietà della base $\vec{\Omega}_n$}
Prima di implementare l'algoritmo è ancora necessario individuare la proprietà di sottostruttura, che permetta di valutare un polinomio in funzione di valutazioni di polinomi di grado minore.

\subsection{Radici complesse}
Ogni numero complesso $z \in \mathbb{C}$ può essere rappresentato con la sua rappresentazione algebrica $z = a+ib$, ma è di maggiore utilità in questo caso la rappresentazione polare
\begin{equation*}
    z = \rho \left( \cos \theta + i \sin \theta \right) = \rho e ^{i \theta} \quad
    \text{ con } \quad \rho = \sqrt{a^2+b^2} \; \text{ e } \; 0 \leq \theta < 2\pi
\end{equation*}
Con questa rappresentazione il prodotto è molto semplice, ma in particolare lo è l'esponenziazione:
\begin{equation*}
    z^n = \rho^n e^{in\theta}
\end{equation*}
da cui si ricavano facilmente le $n$ radici $n-esime$ di un numero complesso $z \in \mathbb{C} \setminus \left\{ 0 \right\}$
\begin{equation*}
    z_k = \rho_k e^{i \theta_k}
    \quad \rightarrow \quad 
    z_k^n = z = \rho e^{i\theta} = \rho_k^n e ^{i \left( n \theta_k - 2 \pi k \right)}
\end{equation*}
da cui si ricavano due vincoli su $\rho_k $ e $ \theta_k$ con $0 \leq k \leq n-1$
\[
    \left\{
        \begin{array}[h]{l}
            \rho = \rho_k^n \\[4pt]
            \theta = n \theta_k - 2 \pi k
        \end{array}
    \right.
    \quad \rightarrow \quad 
    \left\{
        \begin{array}[h]{l}
            \rho_k = \sqrt[n]{\rho} \\[4pt]
            \theta_k = \frac{\theta + 2 \pi k}{n} = \frac{\theta}{n} + 2 \pi \frac{k}{n}
        \end{array}
    \right.
\]
Le radici sono quindi le rotazioni della radice principale intorno all'origine. Nel caso che ci interessa, per $z=1$ risultano $\rho=1$ e $\theta = 0$ da cui
\begin{equation*}
    z_k = \sqrt[n]{1}  \; e^{i \left( \frac{0}{n} + \frac{2 \pi k}{n} \right)} 
    = e ^{i \frac{2 \pi k }{n}} \quad k=0,\cdots,n-1
\end{equation*}
e risultano
\begin{equation*}
    k=1 \quad \rightarrow \quad \omega_n= e^{i \frac{2 \pi}{n}}
\end{equation*}

\subsection{Proprietà delle radici $n-esime$ dell'unità}

\subsubsection{Modulo dell'esponente}

Moltiplicare l'argomento per $k$ corrisponde a esponenziare, e vale
\begin{equation*}
    % \omega_n^j = \omega_n^{j \mod n}
    % \omega_n^j = \omega_n^{j \! \! \! \! \mod n}
    \omega_n^j = \omega_n^{j \bmod n}
\end{equation*}
infatti è come aggiungere o sottrarre multipli di $2\pi$. Questo vale anche in negativo: $\omega_8^{6} = \omega_8^{-2} $

\subsubsection{Lemma di Cancellazione}
\begin{lemma}[Lemma di Cancellazione]
    $\forall n, d > 0$ e $k \geq 0$ vale 
    \begin{equation*}
    % $
        \omega_{dn}^{dk} = \omega_n^k
    % $
    \end{equation*}
    \label{teo:cancellazione}
\end{lemma}
\begin{proof}
    $e^{i 2 \pi \frac{dk}{dn}} = e^{i 2 \pi \frac{k}{n}}$
\end{proof}
In più vale il seguente corollario
\begin{lemma}
    Se $n$ è pari, $ \omega_{n}^{n/2} =-1 $
\end{lemma}
\begin{proof}
    $ \omega_{n}^{n/2} =
    \omega_{2 \cdot n/2}^{1 \cdot n/2} =
    \omega_{2} =
    e^{i \frac{2 \pi}{2}} = -1
    $
\end{proof}
Nota: la radice principale quadrata dell'unità è $-1$

\subsubsection{Lemma di Dimezzamento}
\begin{lemma}
    Sia $n>0$, $n$ pari. Allora gli $n$ quadrati
    \begin{equation*}
        \left( \omega_{n}^{i} \right)^2 \quad 0 \leq i \leq n-1
    \end{equation*}
    coincidono a coppie, e in particolare
    \begin{equation*}
        \left( \omega_{n}^{i} \right)^2 = 
        \left( \omega_{n}^{i + n/2} \right)^2 = 
        \omega_{n/2}^{i}
    \end{equation*}
    \label{teo:dimezzamento}
\end{lemma}
\begin{proof}
    \begin{equation*}
        \left( \omega_{n}^{i} \right)^2 = 
        \omega_{n}^{2i}  = 
        \omega_{2 \cdot n/2}^{2 \cdot i}  = 
        \omega_{n/2}^{i}
    \end{equation*}
    \begin{equation*}
        \left( \omega_{n}^{i + n/2} \right)^2 = 
        \omega_{n}^{2i+n}  = 
        \omega_{n}^{2i} \cdot
        \cancel{ \omega_{n}^{n} } = 
        \omega_{2 \cdot n/2}^{2 \cdot i}  = 
        \omega_{n/2}^{i}
    \end{equation*}
\end{proof}

\subsubsection{Lemma di Somma}
\begin{lemma}
    Sia $n \geq 1$, $k \geq 0$
    \begin{equation*}
        \sum_{j=0}^{n-1} \left( \omega_n^k \right)^j = 
        \begin{cases}
            % n & \text{ se } k \mod n = 0 \\
            % n & \text{ se } k \pmod n = 0 \\
            n & \text{se } k \bmod n = 0 \\
            % n & \text{ se } k \mathrm{mod} n = 0 \\
            0 & \text{altrimenti}
        \end{cases}
    \end{equation*}
    \label{teo:somma}
\end{lemma}
\begin{proof}
    Nel primo caso $k \bmod n = 0$
    \begin{equation*}
        \omega_n^k = \omega_n^{k \bmod n} = \omega_n^0 = 1
        \quad \rightarrow \quad
        \sum_{j=0}^{n-1} 1 = n
    \end{equation*}
    Nel secondo caso $k \bmod n \neq 0$ e $\omega_n^k \neq 1$
    \begin{equation*}
        \sum_{j=0}^{n-1} \left( \omega_n^k \right)^j = 
        \frac{\left( \omega_n^k \right)^n -1}{\omega_n^k -1} =
        \frac{\left( \omega_n^n \right)^k -1}{\omega_n^k -1} =
        \frac{1 -1}{\omega_n^k -1} = 0
    \end{equation*}
\end{proof}

\section{Proprietà di sottostruttura}
Ricordando che $n=2^k$, $A(x) = \vec{a} \in \mathbb{C}^n$ e 
\begin{equation*}
    \vec{y} = DFT_n \left( \vec{a} \right) \quad \text{ con }
    \quad y_i = 
    A\left( \omega_n^i \right) = 
    \sum_{j=0}^{n-1} a_j \left( \omega_n^{i} \right)^{j}
\end{equation*}
si può sviluppare la proprietà di sottostruttura.
Per il caso base $n=1$
\begin{equation*}
    \vec{a} = a_0 \in \mathbb{C}^1
    \quad \rightarrow \quad
    % A(x) = a_0 |_{x=\omega_1^0} = a_0
    A(x) = a_0 \bigr|_{x=\omega_1^0} = a_0
\end{equation*}
ossia
\begin{equation*}
    \vec{y} = DFT_1 \left( \vec{a} \right) = \vec{a}
\end{equation*}
mentre per $n>1$ si può suddividere il vettore
$\vec{a}=\left( a_0, a_1,\cdots, a _{n-1} \right)$ in due vettori di lunghezza $n/2$ composti dalle sue componenti pari e dispari
\begin{equation*}
    \vec{a}^{[0]}=\left( a_0, a_2,\cdots, a _{n-2} \right)
    \quad \text{ e } \quad
    \vec{a}^{[1]}=\left( a_1, a_3,\cdots, a _{n-1} \right)
\end{equation*}
che avendo $n/2$ coefficienti sono di grado limitato da $n/2$

Si può quindi definire la seguente identità polinomiale su cui verrà costruita la proprietà di sottostruttura, che permetterà di ottenere $A(x)$ in funzione solamente di $A^{[0]} \left( x^2 \right)$ e $A^{[1]} \left( x^2 \right)$ 
\begin{definition}[Identità polinomiale]
    $\forall x \in \mathbb{C}$ vale
    \begin{equation*}
        A(x) = A^{[0]} \left( x^2 \right) + x \cdot A^{[1]} \left( x^2 \right) 
    \end{equation*}
    \label{def:identitapolinomiale}
\end{definition}
\begin{proof}
    I due polinomi sono definiti come
    \begin{equation*}
        A^{[0]}(x) = \sum_{j=0}^{\frac{n}{2}-1} a_{2j}x^j
        \quad \text{ e } \quad
        A^{[1]}(x) = \sum_{j=0}^{\frac{n}{2}-1} a_{2j+1}x^j
    \end{equation*}
    quindi valutando $A^{[0]}(x)$ in $x^2$ e sviluppando $x \cdot A^{[1]} \left( x^2 \right) $ risultano
    \begin{equation*}
        A^{[0]} \left( x^2 \right) = \sum_{j=0}^{\frac{n}{2}-1} a_{2j}x^{2j}
        \quad \text{ e } \quad
        x \cdot A^{[1]} \left( x^2 \right) =
        x \sum_{j=0}^{\frac{n}{2}-1} a_{2j+1}x^{2j} = 
        \sum_{j=0}^{\frac{n}{2}-1} a_{2j+1}x^{2j+1}
    \end{equation*}
    dove il primo contiene tutti i monomi pari di $A(x)$ e il secondo tutti i monomi dispari. Sommando i due termini si ottiene $A(x)$ come cercato.
\end{proof}

Il caso base è stato già trattato, supponiamo ora di aver già calcolato 
\begin{equation*}
    \vec{y}^{\,[0]} = DFT_{n/2} \left( \vec{a}^{[0]} \right)
    \quad \text{ e } \quad
    \vec{y}^{\,[1]} = DFT_{n/2} \left( \vec{a}^{[1]} \right)
\end{equation*}
mostriamo come ricavare
\begin{equation*}
    \vec{y} = DFT_{n} \left( \vec{a} \right)
\end{equation*}
utilizzando l'identità polinomiale.
In altri termini, si vogliono ottenere i valori incogniti 
\begin{equation*}
    y_i = A \left( \omega_{n}^{i} \right)
    \quad \forall \: 0 \leq i \leq n-1
\end{equation*}
a partire dai valori noti
\begin{equation*}
    y_i^{[0]} = A^{[0]} \left( \omega_{n/2}^{i} \right)
    \quad \text{ e } \quad
    y_i^{[1]} = A^{[1]} \left( \omega_{n/2}^{i} \right)
    \quad \forall \: 0 \leq i \leq \frac{n}{2}-1
\end{equation*}
Sia $0 \leq i \leq \frac{n}{2}-1$
\begin{align*}
    y_i &= A \left( \omega_{n}^{i} \right) 
    & \text{per l'identità polinomiale}
    \\
    &= A^{[0]} \left( \left( \omega_{n/2}^{i} \right)^2 \right) +
    A^{[1]} \left( \left( \omega_{n/2}^{i} \right)^2 \right)
    & \text{per il lemma di Dimezzamento}
    \\
    &= A^{[0]} \left( \omega_{n/2}^{i} \right) +
    A^{[1]} \left( \omega_{n/2}^{i} \right)
    \\
    &= y_i^{[0]} + \omega_{n}^{i} y_i^{[1]} 
\end{align*}
quindi i primi $n/2$ valori vengono trovati con semplici combinazioni dei valori ottenuti per ricorsione. Cerchiamo gli altri $n/2$ valori, sia nuovamente $0 \leq i \leq \frac{n}{2}-1$
\begin{align*}
    y_{i+\frac{n}{2}} &= A \left( \omega_{n}^{i} \right) 
    & \text{per l'identità polinomiale}
    \\
    &= A^{[0]} \left( \left( \omega_{n/2}^{i+n/2} \right)^2 \right) +
    \omega_{n}^{i+n/2} \,
    A^{[1]} \left( \left( \omega_{n/2}^{i+n/2} \right)^2 \right)
    & \text{per il lemma di Dimezzamento}
    \\
    &= A^{[0]} \left( \omega_{n/2}^{i} \right) +
    \omega_{n}^{i} \,
    \omega_{n}^{n/2} \,
    A^{[1]} \left( \omega_{n/2}^{i} \right)
    & \omega_{n}^{n/2} = -1
    \\
    &= y_i^{[0]} - \omega_{n}^{i} \, y_i^{[1]} 
\end{align*}
Si possono quindi calcolare tutti i punti di $A$ con quelli di $ A^{[0]} $ e $ A^{[1]}$

\section{Trasformata veloce di \textit{Fourier}}

\subsection{Algoritmo della trasformata veloce di \textit{Fourier}}

Sfuttando la proprietà di sottostruttura sviluppata nella sezione precedente, si può scrivere l'algoritmo che implementa la trasformata veloce di \textit{Fourier}
\begin{algorithm}[H]
\caption{Trasformata veloce di \textit{Fourier}}\label{alg:fft}
\begin{algorithmic}[1]
    \Procedure{FFT}{$\vec{a}$}
        \State $n \gets \vec{a}.length$
        \If{$n=1$}
        \Comment{BASE}
            \State return $\vec{a}$
        \EndIf
        \State $\vec{a}^{[0]}=\left( a_0, a_2,\cdots, a _{n-2} \right)$
        \label{alg:fft:a0}
        \Comment{DIVIDE}
        \State $\vec{a}^{[1]}=\left( a_1, a_3,\cdots, a _{n-1} \right)$
        \State $\vec{y}^{[0]} \gets \Call{FFT}{\vec{a}^{[0]}}$
        \Comment{RECURSE}
        \State $\vec{y}^{[1]} \gets \Call{FFT}{\vec{a}^{[1]}}$
        \State $ON \gets e^{i 2 \pi/n}$   
        \Comment{CONQUER}
        \State $ONI \gets 1$   
        \label{alg:fft:oni}
        \For{$i \gets 0 $ to $ n/2-1 $ }                   
            \State $y_{i} \gets y_{i}^{[0]} + ONI \cdot y_i^{[1]}$
            \label{alg:fft:con}
            \State $y_{i+n/2} \gets y_{i}^{[0]} - ONI \cdot y_i^{[1]}$
            \State $ONI \gets ONI \cdot ON$
        \EndFor
        \State return $\vec{y}$
    \EndProcedure
\end{algorithmic}
\end{algorithm}

Note: il $Divide$ può essere implementato senza utilizzare spazio aggiuntivo. Alla riga~\ref{alg:fft:oni} viene salvato il valore iniziale di $\omega_n^i$, che verrà poi incrementato nel corso dell'algoritmo. Alla riga~\ref{alg:fft:con} è presente un'inefficienza, ma un compilatore degno ottimizzerebbe automaticamente il codice, salvando il risultato del prodotto in un temporaneo.

La correttezza dell'algoritmo è provata facilmente per la base, e assumendo di avere a disposizione i valori corretti per $n/2$, il \textit{Conquer} implementa la proprietà di sottostruttura e la corretteza segue immediatamente.

L'algoritmo effettua $2$ chiamate ricorsive su istanze di taglia $n/2$ e $5$ operazioni di somma e prodotto su istanze di taglia $n/2$ e l'equazione del \textit{Master Theorem} risulta
\begin{equation*}
    T_{FFT} (n) = 2 T_{FFT} \left( \frac{n}{2} \right) + 5 \frac{n}{2} %\Theta \left( n \right)
\end{equation*}
da cui ricaviamo la funzione di soglia $n^{\log_2 2} = n$, $w(n)= \Theta \left( n \right)$ e quindi $n = \Theta \left( w\left( n \right) \right)$ per cui siamo nel caso $2$, dove è stato ottenuto un perfetto bilanciamento di soglia e lavoro. La complessità risulta
\begin{equation*}
    T_{FFT} (n) = \Theta \left( n \log n \right)
\end{equation*}

\subsection{Trasformata inversa}
Anche per l'interpolazione esiste un algoritmo veloce, ricordiamo che si può riscivere la trasformata discreta come operatore lineare 
$$\vec{y} = DFT_n \left( \vec{a} \right) = F_n \vec{a}$$
dove $F_n$ è una matrice di \textit{Vandermonde} della forma
\begin{equation*}
    F_n = V\left( \omega_n^0, \cdots, \omega_n^i, \cdots \right) =
    \left(
        \begin{array}[h]{ccccc}
            1 & \cdots & & \cdots & 1 \\
            \vdots \\
            & & \omega_n^{ij} \\
            \vdots \\
            1\\
        \end{array}
    \right)
\end{equation*}
La trasformata inversa si può scrivere come 
$$\vec{a} = DFT_n^{-1} \left( \vec{y} \right) = F_n^{-1}\vec{y}$$
dimostriamo che gli elementi di $F_n^{-1}$ sono 
\begin{equation*}
    \left( F_n^{-1} \right)_{ij} = \frac{1}{n} \omega_n^{-ij}
\end{equation*}
dimostrando che
\begin{equation*}
    F_n F_n^{-1} = F_n^{-1} F_n = I_n
\end{equation*}
nel secondo caso, scriviamo il generico prodotto righe per colonne dell'elemento $ij$
\begin{equation*}
    \left( F_n^{-1} F_n \right)_{ij} =
    \frac{1}{n} \sum_{k=0}^{n-1} \omega_n^{-ik} \omega_n^{kj} =
    \frac{1}{n} \sum_{k=0}^{n-1} \omega_n^{(j-i)k}
\end{equation*}
sulla diagonale principale $i=j$ deve valere $1$ e infatti
\begin{equation*}
    \frac{1}{n} \sum_{k=0}^{n-1} \omega_n^{0} =
    \frac{n}{n} = 1
\end{equation*}
negli altri casi, con $i \neq j$ osserviamo che $j-i$ può assumere solo valori maggiori di $-(n-1)$ quando $j=0, \; i=n-1$ e minori di $n-1$ quando $j=n-1, \; i=0$
\begin{equation*}
    -(n-1) \leq j-i \leq n-1
    \quad \text{con }j-i \neq 0 
    \text{ considerato nell'altro caso}
\end{equation*}
e per il lemma di somma
\begin{equation*}
    \frac{1}{n} \sum_{k=0}^{n-1} \left( \omega_n^{j-i} \right)^k = 0
\end{equation*}
dato che l'esponente non è multiplo di $n$. La prima uguaglianza si dimostra in modo analogo.

Possiamo interpretare $\vec{y} \equiv Y(x)$ come rappresentazione per coefficienti di $Y$ e quindi
\begin{equation*}
\left( F_n^{-1} \vec{y} \, \right)_i = \frac{1}{n} Y \left( \omega_n^{-1} \right)
\end{equation*}
vedendo l'interpolazione come una valutazione scalata di $Y$ in $\left( \omega_n^{-1} \right)$, avendo interpretato come una matrice di \textit{Vandermonde} $F_n^{-1}$
\begin{equation*}
    F_n^{-1} = \frac{1}{n} V \left( \omega_n^{-0}, \omega_n^{-1}, \cdots, \omega_n^{-\left( n-1 \right)} \right)
\end{equation*}
% in più definendo $\bar{\omega}_n$
In più definendo $\overline{\omega}_n = \omega_n^{-1}$
% in più definendo $\overline{\omega_n}$
(da cui $\overline{\omega}_n^{\,2} = \omega_n^{-2}$ e così via fino a $\overline{\omega}_n^{\,n-1} = \omega_n$)
rispetto a $\overline{\omega}_n$ si applicano tutti i lemmi e la proprietà di sottostruttura è valida. Si possono quindi ottenere le valutazioni di un polinomio $Y(x)$ su $\left( \left( \omega_n^{-1} \right)^{0}, \omega_n^{-1}, \cdots \left( \omega_n^{-1} \right)^{n-1}, \right)$ utilizzando un algoritmo identico a $FFT$, $\overline{FFT}$, dove l'unica differenza nel codice è $ON \gets e^{-i2\pi/n} = \omega_n^{-1}$ che è appunto corretto perché rispetto a $\omega_n^{-1}$ vale la proprietà di sottostruttura. Questo algoritmo \emph{non} implementa la trasformata inversa, occorre ancora dividere per $1/n$. Supponendo $\overline{FFT} \left( \vec{y} \right)$ sia noto, si ricava $DFT_n^{-1} \left( \vec{y} \right)$

\begin{algorithm}[H]
\caption{Trasformata inversa}\label{alg:invfft}
\begin{algorithmic}[1]
    \Procedure{INV\_FFT}{$\vec{y}$}
        \State $n \gets \vec{y}.length$
        \State $\vec{a} \gets \overline{FFT} \left( \vec{y} \right)$
        \Comment{algoritmo ausiliario}
        \For{$i \gets 0 $ to $ n-1 $ }
            % \State $a_i \gets \frac{1}{n}a_i$
            \State $a_i \gets a_i / n$
        \EndFor
        \State return $\vec{a}$
    \EndProcedure
\end{algorithmic}
\end{algorithm}

Nota: vale $T_{INV\_FFT}(n) = \Theta \left( n \log n \right)$

L'algoritmo $INV\_FFT$ \emph{non} è ricorsivo, $\overline{FFT}$ lo è.

Si può ottenere l'algoritmo per $DFT_n^{-1} \left( \vec{y} \right)$ in un altro modo. Notiamo che 
$\omega_n^{-0} = \omega_n^{0} = 1$ e $\forall 1 \leq j \leq n-1$ vale
$\omega_n^{-j} = \omega_n^{-j \bmod n} = \omega_n^{n-j}$
quindi possiamo scrivere
\begin{equation*}
    Y \left( \omega_n^{-j} \right) =
        \begin{cases}
            Y\left( \omega_n^{0} \right) & j=0 \\
            Y \left( \omega_n^{n-j} \right) & j \neq 0
        \end{cases}
\end{equation*}
e le componenti sono quindi quelle di \textit{Fourier} positive ma in ordine diverso e le valutazioni so potenze negative sono le positive permutate.
\begin{equation*}
    \vec{a} = DFT_n^{-1} \left( \vec{y} \right) \quad \rightarrow \quad
    \begin{cases}
        a_0 = \frac{1}{n} Y\left( \omega_n^{0} \right) = 
        \frac{1}{n} \left[ DFT_n \left( \vec{y} \right) \right]_0
        & j=0
        \\
        a_j = \frac{1}{n} Y\left( \omega_n^{-j} \right) = 
        \frac{1}{n} \left[ DFT_n \left( \vec{y} \right) \right]_{n-j}
        & j>0
    \end{cases}
\end{equation*}

\begin{algorithm}[H]
\caption{Trasformata inversa}\label{alg:invfftbis}
\begin{algorithmic}[1]
    \Procedure{INV\_FFT}{$\vec{y}$}
        \State $n \gets \vec{y}.length$
        \State $\vec{z} \gets FFT \left( \vec{y} \right)$
        \State $a_0 \gets z_0 / n$
        \For{ $j \gets 1 $ to $ n-1 $ }
            \State $a_j \gets z_{n-j} / n $
        \EndFor
        \State return $\vec{a}$
    \EndProcedure
\end{algorithmic}
\end{algorithm}

Introducendo l'operatore vettoriale $\cdot^{rev}$ definito come
\begin{equation*}
    \vec{a} = \left( a_0, a_1, \cdots, a_{n-1} \right)
    \quad \rightarrow \quad
    \vec{a}^{\, rev} = \left( a_0, a_{n-1}, \cdots, a_1 \right)
\end{equation*}
si può riscrivere la trasformata inversa come
\begin{equation*}
    DFT_{n}^{-1} \left( \vec{y} \, \right) = \frac{1}{n} \left( DFT_{n} \left( \vec{y} \, \right) \right)^{rev}
\end{equation*}

\section{Convoluzione lineare}

Utilizzando l'algoritmo della trasformata veloce, si può scrivere un algoritmo efficiente per la convoluzione lineare
\begin{equation*}
    \vec{c} = \vec{a} * \vec{b}
    \quad
    \vec{a}, \vec{b} \in \mathbb{C}^n, \vec{c} \in \mathbb{C}^{2n-1}
\end{equation*}
supponendo entrambi i vettori di lunghezza uguale e potenza di due.

\begin{algorithm}[H]
\caption{Convoluzione lineare}\label{alg:convlin}
\begin{algorithmic}[1]
    \Procedure{LIN\_CONV}{$\vec{a}, \vec{b}$}
        \State $n \gets \vec{a}.length$
        \State $\vec{a}\,' \gets \left( \vec{a} \, | 0_n \right)$
        \Comment{$\vec{a}\,', \vec{b}\,' \in \mathbb{C}^{2n} $ }
        % \State $\vec{a}' \gets \left( \vec{a} \, | 0_n \right)$
        \State $\vec{b}\,' \gets \left( \vec{b} \, | 0_n \right)$
        \State $\vec{y}_a \gets FFT \left( \vec{a}\,' \right)$
        \Comment{valutazioni di $A(x)$ e $B(x)$ estese su $2n$ punti}
        \State $\vec{y}_b \gets FFT \left( \vec{b}\,' \right)$
        \For{ $i \gets 0 $ to $ 2n-1 $ }
        \State $ \left( y_c \right)_i = \left( y_a \right)_i \cdot \left( y_b \right)_i $
        \EndFor
        \State return \Call{INV\_FFT}{$\vec{y}_c$}
    \EndProcedure
\end{algorithmic}
\end{algorithm}

Dove il polinomio prodotto $\vec{c} = \text{INV\_FFT}\left( \vec{y}_c \right) \in \mathbb{C}^{2n}$ e vale sempre $c_{2n-1}=0$

Vengono eseguite $3$ trasformate su vettori di $2n$ elementi, per cui
\begin{equation*}
    T_{LC} (n) = \Theta \left( n \log n \right)
\end{equation*}

Si può quindi enunciare il seguente teorema della convoluzione lineare

\begin{theorem}[Convoluzione lineare]
    Se $\vec{c} = \vec{a} * \vec{b}$ allora
    \begin{equation*}
        \vec{c} = DFT_{2n}^{-1} \left( 
            DFT_{2n} \left( \vec{a} \, | 0_n \right)
            \odot
            DFT_{2n} \left( \vec{b} \, | 0_n \right)
        \right)
    \end{equation*}
    \label{teo:convlin}
\end{theorem}

\section{Trasformate notevoli}

\subsection{Vettore costante}
Calcoliamo il valore di $DFT_n \left( a, a, \cdots, a \right) = F_n \vec{a}$, con $F_n$ definita all'equazione \ref{eq:matricefourier}.
\begin{equation*}
    F_n
    \left( 
        \begin{array}[h]{c}
            a \\ a \\ \vdots \\ a
        \end{array}
    \right)
    =
    a F_n
    \left( 
        \begin{array}[h]{c}
            1 \\ 1 \\ \vdots \\ 1
        \end{array}
    \right)
    \overset{\circola{1}}{=}
    a
    \left( 
        \begin{array}[h]{c}
            \sum_{j=0}^{n-1} \left( \omega_n^i \right)^j  \\
            \vdots
        \end{array}
    \right)
    \overset{\circola{2}}{=}
    a
    \left( 
        \begin{array}[h]{c}
            n \\ 0 \\ \vdots \\ 0
        \end{array}
    \right)
    =
    \left( 
        \begin{array}[h]{c}
            an \\ 0 \\ \vdots \\ 0
        \end{array}
    \right)
\end{equation*}
Dove in $\circola{1}$ il vettore contiene in ogni componente la somma di tutti gli elementi sulla stessa riga della matrice di \textit{Fourier} e in $\circola{2}$ si applica il lemma di somma \ref{teo:somma}.

\subsection{Vettore impulsivo}
\label{sss:vetimp}
Calcoliamo il valore di $DFT_n \left( a, 0, \cdots, 0 \right)$
\begin{equation*}
    F_n
    \left( 
        \begin{array}[h]{c}
            a \\ 0 \\ \vdots \\ 0
        \end{array}
    \right)
    =
    a F_n
    \left( 
        \begin{array}[h]{c}
            1 \\ 0 \\ \vdots \\ 0
        \end{array}
    \right)
    \overset{\circola{1}}{=}
    a \left( F_n \right)^1
    =
    a
    \left( 
        \begin{array}[h]{c}
            1 \\ 1 \\ \vdots \\ 1
        \end{array}
    \right)
    =
    \left( 
        \begin{array}[h]{c}
            a \\ a \\ \vdots \\ a
        \end{array}
    \right)
\end{equation*}
Dove in $\circola{1}$ viene selezionata la prima riga di $F_n$, composta tutta di $1$.

\subsection{Vettore $(n,k)$-sparso}
\begin{definition}[Vettore $(n,k)$-sparso]
    Un vettore si dice $(n,k)$-sparso se, dati $1 \leq k \leq n$, con $n,k$ potenze di due, e dato $\vec{x} \in \mathbb{C}^n$ vale
    \begin{equation*}
        x_i = 0 \quad \forall i : i \bmod k \neq 0
    \end{equation*}
    Ossia il vettore ha componenti nulle per gli indici che \emph{non} sono multipli di $k$.
    \label{def:nksparso}
\end{definition}
Il segnale è spaziato in maniera uniforme, diviso in $n/k$ sottocomponenti di lunghezza $k$.
Esempio: $n=8, k=4 \rightarrow \left( x_0,0,0,0,x_4,0,0,0 \right)$

Si può quindi seguire l'algoritmo della $FFT$ e cercare una sottostruttura più potente. Nel caso base $k=1$ il vettore $\left( n,1 \right)$-sparso risulta un vettore generico $\vec{x}=\left( x_0, x_1,\cdots, x _{n-1} \right) \in \mathbb{C}^n$ quindi non si può fare di meglio di $FFT\left( \vec{x} \right)$. Per il caso generico $k>1$, riscrivendo i due vettori delle componenti pari e dispari possiamo notare come il secondo abbia appunto tutte componenti relative ad indici dispari, che per un vettore sparso sono tutte pari a zero.
\begin{equation*}
    \vec{x}^{[0]}=\left( x_0, x_2,\cdots, x _{n-2} \right)
    \quad 
    \vec{x}^{[1]}=\left( x_1, x_3,\cdots, x _{n-1} \right) = \vec{0}_{n/2}
\end{equation*}
Metà del lavoro ricorsivo è quindi eliminato.
Per essere una proprietà di sottostruttura valida, per qualche $k'$ anche $\vec{x}^{[0]} \in \mathbb{C}^{n/2}$ deve essere sparso. Consideriamo $0 \leq i \leq n/2-1$, vale
\begin{align*}
    \vec{x}_i^{[0]} = x_{2i} \meq 0 
    & \leftrightarrow
    2i \bmod k \neq 0
    \\
    & \leftrightarrow
    \nexists \, m : 2i = mk
    \\
    & \leftrightarrow
    \nexists \, i : m\frac{k}{2} = i
    \\
    & \leftrightarrow
    \nexists \, i \bmod \frac{k}{2} \neq 0
\end{align*}
Quindi $\vec{x}_i^{[0]}$ è uguale a zero per ogni $i$ non multiplo di $k/2$, ovvero $\vec{x}_i^{[0]}$ è $\left( \frac{n}{2}, \frac{k}{2} \right)$-sparso.

Per un vettore $\vec{x}$ $(n,k)$-sparso, con $1 \leq k \leq n$ e $n,k$ potenze di $2$, si può riscrivere l'algoritmo della trasformata di \textit{Fourier} come
\begin{algorithm}[H]
    \caption{Trasformata veloce di \textit{Fourier} per vettori $(n,k)$-sparsi}\label{alg:fftsparsa}
\begin{algorithmic}[1]
    \Procedure{SPARSE\_FFT}{$\vec{x}, k$}
        \State $n \gets \vec{x}.length$
        \If{$k=1$}
        \Comment{BASE}
            \State return $\Call{FFT}{\vec{x}}$
        \EndIf
        \State $\vec{x}^{[0]}=\left( x_0, x_2,\cdots, x _{n-2} \right)$
        \Comment{DIVIDE}
        \State $\vec{y}^{[0]} \gets \Call{SPARSE\_FFT}{\vec{x}^{[0]}, k/2}$
        \Comment{RECURSE}
        \For{$i \gets 0 $ to $ n/2-1 $ }                   
        \Comment{CONQUER}
            \State $y_{i} \gets y_{i}^{[0]}$
            \State $y_{i+n/2} \gets y_{i}^{[0]}$
        \EndFor
        \State return $\vec{y}$
    \EndProcedure
\end{algorithmic}
\end{algorithm}
Dove si può notare che non è stato necessario calcolare $\omega_n$, e le due componenti del vettore $\vec{y}$ sono uguali.

La complessità è funzione di due parametri:
\begin{equation*}
    T(n,k) = 
    \begin{cases}
        c n \log n & k=1 \\
        T \left( \frac{n}{2}, \frac{k}{2} \right) + 0 & k>1
    \end{cases}
\end{equation*}
dove nel modello di costo il lavoro del \emph{Conquer} è pari a zero.
Si può calcolare la soluzione per \emph{unfolding}:
\begin{equation*}
    T(n,k) = T \left( \frac{n}{2}, \frac{k}{2} \right)
    = \cdots
    =  T \left( \frac{n}{2^{i}}, \frac{k}{2^{i}} \right)
    \overset{\circola{1}}{=}
    T \left( \frac{n}{2^{\log_2 k}}, 1 \right)
    = c \frac{n}{k} \log_2 \left( \frac{n}{k} \right)
\end{equation*}
Dove in $\circola{1}$ si è calcolato il valore di $i$ per cui $k/2^i=1$

TODO per ESERCIZIO si prova la correttezza per induzione.

Notiamo che al crescere di $k$ la complessità diminuisce. Nei casi limite per $k=1$ la complessità diventa $c n \log n$ come l'algoritmo standard, mentre per $k=n$ un vettore $(n,n)$-sparso è un vettore impulsivo e la sua trasformata è un vettore costante, confermando quanto calcolato nella sezione \ref{sss:vetimp}.

TODO dimostra per ESERCIZIO che la forma generica di una trasformata di un vettore $(n,k)$-sparso è periodica.


\chapter{Programmazione dinamica}
\section{Introduzione}

% TODO disegno albero TD BU pag 53.2
Il problema del memorylessness del paradigma \emph{Divide and Conquer} è già stato sottolineato nella sezione \ref{sss:alberochiamate}.
Per esemplificare il problema, si introduce la sequnza di Fibonacci, definita come
\begin{equation*}
    F_n = 
    \begin{cases}
        1 & n=0,1 \\
        F_{n-1} + F_{n-2} & n>1
    \end{cases}
\end{equation*}
da cui si ricava immediatamente un algoritmo ricorsivo
\begin{algorithm}[H]
\caption{Fibonacci ricorsivo}\label{alg:rfib}
\begin{algorithmic}[1]
    \Procedure{R\_FIB}{$n$}
        \If{$ n=0 $ or $n=1$}
            \State return $1$
        \EndIf
        \State return \Call{R\_FIB}{$n-1$} + \Call{R\_FIB}{$n-2$}
    \EndProcedure
\end{algorithmic}
\end{algorithm}
% TODO disegno albero chiamate R_FIB pag 53.4
\noindent
la cui equazione di ricorrenza è
\begin{equation*}
    T_{RF}(n) = 
    \begin{cases}
        0 & n=0,1 \\
        T_{RF}(n-1) + T_{RF}(n-2) + 1 & n>1
    \end{cases}
\end{equation*}
che risulta limitata inferiormente da un esponenziale, infatti per $n>1$
\begin{align*}
    T_{RF}(n) 
    &= T_{RF}(n-1) + T_{RF}(n-2) + 1 \\
    & \geq \, 2 T_{RF}(n-2) + 1 \\
    & \geq \, 2^2 T_{RF}(n-2-2) + 2 + 1 \\
    & \geq \, 2^i T_{RF}(n-2i) + \sum_{j=0}^{i-1} 2^j \\
    \intertext{che raggiunge il caso base quando $n-2i=0$ o $n-2i=1$ per $i= \left\lfloor n/2 \right\rfloor $ sia nel caso di $n$ pari sia nel caso di $n$ dispari}
    & \geq \, 2^{\left\lfloor n/2 \right\rfloor} \cancel{ T_{RF}(0 \text{ o } 1)}
    + \sum_{j=0}^{\left\lfloor n/2 \right\rfloor -1} 2^j \\
    &= 2^{\left\lfloor n/2 \right\rfloor} -1 \\
    &=  \Omega \left( 2^{n/2} \right) = \sqrt{2}^{\,n}
    \intertext{ed essendo $\sqrt{2}>1$ cresce più velocemente di ogni polinomio. Il valore esatto di $T_{RF}$ è}
    T_{RF}(n) &= \Theta \left( \left( \frac{1+\sqrt{5}}{2} \right)^n \right)
\end{align*}

Il sugo della storia è che viene calcolata un numero molto elevato di volte la stessa sottoistanza. Ispirandosi alla fase \emph{Bottom-Up} del \emph{D\&C}, e sfruttando strutture dati che contengano le informazioni necessarie, si può scrivere un algoritmo iterativo che risolve il problema.
\begin{algorithm}[H]
\caption{Fibonacci iterativo}\label{alg:itfib}
\begin{algorithmic}[1]
    \Procedure{IT\_FIB}{$n$}
        \If{$ n=0 $ or $n=1$}
            \State return $1$
        \EndIf
        \State $F[0] \gets 1$
        \State $F[1] \gets 1$
        \For{$i \gets 2 $ to $ n $ } 
            \State $F[i] \gets F[i-1] + F[i-2]$
        \EndFor
        \State return $F[n]$
    \EndProcedure
\end{algorithmic}
\end{algorithm}
\noindent
dove le soluzioni intermedie sono salvate in $F$. In realtà sarebbe sufficiente memorizzare solo gli ultimi due valori della sequenza.

Si può quindi introdurre il paradigma del \emph{Dynamic Programming}, basandosi su queste osservazioni.

\subsection{Paradigma del \emph{Dynamic Programming}}
% \subsection{Introduzione}
I due concetti fondamentali su cui si basa il paradigma \emph{Dynamic Programming} sono
\begin{itemize}[noitemsep,topsep=0pt,parsep=0pt,partopsep=0pt]
    \item[--] dotare di memoria l'algoritmo
    \item[--] implementare la computazione in direzione \emph{bottom-up} (tutti i dati necessari sono già stati calcolati nelle iterazioni precedenti)
\end{itemize}
In ogni caso si lavora con la proprietà di sottostruttura, generando la soluzione ad un'istanza in funzione di sottoistanze di taglia minore.o

Il vantaggi sono una maggiore velocità dovuta al non replicare la computazione, gli svantaggi sono legati al dover implementare la computazione in maniera da essere sicuri di avere a disposizione le soluzioni necessarie al momento giusto, che per casi articolati non è triviale. Nel caso del \emph{D\&C} si sfrutta la convenienza della fase \emph{top-down}, che genera e risolve le istanze in modo coerente.

L'algoritmo si può comunque scrivere in maniera ricorsiva.

\subsection{Memoizzazione di un algoritmo ricorsivo}

È possibile modificare un algoritmo ricorsivo \emph{D\&C} memorizzando le soluzioni intermedie attraverso un processo di memoizzazione.

\subsubsection{Metodo generale}

Un algoritmo memoizzato è costituito da due subroutine:
\begin{enumerate}
    \item \textbf{Routine di inizializzazione} INIT\_\{AlgName\}
        \begin{itemize}
            \item risolve i casi di base direttamente
            \item inizializza una struttura tabellare \emph{globale} con
                \begin{itemize}
                    \item valori delle istanze di base, nelle locazioni associate alle istanze di base
                    \item valori di default in posizioni associate a istanze non di base (il valore di default deve essere scelto in modo da far capire che non è stato ancora calcolato)
                \end{itemize}
            \item invoca la seconda procedura (ricorsiva) \\ % FORMATTA
        \end{itemize}
    \item \textbf{Routine ricorsiva} REC\_\{AlgName\}($i$)
        \begin{itemize}
            \item controlla sulla tabella per vedere se $i$ è già stata risolta
                \begin{itemize}
                    \item se sì la ritorna
                    \item se no
                        \begin{itemize}
                            \item la calcola con la proprietà di sottostruttura
                            \item la memorizza nella tabella
                            \item la ritorna
                        \end{itemize}
                \end{itemize}
        \end{itemize}
\end{enumerate}

\textbf{Nota:} lo spazio delle sottoistanze deve essere 
\begin{itemize}[noitemsep,topsep=0pt,parsep=0pt,partopsep=0pt]
    \item[--] piccolo
    \item[--] facilmente indicizzabile
\end{itemize}

\subsubsection{Algoritmo di Fibonacci memoizzato}

\section{Problemi di ottimizzazione combinatoria}

\subsection{Definizione}

\subsection{Paradigma generale \emph{Dynamic Programming}}

\subsection{autocomplete}
Una bella scatola:
\begin{equation}
    \boxed{x^2+y^2 = z^2}
\end{equation}

Numeri nei casi
\begin{numcases}{T(n)=}
    2^3 \label{escaso1} \\
    2^4 \label{escaso2} 
\end{numcases}

Sotto numeri
\begin{subnumcases}{T(n)=}
    2^3 \label{escaso3} \\
    2^4 
\end{subnumcases}

\begin{itemize}[noitemsep,topsep=0pt,parsep=0pt,partopsep=0pt]
    \item qualcosa
    \item[+] qualcosa
    \item[*] qualcosa
    \item[--] qualcosa
\end{itemize}
àg
èg
ìg
òg
ùg
perché

delirio di vim se scrivi \verb|<C-k>`e| o \verb|<C-k>e`| in insert mode mette una è

% delirio doppio di vim se scrivi \verb|<C-k>da| in insert mode mette ``Hiragana letter DA'' che purtroppo non posso mostrarvi %だ
% insomma i digraph sono tanti e belli

Spazietti fra equazioni
\begin{equation*}
    A^{[0]}(x) = \sum_{j=0}^{\frac{n}{2}-1} a_{2j}x^j
    \quad \text{ e } \quad
    A^{[1]}(x) = \sum_{j=0}^{\frac{n}{2}-1} a_{2j+1}x^j
\end{equation*}

Un gustoso algoritmo
\begin{algorithm}[H]
\caption{Divide and Conquer}\label{alg:dncmock}
\begin{algorithmic}[1]
    \Procedure{D\&C}{$i$}
        \If{$|i| \leq n_0$}                             \Comment{BASE}
            \State *risolvo direttamente*
        \EndIf
        \State $<i_1, i_2, \dots, i_k> \gets A_D(i)$    \Comment{DIVIDE}
        \For{$j \gets 1 $ to $ k $ }                    \Comment{RECURSE}
            \State $s_j \gets $ \Call{D\&C}{$i_j$}
        \EndFor
        \State $s \gets A_C(<s_1, s_2, \dots, s_k>)$    \Comment{CONQUER}
        \State return $s$
    \EndProcedure
\end{algorithmic}
\end{algorithm}





\chapter{Paradigma \emph{Greedy}}
\section{Paradigma \emph{Greedy}}

\subsection{Introduzione}

Il problema del memorylessness del paradigma \emph{Divide and Conquer} è stato risolto utilizzando il paradigma del \emph{Dynamic Programming}. Risolvendo un problema con la programmazione dinamica, però, la soluzione viene costruita componendo le soluzioni di sottoistanze, scegliendo di volta in volta la sottosoluzione migliore. Questa scelta può avvenire solo dopo aver calcolato \emph{tutte} le soluzioni alle sottoistanze.

Il paradigma \emph{Greedy} seleziona ad ogni iterazione la scelta più promettente, e calcola la soluzione alla sottoistanza relativa \emph{solo} a quella scelta.

Occorre dimostrare che la scelta non comprometta l'ottimalità della soluzione.

\subsection{Definizione}

Il paradigma \emph{Greedy} agisce in tre passi:
\begin{enumerate}
    \item Scelta \emph{Greedy}: compie una scelta che sembra essere quella più promettente, localmente ottima, che non comprometta la soluzione: la soluzione ottima conterrà quella scelta.
    \item \emph{Clean up}: l'istanza viene ripulita, in accordo con la scelta effettuata.
    \item \emph{Tail recursion}: viene risolta l'\emph{unica} istanza generata, come ultimo comando della funzione. Questo tipo di ricorsione può sempre essere scritto in maniera iterativa.
\end{enumerate}

Vanno quindi dimostrate due proprietà:
\begin{enumerate}
    \item la scelta \emph{Greedy} (SG) non compromette l'ottimalità della soluzione locale: \\
        $\exists S^*$ che contiene la scelta \emph{Greedy}
    \item $\exists S^*$ che, oltre alla scelta \emph{Greedy}, contiene la soluzione della sottoistanza ottenuta dal \emph{clean up}, detta sottoistanza residua
\end{enumerate}

\subsection{autocomplete}
Snippet di \LaTeX{} che tornano spesso utili

Una bella scatola:
\begin{equation}
    \boxed{x^2+y^2 = z^2}
\end{equation}

Numeri nei casi
\begin{numcases}{T(n)=}
    2^3 \label{escaso1} \\
    2^4 \label{escaso2} 
\end{numcases}

Sotto numeri
\begin{subnumcases}{T(n)=}
    2^3 \label{escaso3} \\
    2^4 
\end{subnumcases}

Liste compatte
\begin{itemize}[noitemsep,topsep=0pt,parsep=0pt,partopsep=0pt]
    \item qualcosa
    \item[+] qualcosa
    \item[*] qualcosa
    \item[--] qualcosa
\end{itemize}

Parole in libertà per l'autocomplete: 
à
è
ì
ò
ù
perché
così
sì
può
più

Viva vim se scrivi \verb|<C-k>`e| o \verb|<C-k>e`| in insert mode mette una è

% delirio doppio di vim se scrivi \verb|<C-k>da| in insert mode mette ``Hiragana letter DA'' che purtroppo non posso mostrarvi %だ
% insomma i digraph sono tanti e belli

Spazietti fra equazioni
\begin{equation*}
    A^{[0]}(x) = \sum_{j=0}^{\frac{n}{2}-1} a_{2j}x^j
    \quad \text{ e } \quad
    A^{[1]}(x) = \sum_{j=0}^{\frac{n}{2}-1} a_{2j+1}x^j
\end{equation*}

Un gustoso algoritmo
\begin{algorithm}[H]
\caption{Divide and Conquer}\label{alg:dncmock}
\begin{algorithmic}[1]
    \Procedure{D\&C}{$i$}
        \If{$|i| \leq n_0$}                             \Comment{BASE}
            \State *risolvo direttamente*
        \EndIf
        \State $<i_1, i_2, \dots, i_k> \gets A_D(i)$    \Comment{DIVIDE}
        \For{$j \gets 1 $ to $ k $ }                    \Comment{RECURSE}
            \State $s_j \gets $ \Call{D\&C}{$i_j$}
        \EndFor
        \State $s \gets A_C(<s_1, s_2, \dots, s_k>)$    \Comment{CONQUER}
        \State return $s$
    \EndProcedure
\end{algorithmic}
\end{algorithm}




\cleardoublepage
% \fancyfoot[LO,RE]{Bibliography}
\lfoot{Bibliography}
\bibliographystyle{unsrt}
\bibliography{biblio}
\cleardoublepage

\appendix
% \fancyfoot[LO,RE]{Appendix \thechapter}
\lfoot{Appendice \thechapter}

\chapter{Appendice}
Appendice con info utili

Dovrebbe chiamarsi Appendice F


\end{document}

%% ci sono cose interessanti su fancyfoot in thesis-example
