\section{Contrazioni}
Definizione contrazione, 13.6

\subsection{Iterate}
Def iterate, pag 13.8

Nota successione decrescente 14.3

\subsection{Ausiliaria}
Def ausiliaria, pag 14.6

\subsection{Esempi}
\subsubsection{ esempio $\bm{n/2}$}
pag 13.9, 14.8

\subsubsection{ esempio $\bm{n-1}$}
\subsubsection{ esempio $\bm{\sqrt{n}}$}
\subsubsection{ esempio $\bm{\left\lfloor n/2 \right\rfloor}$}

\section{Modifica al \textit{Divide and Conquer}}
% titolo meno informativo della storia
\subsection{Metaalgoritmo}
Metaalgoritmo DnC modificato, pag 12.5

Parametri noti, pag 13

Nuova eq di ricorrenza, pag 13.3

\subsubsection{esempio parametri MAX}
parametri max, pag 13.5

\subsection{Equazione delle ricorrenze generica}
Alberone, pag 15.8

Formulone, pag 16

Da qualche parte le convenzioni sugli operatori, pag 15.5; insieme alle cose a fine capitolo DnC

\subsubsection{Esempio formula}
Esempio, pag 16.5

\subsubsection{Esempio formula}
Esempio, pag 17.5

\section{\textit{Master theorem}}
Ipotesi, pag 18

Tesi, pag 18.5

Dimostrazione, pag 20

Considerazioni sull'asintotico, pag 18.9, 19

\subsection{autocomplete}
Una bella scatola:
\begin{equation}
    \boxed{x^2+y^2 = z^2}
\end{equation}

àg
èg
ìg
òg
ùg
