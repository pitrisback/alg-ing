\section{Contrazioni}
% definizione contrazioni, pag 13.5
\begin{definition}[Contrazione]\label{def:contraz}
    Una contrazione è una funzione $f: \mathbb{N} \rightarrow \mathbb{N}$, per cui vale
    \[ f(n) < n \quad \forall n>n_0 \]
\end{definition}

\subsection{Iterate}
% definizione iterata, pag 13.6
\begin{definition}[Iterata]\label{def:iterata}
    Ad una contrazione sono associate le iterate di $f(n)$
    \[
    \begin{cases} 
        f^{(0)} (n) = n      &  i = 0 \\
        f^{(i)} (n) = f \left( f^{(i-i)} (n) \right)   &  i > 0 \\
        
    \end{cases}
    \]
\end{definition}

Nota: le iterate formano una successione decrescente $ f^{(0)} (n) > f^{(1)} (n) > \cdots > f^{(i)} (n) $ \\
Nella prossima sezione, l'iterata $0$ sarà associata alla radice dell'abero e all'istanza generale, l'iterata $i-esima$ rappresenterà la taglia al livello $i$ dell'albero.

\subsection{Ausiliaria}
% Def ausiliaria, pag 14.6
\begin{definition}[Ausiliaria]\label{def:ausiliaria}
    Alle iterate di una funzione si associa anche una funzione ausiliaria, che indica il maggior indice di iterata per cui il valore è ancora maggiore del valore di base. Nell'albero delle ricorrenze, questo indicherà l'ultimo livello.
    \[ f^*(n, n_0) = \max \{ i>0 : f^{(i)}(n) > n_0 \} \]
    La funzione è definita solo per $n>n_0$, per convenzione assume il valore $f^*(n,n_0)=-1$ se $ n \leq n_0$
\end{definition}

\subsection{Esempi}
\subsubsection{$\bm{f(n) = n/2}$}
% pag 13.9, 14.8
Calcoliamo la forma esplicita dell'iterata e la funzione ausiliaria per
\[ f(n) = \frac{n}{2} \quad \text{con} \quad n=2^k \]
Forma esplicita iterata:
\begin{align*}
    & f^{(0)}(n) = n \\
    & f^{(1)}(n) = f(n) = \frac{n}{2} \\
    & f^{(2)}(n) = f \left( f^{(1)}(n) \right) = f \left( \frac{n}{2} \right) = \frac{n}{4} = \frac{n}{2^2} \\
    & f^{(i)}(n) = \frac{n}{2^i} 
\end{align*}
Dove la generalizzazione  è valida perché gli argomenti restano potenze di due: $n/2^i = 2^{k-i}$ \\
Calcolo funzione ausiliaria:
\begin{align*}
    & f^{(i)}(n) \mgeq n_0 
        & \text{per quali $i\:$?}\\
    & \frac{n}{2^i} \mges n_0 \\
    & 2^i \mles \frac{n}{n_0} \\
    & i < \log_2 \left( \frac{n}{n_0} \right)
        & \text{il vincolo è risolto} \\
    \rightarrow \quad & f^*(n, n_0) = \log_2 \left( \frac{n}{n_0} \right) - 1
        & \text{ne prendo il massimo} \\
    n_0 = 1 \rightarrow \quad & f^*(n, 1) = \log_2 \left( n \right) - 1
        & \text{se $n_0=1$}
\end{align*}

\subsubsection{$\bm{f(n) = n-1}$}
% pag 14, 15
Calcoliamo la forma esplicita dell'iterata e la funzione ausiliaria per
\[ f(n) = n-1 \]
Forma esplicita iterata:
\begin{align*}
    & f^{(0)}(n) = n \\
    & f^{(1)}(n) = f(n) = n-1 \\
    & f^{(2)}(n) = f \left( f^{(1)}(n) \right) = f \left( n-1 \right) = n-1-1 = n-2 \\
    & f^{(i)}(n) = n-i
\end{align*}
Calcolo funzione ausiliaria:
\begin{align*}
    & f^{(i)}(n) \mgeq n_0 
        & \text{per quali $i\:$?}\\
    & n-i \mges n_0 \\
    & i < n - n_0
        & \text{il vincolo è risolto} \\
    \rightarrow \quad & f^*(n, n_0) = n - n_0 - 1
        & \text{ne prendo il massimo} \\
    n_0 = 1 \rightarrow \quad & f^*(n, 1) = n - 2
        & \text{se $n_0=1$}
\end{align*}

\subsubsection{$\bm{f(n) = \sqrt{n}}$}
% pag 14.2, 15.2
Calcoliamo la forma esplicita dell'iterata e la funzione ausiliaria per
\[ f(n) = \sqrt{n} \quad \text{con} \quad n=2^{2^k} \]
Forma esplicita iterata:
\begin{align*}
    & f^{(0)}(n) = n \\
    & f^{(1)}(n) = f(n) = \sqrt{n} = n^{\nicefrac{1}{2}} \\
    & f^{(2)}(n) = f \left( f^{(1)}(n) \right) = f \left( n^{\nicefrac{1}{2}} \right) = n^{\nicefrac{1}{2^2}} \\
    & f^{(i)}(n) = n^{\nicefrac{1}{2^i}} \\
\end{align*}
Calcolo funzione ausiliaria:
\begin{align*}
    & f^{(i)}(n) \mgeq n_0 
        & \text{per quali $i\:$?}\\
    & n^{\nicefrac{1}{2^i}} \mges n_0 \\
    & \log_2 n^{\nicefrac{1}{2^i}} \mges \log_2 n_0 \\
    & \frac{1}{2^i} \log_2 n \mges \log_2 n_0 \\
    & 2^i \mles \frac{\log_2 n}{\log_2 n_0} \\
    & i < \log_2 \frac{\log_2 n}{\log_2 n_0} 
        & \text{il vincolo è risolto} \\
    & i < \log_2 \log_2 n + \log_2 \log_2 n_0 \\
    \rightarrow \quad & f^*(n, n_0) = \log_2 \log_2 n + \log_2 \log_2 n_0 - 1
        & \text{ne prendo il massimo} \\
    n_0 = 2 \rightarrow \quad & f^*(n, 2) = \log_2 \log_2 n - 1
        & \text{se $n_0=2$}
\end{align*}
Il vincolo su $n$ può essere generalizzato a $n=a^{2^k}$, in questo caso $n_0 = a$

\subsubsection{$\bm{f(n) = \left\lfloor n/2 \right\rfloor}$}
Nel caso $f(n) = \left\lfloor n/2 \right\rfloor$, si può dimostrare che
$ f^{(2)}(n) = \left\lfloor \left\lfloor n/2 \right\rfloor / 2 \right\rfloor = \left\lfloor n / 2^2 \right\rfloor $
ma non è per niente banale.

\section{Modifica al \textit{Divide and Conquer}}
% titolo meno informativo della storia
\subsection{Metaalgoritmo}
Metaalgoritmo DnC modificato, pag 12.5

Parametri noti, pag 13

Nuova eq di ricorrenza, pag 13.3

\subsubsection{esempio parametri MAX}
parametri max, pag 13.5

\subsection{Equazione delle ricorrenze generica}
Alberone, pag 15.8

Formulone, pag 16

Da qualche parte le convenzioni sugli operatori, pag 15.5; insieme alle cose a fine capitolo DnC -> Appendice

\subsubsection{Esempio formula}
Esempio, pag 16.5

\subsubsection{Esempio formula}
Esempio, pag 17.5

\section{\textit{Master theorem}}
Ipotesi, pag 18

Tesi, pag 18.5

Dimostrazione, pag 20

Considerazioni sull'asintotico, pag 18.9, 19

\subsection{autocomplete}
Una bella scatola:
\begin{equation}
    \boxed{x^2+y^2 = z^2}
\end{equation}

àg
èg
ìg
òg
ùg
perché
