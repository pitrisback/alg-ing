\section{Notazione sulle matrici}
% notazione, pag 25.5

\section{Somma e sottrazione}
% sum sub, pag 25.8, 26

\section{Moltiplicazione di matrici}
\subsection{Definizione}
% definizione, pag 26.4
\subsection{Implementazione ricorsiva}
% moltiplicazione ricorsiva, pag 27.3, 28
\subsection{Algoritmo di \textit{Strassen}}
% moltiplicazione Strassen, pag 29

\subsection{autocomplete}
Una bella scatola:
\begin{equation}
    \boxed{x^2+y^2 = z^2}
\end{equation}

Numeri nei casi
\begin{numcases}{T(n)=}
    2^3 \label{escaso1} \\
    2^4 \label{escaso2} 
\end{numcases}

Sotto numeri
\begin{subnumcases}{T(n)=}
    2^3 \label{escaso3} \\
    2^4 
\end{subnumcases}

\begin{itemize}[noitemsep,topsep=0pt,parsep=0pt,partopsep=0pt]
    \item qualcosa
    \item[+] qualcosa
    \item[*] qualcosa
    \item[--] qualcosa
\end{itemize}
àg
èg
ìg
òg
ùg
perché

