\section{Introduzione - Notazione}
% notazione, pag 25.5
Saranno principalmente considerate matrici quadrate $\mathbb{R}^{n \times n}$, con taglia associata $n$, il numero di righe, o $N=n^2$, il numero di elementi.\\
Verranno considerate operazioni costose solo somme, sottrazioni, prodotti e divisioni tra scalari reali. A volte verrà posta particolare attenzione a prodotti e divisioni, che sono più computazionalmente intensi.\\
Una matrice $A$ è composta da elementi $a_{i,j}$ e si indica
\[ A = \left( a_{i,j} \right)_{1 \leq i,j \leq n} \]

\section{Somma e sottrazione}
% sum sub, pag 25.8, 26
La somma (e, in tutta la sezione, in modo analogo la sottrazione) è una delle operazioni definita fra le matrici.
\begin{definition}
    La somma fra matrici è definita come
    \[
    C = A + B \rightarrow
    c_{i,j} = a_{i,j} + b_{i,j}
    \]
    \label{def:somma}
\end{definition}
\begin{algorithm}[H]
\caption{Somma tra matrici}\label{alg:sum}
\begin{algorithmic}[1]
\Procedure{SUM}{$A,B$}
    \State $n \gets A.rows$
        \For{$i \gets 1 $ to $ n $ }
            \For{$j \gets 1 $ to $ n $ }
                \State $c_{i,j} \gets a_{i,j} + b_{i,j} $
            \EndFor
        \EndFor
    \State return $C$
\EndProcedure
\end{algorithmic}
\end{algorithm}
La correttezza dell'algoritmo discende in maniera diretta dalla definizione.

La complessità dell'algoritmo è $T(n) = n^2$

\section{Moltiplicazione di matrici}
\subsection{Definizione}
% definizione, pag 26.4
La moltiplicazione tra matrici è definita secondo il prodotto righe per colonne.
\begin{definition}
    Il prodotto righe per colonne è definito come
    \[
    C = A \times B \rightarrow
    c_{i,j} = \sum_{k=1}^{n}a_{i,k} b_{k,j}
    \]
    \label{def:prodrc}
\end{definition}
\begin{algorithm}[H]
\caption{Prodotto righe per colonne secondo la definizione}\label{alg:mulrcdef}
\begin{algorithmic}[1]
\Procedure{MUL}{$A,B$}
    \State $n \gets A.rows$
    \For{$i \gets 1 $ to $ n $ }
        \For{$j \gets 1 $ to $ n $ }
            \State $c_{i,j} \gets a_{i,1} \cdot b_{1,j} $
            \For{$k \gets 2 $ to $ n $ }
                \State $c_{i,j} \gets c_{i,j} + a_{i,k} \cdot b_{k,j} $
            \EndFor
        \EndFor
    \EndFor
    \State return $C$
\EndProcedure
\end{algorithmic}
\end{algorithm}
La correttezza dell'algoritmo discende in maniera diretta dalla definizione.

La complessità dell'algoritmo è $T(n) = \Theta (n^3)$, infatti trasformando i cicli $for$ in sommatorie, si ottiene la formula
\[
    T(n) = \sum_{i=1}^{n} \sum_{j=1}^{n} \left( 1 + \sum_{k=2}^{n} 2 \right) = 
    = \sum_{i=1}^{n} \sum_{j=1}^{n} \left( 2n-2+1 \right) 
    = \left( 2n-1 \right) \sum_{i=1}^{n} \sum_{j=1}^{n} 1 
    = (2n-1)n^2 = 2n^3 -n^2
\]

\subsection{Implementazione ricorsiva}
% moltiplicazione ricorsiva, pag 27.3, 28
Nel corso del capitolo ci limiteremo a studiare matrici di taglia $n=2^k$, semplificando notevolmente l'analisi asintotica. Questo non comporta una perdita di generalità, infatti $\forall m \neq 2^k$ è sufficiente scegliere $n=2^{\left\lceil \log_2 m \right\rceil} < 2m$ e si può procedere con gli algoritmi che saranno presentati, avendo al più quadruplicato il numero di elementi.

TODO disegni delle matrici pag 27.5

Il prodotto righe per colonne ha la proprietà frattale, ovvero lo si può risolvere dal prodotto righe per colonne dei suoi blocchi.
% TODO l'italiano quà è brutto forte
Infatti dividendo le matrici in blocchi di dimensione $\frac{n}{2} \times \frac{n}{2}$ vale
\[
    1 \leq r \leq 2,\, 1 \leq s \leq 2 \rightarrow C_{r,s} = A_{r,1} B_{1,s} + A_{r,2} B_{2,s}
\]

TODO divisione in blocchi di C, pag 27.6

TODO altri disegni pag 27.9

La proprietà frattale descritta, che permette di ottenere il prodotto righe per colonne di due matrici $n \times n$ tramite prodotti righe per colonne di matrici $\frac{n}{2} \times \frac{n}{2}$ che poi vengono combinati è una proprietà di sottostruttura. Il caso di base è semplicemente il prodotto di due scalari. Quando $n>1$ si calcolano ricorsivamente tutti i prodotti delle sottomatrici.

\begin{algorithm}[H]
\caption{Prodotto righe per colonne ricorsivo}\label{alg:mulrcric}
\begin{algorithmic}[1]
\Procedure{RMUL}{$A, B$}
    \State $n \gets A.rows$
    \If{$n = 1$}                             \Comment{BASE}
        \State return $\left( a_{11} \cdot b_{11} \right)$
    \EndIf
    \State *divido $A, B$ in blocchi $A_{11}, \cdots, B_{22}$*    \Comment{DIVIDE}
    \For{$r \gets 1 $ to $ 2 $ }                                    \Comment{RECURSE e CONQUER sono simultanei}
        \For{$s \gets 1 $ to $ 2 $ }
        \State $C_{r,s} \gets 
        \Call{SUM}{}
        % \Call{SUM}{ % boh si offende un sacco
        \left(
            \Call{RMUL}{A_{r,1}, B_{1,s}}, 
            \Call{RMUL}{A_{r,2}, B_{2,s}} 
        \right)
        $
        % } $
        \EndFor
    \EndFor
    \State return $C$
    \EndProcedure
\end{algorithmic}
\end{algorithm}
Osservando che il \textit{Divide} non comporta alcun costo, il \textit{Recurse} comporta $2$ chiamate ricorsive per ciascuna delle quattro iterazioni dei cicli $for$ e il \textit{Conquer} necessita di $4$ somme su matrici di dimensione $\frac{n}{2} \times \frac{n}{2}$, si può scrivere l'equazione di ricorrenza
\begin{equation}
    T(n) = 
    \begin{cases}
        1 & n=1 \\
        8 T \left( \frac{n}{2} \right) + 4 \left( \frac{n}{2} \right)^2 & n>1
    \end{cases}
    \label{eq:rcric}
\end{equation}
con parametri
\begin{align*}
    a&=8 & b&=2 & w(n)&=n^2 & n^{log_2 8}&=n^3
\end{align*}
per cui siamo nel caso uno del \textit{Master Theorem}, e risulta
\begin{equation*}
    T(n) = \Theta\left( n^{log_2 8} \right) = \Theta\left( n^{3} \right)  
\end{equation*}
TODO applica per ESERCIZIO la formula generale per ricavare le costanti $2n^3-n^2$

\subsection{Algoritmo di \textit{Strassen}}
% moltiplicazione Strassen, pag 29
Il \textit{Master Theorem}, oltre ad essere strumento di analisi, può essere usato per la progettazione di algoritmi. Si può notare dall'equazione \ref{eq:rcric} che il costo delle chiamate ricorsive è molto maggiore rispetto al costo di ricombinazione. Cercando un algoritmo che riduca il primo, eventualmente a scapito del secondo, si può provare ad abbassare la complessità dell'algoritmo.

L'algoritmo di \textit{Strassen} segue proprio questa strada, mostrando una maniera per ottenere il prodotto di due matrici $n \times n$ usando solo $7$ prodotti di matrici $\frac{n}{2} \times \frac{n}{2}$ e un numero costante($18$) di somme e sottrazioni tra matrici $\frac{n}{2} \times \frac{n}{2}$.

\begin{algorithm}[H]
    \caption{Algoritmo di \textit{Strassen}}\label{alg:strassen}
\begin{algorithmic}[1]
% \Procedure{$D_S$}{$A, B$}
\Procedure{D\textsubscript{S}}{$A, B$}
    \State $A_1 \gets A_{11}$; $B_1 \gets \Call{SUB}{B_{12}, B_{22} }$
    \State $A_2 \gets \Call{SUM}{A_{11}, A_{12}}$; $B_2 \gets B_{22}$
    \State $A_3 \gets \Call{SUM}{A_{21}, A_{22}}$; $B_3 \gets B_{11}$
    \State $A_4 \gets A_{22}$; $B_4 \gets \Call{SUB}{B_{21}, B_{11} }$
    \State $A_5 \gets \Call{SUM}{A_{11}, A_{22}}$; $B_5 \gets \Call{SUM}{B_{11}, B_{22} }$
    \State $A_6 \gets \Call{SUM}{A_{12}, A_{22}}$; $B_6 \gets \Call{SUM}{B_{21}, B_{22} }$
    \State $A_7 \gets \Call{SUB}{A_{11}, A_{21}}$; $B_7 \gets \Call{SUB}{B_{11}, B_{12} }$
    \State return $A_1, \cdots, A_7, B_1, \cdots, B_7$
\EndProcedure
\Procedure{R\textsubscript{S}}{$A_1, \cdots, A_7, B_1, \cdots, B_7$}
    \For{$i \gets 1 $ to $ 7 $ }
    \State $P_i \gets \Call{SMUL}{A_{i}, B_{i} }$
    \EndFor
    \State return $P_1, \cdots, P_7$
\EndProcedure
\Procedure{C\textsubscript{S}}{$P_1, \cdots, P_7$}
    \State $C_{11} \gets \Call{SUB}{} \left( \Call{SUM}{P_4, P_5}, \Call{SUB}{P_2, P_6} \right) $
    \State $C_{12} \gets \Call{SUM}{P_1, P_2}$
    \State $C_{21} \gets \Call{SUM}{P_3, P_4}$
    \State $C_{22} \gets \Call{SUB}{} \left( \Call{SUM}{P_1, P_5}, \Call{SUB}{P_3, P_7} \right) $
    \State return $C_1, \cdots, C_4$
\EndProcedure
\Procedure{SMUL}{$A, B$}
    \State $n \gets A.rows$
    \If{$n = 1$}                             \Comment{BASE}
        \State return $\left( a_{11} \cdot b_{11} \right)$
    \EndIf
    \State $\left( A_1, \cdots, A_7, B_1, \cdots, B_7 \right) \gets \Call{D\textsubscript{S}}{A, B}$
    \Comment{DIVIDE}
    \State $\left( P_1, \cdots, P_7 \right) \gets \Call{R\textsubscript{S}}{A_1, \cdots, A_7, B_1, \cdots, B_7}$
    \Comment{RECURSE}
    \State $C  \gets \Call{C\textsubscript{S}}{P_1, \cdots, P_7}$
    \Comment{CONQUER}
    \State return $C$
    \EndProcedure
\end{algorithmic}
\end{algorithm}

La correttezza dell'algoritmo si prova mostrando la correttezza dei singoli blocchi di $C$, per esempio per $C_{12}$
\begin{equation*}
    C_{12} = P_1+P_2 = A_{11} \left( B_{12}-B_{22} \right) + \left( A_{11}+A_{12} \right) B_{22} =
    A_{11}B_{12} - \cancel{A_{11}B_{22}} + \cancel{A_{11}B_{22}} + A_{12}B_{22}
\end{equation*}

L'equazione di ricorrenza associata all'algoritmo è
\begin{equation}
    T(n) = 
    \begin{cases}
        1 & n=1 \\
        7 T \left( \frac{n}{2} \right) + 18 \left( \frac{n}{2} \right)^2 & n>1
    \end{cases}
    \label{eq:ricstrassen}
\end{equation}
per cui la funzione di soglia è diventata $n^{log_2 7} \approx n^{2.80}$, si è ancora nel caso uno e vale
\begin{equation*}
    T_S(n) = \Theta \left( n^{log_2 7} \right)
\end{equation*}
TODO verifica per ESERCIZIO il valore alle costanti, deve venire $T_S(n)=7n^{log_2 7}-6n^2$

Per quanto l'esponente sia diminuito rispetto all'implementazione ricorsiva base, è aumentato il coefficiente da $2$ a $7$. Per valori piccoli di $n$ quindi, la complessità dell'algoritmo di \textit{Strassen} è maggiore, e solo per $n>1024$ diventa conveniente.

\subsection{Algoritmo ibrido}
Si può notare come l'algoritmo di Strassen sia asintoticamente veloce, ma con costanti elevate, mentre l'algoritmo diretto sia asintoticamente più lento, ma abbia costanti minori. Utilizzando la tecnica dell'ibridizzazione, si possono combinare i due algoritmi, mantenendo l'andamento asintotico del primo e diminuendo le costanti.

Il primo approccio, immediato, si limita a selezionare uno dei due algoritmi in base alla taglia della matrice


\begin{algorithm}[H]
    \caption{Algoritmo ibrido naive}\label{alg:mulrcnaive}
\begin{algorithmic}[1]
\Procedure{Naive\_Hybrid\_MUL}{$A, B$}
    \State $n \gets A.rows$
    \If{$n < 1024 $}
        \State return $\Call{MUL}{A,B}$
        \Else
        \State return $\Call{SMUL}{A,B}$
    \EndIf
    \EndProcedure
\end{algorithmic}
\end{algorithm}

L'equazione di ricorrenza associata all'algoritmo è
\begin{equation*}
    T_{NH}(n) = 
    \begin{cases}
        2n^3 - n^2 & n<1024 \\
        7 n^{log_2 7} - 6n^2& n \geq 1024
    \end{cases}
    \label{eq:ricnaivehybrid}
\end{equation*}
e si può notare non è che sia cambiata molto, in particolare per $n \geq 1024 $ è rimasta identica.

Innestiamo quindi in modo più efficace l'algoritmo veloce, modificando il caso base di \textit{Strassen}, cambiando $n_0$ e bloccando prima la ricorsione.

TODO albero pag 31.7

\begin{algorithm}[H]
    \caption{Algoritmo ibrido}\label{alg:mulrcibrido}
\begin{algorithmic}[1]
\Procedure{Hybrid\_SMUL}{$A, B$}
    \State $n \gets A.rows$
    \If{$n \leq n_0 $}
        \State return $\Call{MUL}{A,B}$
        \Comment{BASE}
    \EndIf
    \State $\left( A_1, \cdots, A_7, B_1, \cdots, B_7 \right) \gets \Call{D\textsubscript{S}}{A, B}$
    \Comment{DIVIDE}
    \State $\left( P_1, \cdots, P_7 \right) \gets \Call{R\textsubscript{HS}}{A_1, \cdots, A_7, B_1, \cdots, B_7}$
    \Comment{RECURSE}
    \State $C  \gets \Call{C\textsubscript{S}}{P_1, \cdots, P_7}$
    \Comment{CONQUER}
    \State return $C$
    \EndProcedure
\end{algorithmic}
\end{algorithm}

Nota:
\begin{itemize}[noitemsep,topsep=0pt,parsep=0pt,partopsep=0pt]
    \item[--] nell'algoritmo \ref{alg:mulrcibrido} la chiamata ricorsiva è stata modificata, in modo da chiamare l'algoritmo ibrido
    \item[--] l'analisi di correttezza va modificata, la base è valida per la correttezza di MUL, il resto dell'induzione è analogo
\end{itemize}

L'equazione di ricorrenza associata all'algoritmo è
\begin{equation}
    T(n, n_0) = 
    \begin{cases}
        2n^3 - n^2 & n<n_0 \\
        7 T \left( \frac{n}{2}, n_0 \right) + \frac{9}{2}n^2 & n \geq n_0
    \end{cases}
    \label{eq:richybrid}
\end{equation}

Risolvendo l'equazione ricorsiva, si ottiene una complessità in funzione di $n_0$ e si può minimizzare la funzione

Utilizzando la formula generale \ref{eq:formachiusaric}, con $n=2^k$ e scegliendo $n_0 = 2^k$, si ricava per il caso base
\begin{align*}
    T_0 &= 2 n_0^3 - n_0^2
    \intertext{e per l'ultimo livello prima dei nodi foglia $f^*\left( n, n_0 \right)$}
    l \text{ generale } &\rightarrow \text{ taglia } \frac{n}{2^l} \text{ che pongo uguale a } n_0 \\
    \frac{n}{2^l} &= n_0 \rightarrow 2^l = \frac{n}{n_0} \rightarrow l = log_2 \left( \frac{n}{n_0} \right)
    \intertext{come livello delle foglie, per cui}
    f^*\left( n, n_0 \right) &= log_2 \left( \frac{n}{n_0} \right) -1
    \intertext{che si comporta come previsto, al crescere di $n_0$ il numero di livelli diminuisce; ricordando inoltre}
    S(n) = 7 \quad f(n) &= \frac{n}{2} \quad w(n) = \frac{9}{2} n^2
\end{align*}
risulta
\begin{align*}
    T(n, n_0) &=
    \sum_{l=0}^{ log_2 \left( \frac{n}{n_0} \right) -1} 7^l \frac{9}{2} \left( \frac{n}{2^l} \right)^2
    + 7^{log_2 \frac{n}{n_0}} \left( s n_0^3 - n_0^2 \right) \\
    &= \ldots \text{TODO ESERCIZIO} \\
    &= \left( \frac{2n_0+5}{n_0^{log_2 7}-2} \right)n^{log_2 7} - 6n^2
\end{align*}
Per cui la relazione con $n_0$ risulta solo nella costante moltiplicativa, e possiamo minimizzare $g(n_0)$, vincolato a $n_0 = 2^k$, in generale la derivata che si deve studiare è
\begin{equation}
    \frac{\partial}{\partial n_0} T(n, n_0)
\end{equation}

% \begin{align*}
\begin{gather*}
    \left( \frac{d}{d n_0} g(n_0) \right) n^{log_2 7} \quad \text{di cui studio il segno} \\
    \text{sign} \left( \frac{d}{d n_0} g(n_0) \right) 
    = \text{sign} \left( 2 n_0 - \left( 2n_0+5 \right)\left( log_2 7-2 \right) \right) \\
    % TODO parentesi più grandi
    \frac{d}{d n_0} g(n_0) \geq 0  \rightarrow n_0 \geq \frac{5\left( log_2 7 - 2 \right)}{2\left( 3-log_2 7 \right)} \approx 10,48
    \intertext{per cui il massimo vincolato a $n_0=2^k$ vale}
    n_0' = 8 \rightarrow g(8) \approx 3,92 \\
    n_0'' = 16 \rightarrow g(16) \approx 3,94 \\
% \end{align*}
\end{gather*}

L'equazione di ricorrenza associata all'algoritmo è
\begin{equation}
    T(n, 8) 
    \begin{cases}
        = 2n^3 - n^2 & n \leq 8 \\
        \approx 3,92 n^{log_2 7} + 6n^2 & n > 8
    \end{cases}
    \label{eq:richybrid8}
\end{equation}
Che risulta in quasi il $50\%$ di miglioramento. Notare come $8$ non abbia niente a che fare con il $1024$ che si era ottenuto in precedenza. Il metodo con cui è stato ricavato il valore ci assicura però che questo metodo sia sempre migliore di entrambi.

Nota: il modello di costo considerato non è sufficientemente descrittivo, e trascura tutti i costi associati alla ricorsione (chiamate ricorsive, allocazioni di memoria, salvataggio dello stato del programma\ldots) quindi nella realtà si consiglia di fare test empirici modificando il valore di $n_0$ per individuare il valore effettivamente migliore.
