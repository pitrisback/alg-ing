\section{Paradigma divide and conquer}

Seguendo il paradigma \textit{Divide and Conquer}, si cerca una soluzione ad una data istanza in funzione delle soluzioni a determinate istanze \textbf{più piccole}, dette sottoistanze.

\subsection{Stella di Kleene}
\begin{definition}[Stella di Kleene]\label{def:kleene}
    La stella di Kleene viene definita come l'insieme di tutte le sequenze finite di elementi di $A$.
    $$ A^* = \left\{ < a_1, a_2, \dots, a_k > \: : k \geq 0, \: a_i \in A, \: 1 \leq i \leq k \right\}$$
    La sequenza vuota viene indicata con $ < \; > \: = \epsilon $
\end{definition}
\subsection{Proprietà di sottostruttura}

Definizione di sottostruttura

Funzioni di Divide e di Conquer

Albero delle chiamate

Il \textit{merge}, cuore dell'algoritmo \ref{alg:mergesort}, avviene alla riga \ref{alg:mergeline}, quindi guarda a \algref{alg:mergesort}{alg:mergeline} per saperne di più.

\begin{algorithm}[h]
\caption{MergeSort}\label{alg:mergesort}
\begin{algorithmic}[1]
    \Procedure{MergeSort}{$A,p,r$}
    \If{$p<r$}
        \State $ q \gets \left\lfloor \frac{p+r}{2} \right\rfloor $ 
        \State \Call{MergeSort}{$A,p,q$} \Comment{Ordina prima}
        \State \Call{MergeSort}{$A,q+1,r$} \Comment{Ordina dopo}
        \State
            \Call{Merge}{$A,p,q,r$}
            \label{alg:mergeline}
            \Comment{Merge in $\Theta(n)$}
    \EndIf
    \EndProcedure
\end{algorithmic}
\end{algorithm}

\subsection{Correttezza del DnC}

Induzione e magia

Relazione di ricorrenza

\subsection{Analisi della complessità}

\subsubsection{autocomplete}

àg
èg
ùg
