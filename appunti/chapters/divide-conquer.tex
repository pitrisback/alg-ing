% \section{Paradigma divide and conquer}

Seguendo il paradigma \textit{Divide and Conquer}, si cerca una soluzione ad una data istanza in funzione delle soluzioni a determinate istanze \textbf{più piccole}, dette sottoistanze.

% \subsection{Stella di Kleene}
% FORMATTA potrebbe non essere una sezione
\section{Stella di Kleene}
\begin{definition}[Stella di Kleene]\label{def:kleene}
    La stella di Kleene viene definita come l'insieme di tutte le sequenze finite di elementi di $A$.
    $$ A^* = \left\{ < a_1, a_2, \dots, a_k > \: : k \geq 0, \: a_i \in A, \: 1 \leq i \leq k \right\}$$
    La sequenza vuota viene indicata con $ < \; > \: = \epsilon $
\end{definition}

% \subsection{Proprietà di sottostruttura}
\section{Proprietà di sottostruttura}

La proprietà di sottostruttura è una proprietà del problema $\bpi{} \subseteq \bi{} \times \bs{} $ che permette di definire il divide and conquer. Devono esistere due funzioni per avere la proprietà di sottostruttura:

\begin{description}
    \item{Funzione di \textbf{Divisione}} $ D : \bi{} \to \bi{}^*$
    \item{Funzione di \textbf{Ricombinazione}} $ C : \bs{}^* \to \bs{}$
\end{description}

E devono valere le due seguenti proprietà:
\begin{itemize}
    \item $\exists n_0 \in \mathbb{N} \cup \{0\} : \forall i \in \bi{}, |i| > n_0 $
        \begin{itemize}
            \item[--] $ D(i)=<i_1, i_2, \dots, i_k>:|i_j|<|i|, 1 \leq j \leq k$ \\
                (le sottoistanze sono strettamente più piccole dell'istanza originale)
            \item[--] data $ <s_1, s_2, \dots, s_k> \in \bs{}^* $ con $i_j \bpi{} s_j,  1 \leq j \leq k \Rightarrow i \bpi{} C(<s_1, s_2, \dots, s_k>)$ \\
                (la ricombinazione delle soluzioni alle sottoistanze è in relazione con l'istanza originale)
        \end{itemize}
    \item è possibile la risoluzione diretta delle istanze $ i \in \bi{} : |i| \leq n_0 $
\end{itemize}

Note:
% \begin{itemize}
\begin{itemize}[noitemsep,parsep=0pt,partopsep=0pt]
    \item[--] il numero di sottoistanze può variare, in base all'istanza considerata
    \item[--] gli algoritmi che implementano (con sequenze di istruzioni elementari) le funzioni, sono, formalmente, oggetti matematici diversi \\
        $ A_D : i \mapsto <i_1, i_2, \dots, i_k> \sim D(i)$ e $ A_C : <s_1, s_2, \dots, s_k> \mapsto s \sim C(<s_1, s_2, \dots, s_k>) $
\end{itemize}

% \subsubsection{Problema del Sorting di interi - definizione}
\subsection{Problema del Sorting di interi - definizione}

Introduciamo il problema dell'ordinamento di una sequenza di numeri interi, che verrà usato anche in seguito per esemplificare i concetti teorici esposti.

Come insiemi definiamo $\bi{} = \mathbb{Z}^* $ e $\bs{} = \mathbb{Z}^* $. Non tutti gli elementi di $\bs{}$ saranno associati ad un'istanza.

Quando $ (i,s) \in \bpi{}_{SORT} $ ? A parte il caso degenere $i=s=\epsilon$, deve esistere una corrispondenza biunivoca (permutazione) $\phi$ fra gli indici per cui
\begin{itemize}[noitemsep,parsep=0pt,partopsep=0pt]
    \item[--] $i = <i_1, i_2, \dots, i_k> $ e $ s = <s_1, s_2, \dots, s_k> $ con $k \geq 1$
    \item[--] $ \exists \;\phi:\{1, \dots, n\} \leftrightarrow \{1, \dots, n\} \;|\; s_i = i_{\phi(i)}$
    \item[--] $ s_1 \leq s_2 \leq \dots \leq s_k$ 
\end{itemize}

% \subsection{Paradigma Divide and Conquer}
\section{Prototipo Divide and Conquer}

\begin{algorithm}[H]
\caption{Divide and Conquer}\label{alg:dnc}
\begin{algorithmic}[1]
    \Procedure{D\&C}{$i$}
        \If{$|i| \leq n_0$}                             \Comment{BASE}
            \State *risolvo direttamente*
        \EndIf
        \State $<i_1, i_2, \dots, i_k> \gets A_D(i)$    \Comment{DIVIDE}
        \For{$j \gets 1 $ to $ k $ }                    \Comment{RECURSE}
            \State $s_j \gets $ \Call{D\&C}{$i_j$}
        \EndFor
        \State $s \gets A_C(<s_1, s_2, \dots, s_k>)$    \Comment{CONQUER}
        \State return $s$
    \EndProcedure
\end{algorithmic}
\end{algorithm}

% \subsubsection{Albero delle chiamate}
\subsection{Albero delle chiamate}
Nel corso del D\&C, si alternano chiamate all'algoritmo di Divide, e quindi una fase di espansione, generazione \textit{top-down} delle sottoistanze, e chiamate all'algoritmo di Conquer, associato a fasi di contrazione, risoluzione \textit{bottom-up}. L'albero non viene generato tutto contemporaneamente, ma viene creato (e distrutto) nel corso dell'algoritmo, seguendo il cammino di una visita anticipata (\textit{depth-first search}).

Il fatto che nella fase di Conquer venga eliminata la porzione di albero generato per risolvere una sottoistanza, è uno dei difetti principali di questo paradigma. Il non ricordare le soluzioni parziali trovate lo rende un processo computazionale \textit{memoryless}.

TODO pagina 4, albero delle chiamate, disegnini e grafichetti

% \subsubsection{Esempio algoritmo D\&C, ricerca del massimo}
\subsection{Esempio algoritmo D\&C: ricerca del massimo}

Come esempio di algoritmo D\&C, viene presentata una procedura che trova il massimo intero in una sequenza di interi $A[1 \dots n] \in \mathbb{Z}^*$

\begin{algorithm}[H]
\caption{Massimo}\label{alg:max}
\begin{algorithmic}[1]
    \Procedure{MAX}{$A,i,j$}                            
        \If{$i=j$}                             \Comment{BASE}
            \State return $A[i]$
        \EndIf
        \State $k \gets \left\lfloor \frac{i+j}{2} \right\rfloor$    \Comment{DIVIDE - punto di mezzo discreto}
        \State $m_1 \gets $ \Call{MAX}{$A,i,k$} \Comment{RECURSE}
        \State $m_1 \gets $ \Call{MAX}{$A,k+1,j$}
        \If{$m_1 \geq m_2$}              \Comment{CONQUER}
            \State return $m_1$
        \Else
            \State return $m_2$
        \EndIf
    \EndProcedure
\end{algorithmic}
\end{algorithm}

Nota: Nella firma della procedura, sono presenti tutti i parametri necessari a identificare la \textit{generica} sottostruttura

% \subsection{Correttezza del DnC}
\section{Correttezza del DnC}

La correttezza di un algoritmo Divide and Conquer viene provata con un'induzione sulla taglia delle istanze.

\begin{description}
    \item{\textbf{Base:}} provare la correttezza della soluzione diretta per $|i| \leq n_0$
    \item{\textbf{HP:}} formulare l'ipotesi induttiva, assunta come corretta per  $|i| < n$, con $n > n_0$
    \item{\textbf{TH:}} sfruttando l'ipotesi induttiva, dimostrare che la tesi è valida per $|i| = n$.
        La dimostrazione si articola in più parti, in cui viene anche provata la proprietà di sottostruttura.
        \begin{enumerate}[noitemsep,topsep=0pt,parsep=0pt,partopsep=0pt]
            \item $\forall i, D(i) = <i_1, \dots, i_k> \Rightarrow |i_j| < |i|$ \\
                verificare che il divide contragga davvero
            \item date $s_j \Rightarrow i_j \bpi{} s_j, 1 \leq j \leq k$ (vero per l'ipotesi induttiva)
            \item data $s = A_C (s_1, \dots, s_k) \Rightarrow i \bpi{} s$ \\
                correttezza del Conquer
        \end{enumerate}
\end{description}

\subsection{Esempio: correttezza dell'algoritmo MAX}

Per prima cosa va identificata la taglia di una generica sottoistanza in funzione dei parametri che la caratterizzano: $n=j-i+1$


\begin{description}
    \item{\textbf{Base:}} $n=1 \rightarrow j-i+1=1 \rightarrow j=i \rightarrow return \: A[i]$: corretto, $max\{a\}=a$
    \item{\textbf{HP:}} MAX viene considerato corretto per i casi $j-i+1 < n$
    \item{\textbf{TH:}} $j-i+1 = n$, $n>1$ e quindi anche $i<j$
        \begin{enumerate}[noitemsep,topsep=0pt,parsep=0pt,partopsep=0pt]
            \item sottoistanze di taglia minore e non vuote
                % $k= \left\lfloor \frac{i+j}{2}\right\rfloor$, $|i_1| = k-i+1$, $|i_2| = j-(k+i)+1 = j-k$
                % \begin{align*}
                    % k&= \left\lfloor \frac{i+j}{2}\right\rfloor & |i_1| &= k-i+1 \\
                     % &                                          & |i_2| &= j-(k+i)+1 = j-k
                % \end{align*}
                \begin{align*}
                    k&= \left\lfloor \frac{i+j}{2}\right\rfloor & |i_1| &= k-i+1  & |i_2| &= j-(k+i)+1 = j-k
                \end{align*}
                si possono scrivere maggiorazioni e minorazioni su $k$, sfruttando $i \leq j-1$
                \begin{gather*}
                    k= \left\lfloor \frac{i+j}{2}\right\rfloor \leq
                    \left\lfloor \frac{(j-1)+j}{2}\right\rfloor = 
                    \left\lfloor \frac{2j-1}{2}\right\rfloor \leq j-1 \\
                    i= \left\lfloor \frac{2i+1}{2}\right\rfloor
                    = \left\lfloor \frac{i+(i+1)}{2}\right\rfloor 
                    \leq \left\lfloor \frac{i+j}{2}\right\rfloor 
                    \leq k
                \end{gather*}
                da cui 
                \begin{equation*}
                    \begin{split}
                    i \leq &k \leq j-1 \\
                    i-i+1 \leq &k-i+1 \leq j-1-i+1 \\
                    0 < 1 \leq &k-i+1 \leq j-i < j-i+1 = n\\
                    0 < &|i_1| < n
                    \end{split}
                \end{equation*}
                in maniera analoga si dimostra $ 0 < j-k < n $ % ESERCIZIO
            \item 
                $m_1 = MAX(A,i,k) = max(\{A(x):i \leq x \leq k \}) $ \\
                $m_2 = MAX(A,k+1,j) = max(\{A(x):k+1 \leq x \leq j \}) $ \\
                considerati corretti per l'ipotesi induttiva
            \item sfruttando la proprietà $ A = A_1 \cup A_2 \Rightarrow max\{A\} = max\{max\{A_1\}, max\{A_2\}\}$,
                basta verificare che le due sottoistanze coprano tutto $A$: $A[i..j] = A[i..k] + A[k+1..j] $ %$(i,k)(k+1,j)$
        \end{enumerate}
\end{description}

\section{Analisi della complessità - Relazione di ricorrenza}
La relazione, o equazione, di ricorrenza è una relazione \textbf{implicita} su $T(n)$, che esprime $T(n)$ in funzione dei suoi valori assunti su valori del parametro più piccoli
\[
    T(n) = 
    \begin{cases} 
        T_0      &  n \leq n_0 \\
        T_D + T_R + T_C & n > n_0
    \end{cases}
\]
\begin{description}
    \item{$\bm{T_0}\:$} costo massimo istanze di base $i$, $|i| \leq n_0$
    \item{$\bm{T_D}$} costo associato alla fase di \textit{Divide}
    \item{$\bm{T_R}$} costo della ricorsione: $T_R = T(|i_1|) + \dots + T(|i_k|)$
    \item{$\bm{T_C}$} costo associato alla fase di \textit{Conquer}
\end{description}
Nota: $T(n)$ ha dominio sui naturali: $T : \mathbb{N} \rightarrow \mathbb{R}^{+} \cup \{0\}$

\subsection{Esempio: Relazione di ricorrenza per MAX}
Le uniche istruzioni elementari che vengono considerate nell'analisi sono i confronti tra chiavi. Tutte le altre istruzioni (assegnazioni, chiamate\ldots) hanno costo nullo. L'equazione di ricorrenza allora risulta
\[
    T_{MAX}(n) = 
    \begin{cases} 
        \overbracket[0.5pt]{
            \vphantom{\frac{n}{2}}
            \;0\: 
        }^\text{$T_0$}
        &  n = 1 \\
        \underbracket[0.5pt]{
            \vphantom{\frac{n}{2}}
            \:0\:
        }_\text{$T_D$}
        + 
        \underbracket[0.5pt]{
            T_{MAX} \left( \left\lfloor \frac{n}{2} \right\rfloor \right) 
            + T_{MAX} \left( \left\lceil \frac{n}{2} \right\rceil \right)
        }_\text{$T_R$}
        + 
        \underbracket[0.5pt]{
            \vphantom{\frac{n}{2}}
            \:1\:
        }_\text{$T_D$} 
        & n > 1
    \end{cases}
\]
Parti alte e basse introducono un'asimmetria che si amplifica sempre di più, si possono imporre delle condizioni sul valore iniziale per rimuoverle da \textit{tutti} i passi successivi.
\[
    T_{MAX}(n) = 
    \begin{cases} 
        0      &  n = 1 \\
        2 T_{MAX} \left( \frac{n}{2} \right) + 1 & n > 1, \exists k : n=2^k
    \end{cases}
\]
Risolvere l'equazione di ricorrenza in forma chiusa (definita esplicitamente fino alle costanti) è molto spesso estremamente difficile. In questi casi si cercano delle maggiorazioni o minorazioni che descrivano il comportamento asintotico nel modo più stretto possibile.\\
Nelle sezioni successive sono presentate tre tecniche per risolvere le relazioni di ricorrenza.

\subsection{Soluzione per \textit{unfolding}}
\begin{equation*}
    \begin{split}
        T_{MAX} &=
        2 T_{MAX} \left( \frac{n}{2} \right) + 1
        = 2 \left( 2 T_{MAX} \left( \frac{n}{4} \right) + 1 \right) + 1 \\
        &= 2^2 \:T_{MAX} \left( \frac{n}{2^2} \right) +2 +1
        = 2^2 \left( 2 T_{MAX} \left( \frac{n}{2^3} \right) + 1 \right) +2+1 \\
        &= 2^3 \:T_{MAX} \left( \frac{n}{2^3} \right) +2^2 +2 +1 = \cdots \\ 
        \intertext{passo di generalizzazione}
        \cdots &=2^i \:T_{MAX} \left( \frac{n}{2^i} \right) +2^i + \cdots +2^2 +2 +1 \\
        &= 2^i \:T_{MAX} \left( \frac{n}{2^i} \right) + \sum_{j=0}^{i-1} 2^j \\
        \intertext{elimino la dipendenza implicita quando il parametro è un valore base: $n / 2^i=1 \rightarrow i=\log_2 n$}
        &= 2^{\log_2 n} \cancel{ T_{MAX} \left( \cancel{\frac{n}{2^{\log_2 n}} } \right) } 
        + \sum_{j=0}^{\log_2 n-1} 2^j \\
        &= \frac{2^{\log_2 n}-1}{2-1} = n-1
    \end{split}
\end{equation*}

\subsection{Verifica per induzione}
Si può verificare per induzione la correttezza della soluzione trovata 

\begin{description}
    \item{\textbf{Base:}} $T(n)\left. \right|_{n=n_0} \meq T_0 \rightarrow 
            T(n)=n-1 \left. \right|_{n=1} \meq 0 \rightarrow \text{vero}$
    \item{\textbf{HP:}} $s < n$, sia vera $T(s) = s-1 \quad \forall s $ potenza di $2$
    \item{\textbf{TH:}} $T(n) = 2T \left( \frac{n}{2} \right) +1 \overset{\text{HP ind.}}{=} 
            2 \left( \frac{n-1}{2} \right)+1 = n-1$ che è la formula per $n$
\end{description}

\subsection{Soluzione per induzione parametrica}
\begin{enumerate}
    \item fomulare una \textit{guess} (limitazione superiore/inferiore) a $T(n)$ che sia espressa con costanti parametriche da determinare
    \item simulare la verifica per induzione della guess in modo da ottenere e raccogliere vincoli sulle costanti parametriche che permettano di completare con successo i passi dell'induzione
    \item cercare valori dei parametri che verifichino contemporaneamente tutte le condizioni accumulate. Se il sistema generato non ha soluzioni, la \textit{guess} era errata.
\end{enumerate}

\subsubsection{Esempio: Guess per MAX}
\textit{Guess}: $T(n) \leq cn+d$
\begin{description}
    \item{\textbf{Base:}} $T(1) = 0 \mleq cn+d \left. \right|_{n=1} = c+d \rightarrow
            c+d\geq 0 $ \circola{1}
    \item{\textbf{HP:}} $T(k) \leq ck+d \quad \forall k<n $ generico
    \item{\textbf{TH:}} 
        \begin{equation*}
            \begin{split}
                T(n) &=
                T \left( \left\lfloor \frac{n}{2} \right\rfloor \right) 
                + T \left( \left\lceil \frac{n}{2} \right\rceil \right) + 1 \\
                &\leq c \left\lfloor \frac{n}{2} \right\rfloor +d + c \left\lceil \frac{n}{2} \right\rceil +d + 1 \\
                &= c \left( \left\lfloor \frac{n}{2} \right\rfloor + \left\lceil \frac{n}{2} \right\rceil \right) +2d + 1 \\
                &= cn + 2d + 1 \mleq cn+d \\
                \rightarrow d &\leq -1 \quad \circola{2}
            \end{split}
        \end{equation*}
\end{description}
Combinando i vincoli \circola{1} e \circola{2} risultano possibili molte coppie di parametri. Si sceglie quella più restrittiva, che nel caso di una maggiorazione risulta essere il valore minimo della $c$, parametro della $n$ con esponente massimo.
% TODO grafico a pag 9.9
I valori trovati sono $c=1, d=-1$, ossia $T(n) \leq n-1$

\subsubsection{Esempio: Guess errata}
\textit{Guess}: $T(n) \leq cn$
\begin{description}
    \item{\textbf{Base:}} $T(1) = 0 \mleq cn \left. \right|_{n=1} = c \rightarrow c \geq 0 $ \circola{1}
    \item{\textbf{HP:}} $T(k) \leq ck \quad \forall k<n $
    \item{\textbf{TH:}} 
        \begin{equation*}
            \begin{split}
                T(n) &=
                T \left( \left\lfloor \frac{n}{2} \right\rfloor \right) 
                + T \left( \left\lceil \frac{n}{2} \right\rceil \right) + 1 \\
                &\leq c \left\lfloor \frac{n}{2} \right\rfloor + c \left\lceil \frac{n}{2} \right\rceil + 1 \\
                &= c \left( \left\lfloor \frac{n}{2} \right\rfloor + \left\lceil \frac{n}{2} \right\rceil \right) + 1 \\
                &= cn + 1 \mleq cn \\
                \rightarrow  1 &\leq 0 \quad \text{impossibile}
            \end{split}
        \end{equation*}
\end{description}

\subsubsection{Esempio: Guess search}
\[
    T(n) = 
    \begin{cases} 
        20      &  n \leq 10 \\
        T \left( \left\lceil \frac{n}{5} \right\rceil \right) + 
        T \left( \left\lfloor \frac{7n}{10} \right\rfloor \right) + 3n & n > 10
    \end{cases}
\]
\textit{Guess}: $T(n) \leq cn$
\begin{description}
    \item{\textbf{Base:}}
        $ T(n) = 20 \mleq cn \left. \right|_{1 \leq n \leq 10} $
        \begin{align*}
            20& \leq cn& \rightarrow c& \geq \frac{20}{n} \quad n=1,\cdots,10 \\
              &        & \rightarrow c& \geq 20 \quad \text{per} \quad n=1 \quad \circola{1} 
              % fa proprio schifo
        \end{align*}
    \item{\textbf{HP:}} $T(k) \leq ck \quad \forall k<n $ con $ n \geq 11 $
    \item{\textbf{TH:}} 
        \begin{equation*}
            \begin{split}
                T(n) &=
                T \left( \left\lceil \frac{n}{5} \right\rceil \right) + 
                T \left( \left\lfloor \frac{7n}{10} \right\rfloor \right) + 3n \\
                &\leq c \left\lceil \frac{n}{5} \right\rceil  + 
                c \left\lfloor \frac{7n}{10} \right\rfloor  + 3n \\
                & \leq c \left( \frac{n}{5}+1 \right) + c \frac{7n}{10} + 3n \\
                & \overset{\circola{A}}{=} c \left( \frac{n}{5} + \frac{7n}{10} \right) + c + 3n \\
                &= c \frac{9n}{10} + c + 3n \mleq cn\\
                \rightarrow 3n & \mleq \frac{c}{10}n -c \\
            \end{split}
        \end{equation*}
        % TODO grafico a pag 11.1
        risolvendo, si ottiene il vincolo $c \geq 330$ \\
        mettendo a sistema le due condizioni ottenute, si ottiene $T(n) \leq 330n$.\\
        Prima di \circola{A}, è il $+1$ che introduce problemi, maggioriamolo con $n \geq 11 \rightarrow 1 \leq n/11$
        \begin{equation*}
            \begin{split}
                \hphantom{T(n)} &
                \overset{\circola{A}}{\leq} c \left( \frac{n}{5}+\frac{n}{11} \right) + c \frac{7n}{10} + 3n \\
                &=c \left( \frac{22+10+77}{110}n \right) + 3n \\
                &= c \frac{109}{110} \cancel{n}+  3\cancel{n} \mleq c\cancel{n}\\
                \rightarrow 3 & \leq \frac{c}{110} \\
                \rightarrow c & \geq 330 \\
            \end{split}
        \end{equation*}
\end{description}

Verifica per induzione correttezza cn, pag 11.5 (ESERCIZIO)

In versione cn+d, pag 11.5 (ESERCIZIO)

\subsection{Soluzione con l'albero delle ricorrenze}
Complessità dall'albero delle ricorrenze
% forse nel prossimo capitolo

% \subsubsection{autocomplete}
\subsection{autocomplete}

Guarda che gustoso simbolino: $ \mleq{} $

\[
    a \mgeq{} b \mles{} c \mges{} d \meq{} 
\]

E tipiche funzioni

\[
    u(x) = 
    \begin{cases} 
        \exp{x} & \text{if } x \geq 0 \\
        1       & \text{if } x < 0
    \end{cases}
\]

Usa dcases se la matematica è alta, dcases* se le condizioni sono testo, serve pacchetto mathtools

Una bella scatola:
\begin{equation}
    \boxed{x^2+y^2 = z^2}
\end{equation}

Grande:

\fbox{
 \addtolength{\linewidth}{-2\fboxsep}%
 \addtolength{\linewidth}{-2\fboxrule}%
 \begin{minipage}{\linewidth}
  \begin{equation}
   x^2+y^2=z^2
  \end{equation}
 \end{minipage}
}

Multiriga sotto simbolone:

\begin{equation}
  \prod_{\substack{
            1\le i \le n\\
            1\le j \le m}}
     M_{i,j}
\end{equation}

Testo fra lista di equazioni:

\begin{minipage}{3in}
\begin{align*}
\intertext{If}
   A &= \sigma_1+\sigma_2\\
   B &= \rho_1+\rho_2\\
\intertext{then}
C(x) &= e^{Ax^2+\pi}+B
\end{align*} 
\end{minipage}


Testo fra lista di equazioni, senza minipage:

\begin{align*}
\intertext{If}
   A &= \sigma_1+\sigma_2\\
   B &= \rho_1+\rho_2\\
\intertext{then}
C(x) &= e^{Ax^2+\pi}+B
\end{align*} 

Il \textit{merge}, cuore dell'algoritmo \ref{alg:mergesort}, avviene alla riga \ref{alg:mergeline}, quindi guarda a \algref{alg:mergesort}{alg:mergeline} per saperne di più.

\begin{algorithm}[h]
\caption{MergeSort}\label{alg:mergesort}
\begin{algorithmic}[1]
    \Procedure{MergeSort}{$A,p,r$}
    \If{$p<r$}
        \State $ q \gets \left\lfloor \frac{p+r}{2} \right\rfloor $ 
        \State \Call{MergeSort}{$A,p,q$} \Comment{Ordina prima}
        \State \Call{MergeSort}{$A,q+1,r$} \Comment{Ordina dopo}
        \State
            \Call{Merge}{$A,p,q,r$}
            \label{alg:mergeline}
            \Comment{Merge in $\Theta(n)$}
    \EndIf
    \EndProcedure
\end{algorithmic}
\end{algorithm}

\begin{description}
    \item{Funzione di \textbf{Divisione}} $ D : \bi{} \to \bi{}^*$
    \item{Funzione di \textbf{Ricombinazione}} $ C : \bs{}^* \to \bs{}$
\end{description}

\begin{itemize}
    \item $\exists n_0 \in \mathbb{N} \cup \{0\} : \forall i \in \bi{}, |i| > n_0 $
        \begin{itemize}[noitemsep,topsep=0pt,parsep=0pt,partopsep=0pt]
            \item[--] $ D(i)=<i_1, i_2, \dots, i_k>:|i_j|<|i|, 1 \leq j \leq k$ \\
                (le sottoistanze sono strettamente più piccole dell'istanza originale)
            \item[--] data $ <s_1, s_2, \dots, s_k> \in \bs{}^* $ con $i_j \bpi{} s_j,  1 \leq j \leq k \Rightarrow i \bpi{} C(<s_1, s_2, \dots, s_k>)$ \\
                (la ricombinazione delle soluzioni alle sottoistanze è in relazione con l'istanza originale)
        \end{itemize}
    \item è possibile la risoluzione diretta delle istanze $ i \in \bi{} : |i| \leq n_0 $
\end{itemize}

àg
èg
ìg
òg
ùg

ede(scription) eit(emize) eeq(uation) eal(ign)
% TODO
% pag 7.1 formula sommatoria geometrica
% pag 9.1 formule algebra discreta
