\section{Introduzione}

Il corso vuole insegnare a progettare ed analizzare algoritmi efficienti.

\begin{description}
    \item{\textbf{Progetto}} ideare una strategia di risoluzione secondo paradigmi generali:
        \begin{itemize}
            \item divide and conquer
            \item dynamic programming
            \item greedy
        \end{itemize}
    \item{\textbf{Analisi}} si valuta la correttezza e la complessità degli algoritmi con prove matematiche
    \item{\textbf{Algoritmi}} si vedranno applicazioni notevoli, negli ambiti del calcolo matematico e la manipolazione di stringhe
    \item{\textbf{Efficienti}} la complessità degli algoritmi è legata alla teoria riguardante la NP-completezza dei problemi
\end{description}

\section{Problema computazionale}

\begin{definition}[Algoritmo]\label{def:alg}
    Un algoritmo è una procedura computazionale finita (terminante) e deterministica, specificata come una sequenza di passi elementari (istruzioni) estratte da un insieme standard associato a un modello computazionale (astrazione di un computer) che trasforma in maniera univoca un ingresso in un uscita.
\end{definition}


\begin{definition}[Problema computazionale]\label{def:probcomp}
    Un problema computazionale $\bpi{}$ è una relazione tra un insieme di istanze $\bi{}$ e un insieme di soluzioni $\bs{}$: $\bpi{} \subseteq \bi{} \times \bs{}$
    \\
    Un problema computazionale definisce la specifica astratta. % ma de che?
\end{definition}

Note:
% \begin{itemize}[noitemsep,topsep=0pt,parsep=0pt,partopsep=0pt]
\begin{itemize}[noitemsep,parsep=0pt,partopsep=0pt]
    \item[--] le seguenti notazioni sono usate in maniera equivalente: $i_1 \bpi{} s_1 \iff (i_1 , s_1 ) \in \bpi{}$
    \item[--] la relazione non è univoca, ad un'istanza possono essere associate più soluzioni
    \item[--] nella specifica astratta si assume esistano le soluzioni per ogni istanza: $\forall i \in \bi{},\exists s \in \bs{} | i \bpi{} s$
\end{itemize}

Un algoritmo $A_{\bpi{}}$ è un algoritmo per $\bpi{}$, ossia risolve $\bpi{}$ se, quando i suoi ingressi sono elementi di $\bi{}$, le sue uscite sono gli elementi di $\bs{}$ in relazione all'ingresso, formalmente:

$$ A: \bi{} \to \bs{} \quad \textrm{e} \quad A(i)=s \iff i \, \bpi{} \, s $$

L'algoritmo realizza in modo procedurale la relazione specificata in modo astratto dal problema computazionale, ossia risolve istanze di un problema computazionale. Più precisamente, un algoritmo è scritto per un modello di computazione, non lavora con istanze astratte ma con le loro codifiche.

\section{Analisi di correttezza e complessità}

\subsection{Analisi di correttezza}

Occorre provare il teorema di correttezza:
$$ \forall i \in \bi{} \: : \: i \, \bpi{} \, A(i) $$

\subsection{Analisi di complessità}

\subsubsection{Modello di costo}
Ad ogni istruzione elementare viene associato un costo. Scegliendo $c \in \mathbb{R} \cup \{ 0 \} $ l'analisi risulta eccessivamente complessa. I possibili costi vengono quindi limitati a $c \in \{ 0,1 \} $, assegnando $1$ solo alle operazioni che determinano effettivamente la complessità. Chiaramente vanno compiute scelte ragionevoli.

La complessità di eseguire l'algoritmo $A_{\bpi{}}$ su $i \in \bi{}$ si definisce come
$$t_{A_{\bpi{},i}}=\;somma\;dei\;costi\;associati\;alle\;istruzioni\;eseguite\;per\;ottenere\;A_{\bpi{}}(i)\;$$

\subsubsection{Taglia di un'istanza}
Calcolare la complessità per ogni singola istanza risulta nuovamente troppo complesso. L'insieme delle instanze viene quindi partizionato raggruppando tutte le istanze di taglia uguale.

\begin{definition}[Taglia di un'istanza]\label{def:taglia}
    La taglia di un'istanza è una misura intera, non negativa, ragionevole, della lunghezza associata a una qualche codifica ragionevole di una data istanza.\\
    $$|\cdot|:\bi{}\to \mathbb{N} \cup \{0\}$$
\end{definition}

Le istanze di taglia uguale vengono raccolte, partizionando l'insieme $\bi{}$
$$ \bi{}_{n} = \left\{ i \in \bi{} : |i|=n \right\} \quad n \geq 0$$ 

Si definisce una metrica sintetica, la funzione di complessità, che è associata alla taglia delle istanze.
$$ T_{A_{\bpi{}}} : \mathbb{N} \cup \{0\} \to \mathbb{R}^{+} \cup \{0\}$$

\textbf{Nota bene:} l'ingresso della funzione complessità deve essere intero.

Si possono definire diverse funzioni complessità:

\begin{align*}
\textrm{\textbf{Caso peggiore:}} &\quad T_{A_{\bpi{}}}^{WORST} (n) = \sup_{i \in \bi{}_{n}} \{t_{A_{\bpi{},i}} \} \\
\textrm{\textbf{Caso migliore:}} &\quad T_{A_{\bpi{}}}^{BEST} (n) = \inf_{i \in \bi{}_{n}} \{t_{A_{\bpi{},i}} \} \\
\textrm{\textbf{Caso medio:}} &\quad T_{A_{\bpi{}}}^{AVE} (n) = \mathbb{E} \left[t_{A_{\bpi{},i}} \right]
\end{align*}

