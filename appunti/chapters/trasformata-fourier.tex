\section{Rappresentazione di Polinomi}
% polinomi definizione 33.7
In generale, avere diverse rappresentazioni di un oggetto permette di eseguire operazioni diverse su rappresentazioni diverse, dove risulta più comodo. Chiaramente la rappresentazione è legata a come si possono fare le operazioni, e sono inoltre necessarie operazioni efficienti di conversione.

Introduciamo quindi un nuovo dominio applicativo, i polinomi, su cui verrà costruito l'algoritmo per la trasformata veloce di \textit{Fourier}.

\begin{definition}[Polinomio]
    Un polinomio è una funzione $p: \mathbb{C} \rightarrow \mathbb{C}$ definita su un'indeterminata e un insieme di coefficienti, come somma di monomi.
    \begin{equation*}
        p(x) = a_0 + a_1 x + a_2 x^2 + \dots + a_{n-1} x^{n-1}
    \end{equation*}
    \label{def:polinomio}
\end{definition}

\begin{definition}[Grado di un polinomio]
    Il grado di un polinomio è definito come l'indice massimo del coefficiente non nullo
    \begin{equation*}
        deg(p(x)) = \max \left\{ i: a_i \neq 0 \right\}
    \end{equation*}
    \label{def:poligrado}
\end{definition}

% grado limitato da n, pag 33.9
\begin{definition}[Polinomio di grado limitato da $n$]
    Un polinomio si dice di grado limitato da $n$ se il suo massimo grado può essere $n-1$
    \label{def:polilimitato}
\end{definition}

\subsection{Rappresentazione per coefficienti}
% rapp coeff 33.8
Un polinomio può essere rappresentato a $n$ coefficienti
\begin{equation*}
    p(x) \equiv \vec{a} \in \mathbb{C}^n
\end{equation*}
La rappresentazione può essere facilmente estesa con un'operazione di padding
\begin{equation*}
    p(x) \equiv \left( \vec{a}, 0_m \right) \in \mathbb{C}^{n+m}
\end{equation*}

\subsection{Rappresentazione per punti}
% forse non serve una subsubsection
\begin{theorem}[Teorema di interpolazione]
    Date $n$ coppie di punti $\left( x_i, y_i \right) \in \mathbb{C}^2 \text{ con }
    % x_i~\neq~x_j
    x_i \neq x_j
    % \forall~i~\neq~j
    \; \forall i \neq j
    \; \exists ! p(x)
    $
    di grado limitato da $n$ per cui
    $p(x_i) = y_i$
    detto polinomio interpolante.
    \label{teo:interpolazione}
\end{theorem}
C'è quindi una corrispondenza tra $n-uple$ di punti e un \emph{singolo} polinomio, quindi una $n-upla$ di punti è una rappresentazione di un polinomio di grado limitato da $n$
\begin{equation*}
    p(x) \equiv \left( \vec{x}, \vec{y}\right) \quad
    \vec{x}, \vec{y} \in \mathbb{C}^{n}, x_i \neq x_j \; \forall i \neq j
\end{equation*}
dove $\vec{x}$ si dice base della rappresentazione

Anche la rappresentazione per punti si può estendere
\begin{equation*}
    \left( \vec{x}, \vec{y}\right) \rightarrow
    \left( \vec{x}^E, \vec{y}^E\right) \quad
    \vec{x}^E, \vec{y}^E \in \mathbb{C}^{m}, x_i \neq x_j \; \forall i \neq j, \text{ con } m>n
\end{equation*}
Per estendere la rappresentazione occorre valutare il polinomio in $m-n$ punti aggiuntivi

\subsection{Conversione tra rappresentazioni}
Se si dispone della rappresentazione per coefficienti, è sufficiente valutare il polinomio in un certo numero di punti per ricavare la rappresentazione per punti, mentre se si dispone di una tabulazione occorre interpolare il polinomio.

\subsection{autocomplete}
Una bella scatola:
\begin{equation}
    \boxed{x^2+y^2 = z^2}
\end{equation}

Numeri nei casi
\begin{numcases}{T(n)=}
    2^3 \label{escaso1} \\
    2^4 \label{escaso2} 
\end{numcases}

Sotto numeri
\begin{subnumcases}{T(n)=}
    2^3 \label{escaso3} \\
    2^4 
\end{subnumcases}

\begin{itemize}[noitemsep,topsep=0pt,parsep=0pt,partopsep=0pt]
    \item qualcosa
    \item[+] qualcosa
    \item[*] qualcosa
    \item[--] qualcosa
\end{itemize}
àg
èg
ìg
òg
ùg
perché

